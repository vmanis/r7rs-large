\section{Title}\label{title}

Boxes

\section{Author}\label{author}

John Cowan

\section{Status}\label{status}

This SRFI is currently in ``final'' status. To see an explanation of
each status that a SRFI can hold, see
\href{http://srfi.schemers.org/srfi-process.html}{here}. To provide
input on this SRFI, please
\href{mailto:srfi\%20minus\%20111\%20at\%20srfi\%20dot\%20schemers\%20dot\%20org}{mail
to
\texttt{\textless{}srfi\ minus\ 111\ at\ srfi\ dot\ schemers\ dot\ org\textgreater{}}}.
See \href{../srfi-list-subscribe.html}{instructions here} to subscribe
to the list. You can access previous messages via
\href{mail-archive/maillist.html}{the archive of the mailing list}. You
can access post-finalization messages via
\href{http://srfi.schemers.org/srfi-111/post-mail-archive/maillist.html}{the
archive of the mailing list}.

\begin{itemize}
\tightlist
\item
  Received:
  \href{http://srfi.schemers.org/srfi-111/srfi-111-1.1.html}{2013/04/14}
\item
  Draft: 2013/04/25-2013/06/24
\item
  Revised:
  \href{http://srfi.schemers.org/srfi-111/srfi-111-1.2.html}{2013/5/18}
\item
  Final:
  \href{http://srfi.schemers.org/srfi-111/srfi-111-1.3.html}{2013/5/18}
\end{itemize}

\section{Abstract}\label{abstract}

Boxes are objects with a single mutable state. Several Schemes have
them, sometimes called \emph{cells}. A constructor, predicate, accessor,
and mutator are provided.

\section{Issues}\label{issues}

None in this draft.

\section{Rationale}\label{rationale}

A box is a container for an object of any Scheme type, including another
box. It is like a single-element vector, or half of a pair, or a direct
representation of state. Boxes are normally used as minimal mutable
storage, and can inject a controlled amount of mutability into an
otherwise immutable data structure (or one that is conventionally
treated as immutable). They can be used to implement call-by-reference
semantics by passing a boxed value to a procedure and expecting the
procedure to mutate the box before returning.

Some Scheme systems use boxes to implement \texttt{set!}. In this
transformation, known as \emph{assignment conversion}, all variables
that are actually mutated are initialized to boxes, and all
\texttt{set!} syntax forms become calls on \texttt{set-box!}. Naturally,
all ordinary references to those variables must become calls on
\texttt{unbox}. By reducing all variable mutation to data-structure
mutation in this way, such Scheme systems are free to maintain variables
in multiple hardware locations, such as the stack and the heap or
registers and the stack, without worrying about exactly when and where
they are mutated.

Boxes are also useful for providing an extra level of indirection,
allowing more than one body of code or data structure to share a
reference, or pointer, to an object. In this way, if any procedure
mutates the box in any of the data structures, all procedures will
immediately ``see'' the new value in all data structures containing it.

Racket and Chicken provide \emph{immutable boxes}, which look like boxes
to \texttt{box?} and \texttt{unbox} but which cannot be mutated. They
are not considered useful enough to be part of this SRFI. If they are
provided nevertheless, the recommended constructor name is
\texttt{immutable-box}.

The following table specifies the procedure names used by the Scheme
implementations that provide boxes:

\begin{longtable}[]{@{}llllllll@{}}
\toprule
Proposed &
\href{http://docs.racket-lang.org/reference/boxes.html}{Racket} &
\href{http://www.iro.umontreal.ca/~gambit/doc/gambit-c.html\#index-boxes}{Gambit}
& \href{http://sisc-scheme.org/manual/html/ch03.html\#Boxing}{SISC} &
\href{http://www.scheme.com/csug7/objects.html\#g50}{Chez} &
\href{http://wiki.call-cc.org/eggref/4/box}{Chicken} &
\href{http://web.mit.edu/scheme_v9.0.1/doc/mit-scheme-ref/Cells.html}{MIT}
&
\href{http://s48.org/1.1/manual/s48manual_42.html}{Scheme48/scsh}\tabularnewline
box & box & box & box & box & make-box & make-cell &
make-cell\tabularnewline
box? & box? & box? & box? & box? & box-mutable? & cell? &
cell?\tabularnewline
unbox & unbox & unbox & unbox & unbox & box-ref & cell-contents &
cell-ref\tabularnewline
set-box\texttt{!} & set-box\texttt{!} & set-box\texttt{!} &
set-box\texttt{!} & set-box\texttt{!} & box-set\texttt{!} &
set-cell-contents\texttt{!} & cell-set\texttt{!}\tabularnewline
\bottomrule
\end{longtable}

Note that Chicken also supports the procedure names defined by this SRFI
in addition to its native API. Although the native API uses the standard
naming pattern, for the purposes of this SRFI the unanimous voice of
Racket, Gambit, SISC, and Chez is considered more significant.

Racket, Gambit, SISC, Chez, and Chicken all support the lexical syntax
\texttt{\#\&}\emph{datum} to create a literal box (immutable in Racket,
mutable in the other implementations). However, this SRFI does not.
Incorporating a specifically mutable object into code makes it hard for
an implementation to separate compile-time and run-time reliably, and
it's not clear that providing them in data files provides enough
benefit, as the box has to be located by tree-walking in any case. MIT
Scheme and Scheme48/scsh do not provide a lexical syntax.

The features specified in the autoboxing section of specification are
based on those specified by RnRS for promises, which are analogous to
immutable boxes except that their value is specified by code instead of
data.

\section{Specification}\label{specification}

\subsection{Procedures}\label{procedures}

The following procedures implement the box type (which is disjoint from
all other Scheme types), and are exported by the \texttt{(srfi\ 111)}
library (or \texttt{(srfi\ :111)} on R6RS).

\texttt{(box\ value)}

Constructor. Returns a newly allocated box initialized to \emph{value}.

\texttt{(box?\ object)}

Predicate. Returns \texttt{\#t} if \emph{object} is a box, and
\texttt{\#f} otherwise.

\texttt{(unbox\ box)}

Accessor. Returns the current value of \emph{box}.

\texttt{(set-box!\ box\ value)}

Mutator. Changes \emph{box} to hold \emph{value}.

The behavior of boxes with the equivalence predicates \texttt{eq?},
\texttt{eqv?}, and \texttt{equal?} is the same as if they were
implemented with records. That is, two boxes are both \texttt{eq?} and
\texttt{eqv?} iff they are the product of the same call to \texttt{box}
and not otherwise, and while they must be \texttt{equal?} if they are
\texttt{eqv?}, the converse is implementation-dependent.

\section{Implementation}\label{implementation}

Here is a definition as an R7RS library:

\begin{verbatim}
(define-library (srfi 111)
  (import (scheme base))
  (export box box? unbox set-box!)
  
  (begin
    (define-record-type box-type
      (box value)
      box?
      (value unbox set-box!))))
\end{verbatim}

The \texttt{define-record-type} macro will also work on R5RS Schemes
that provide SRFI 9.

Here is a translation into R6RS Scheme:

\begin{verbatim}
(library (srfi :111)
  (export box box? unbox set-box!)
  (import (rnrs base) (rnrs records syntactic))
  
  (define-record-type
    (box-type box box?)
    (fields
      (mutable value unbox set-box!))))
\end{verbatim}

Finally, here is an implementation in pure R5RS (plus \texttt{error})
that depends on redefining \texttt{pair?}.

\begin{verbatim}
;; Prepare to redefine pair?.
(define %true-pair? pair?)

;; Unique object in the cdr of a pair flags it as a box.
(define %box-flag (string-copy "box flag"))

;; Predicate
(define (box? x) (and (%true-pair? x) (eq? (cdr x) %box-flag)))

;; Constructor
(define (box x) (cons x %box-flag))

;; Accessor
(define (unbox x)
  (if (box? x)
    (car x)
    (error "Attempt to unbox non-box")))
    
;; Mutator
(define (set-box! x y)
  (if (box? x)
    (set-car! x y)
    (error "Attempt to mutate non-box")))

;; Override pair?.
(set! pair?
  (lambda (x)
    (and (%true-pair? x) (not (box? x)))))
\end{verbatim}

Note that these implementations do not support the lexical syntax.

\subsection{Autoboxing (optional)}\label{autoboxing-optional}

The following provisions of this SRFI are optional:

\begin{itemize}
\item
  A procedure, whether system-provided or user-written, that expects a
  box as an argument but receives a non-box may, if appropriate,
  allocate a box itself that holds the value, thus providing autoboxing.
\item
  A procedure that accepts arguments only of specified types (such as
  \texttt{+}) but receives a box instead may, if appropriate, unbox the
  box. Procedures that accept arguments of any type (such as
  \texttt{cons}) must not unbox their arguments.
\item
  Calling \texttt{unbox} on a non-box may simply return the non-box.
\end{itemize}

\section{Copyright}\label{copyright}

Copyright (C) John Cowan 2013. All Rights Reserved.

Permission is hereby granted, free of charge, to any person obtaining a
copy of this software and associated documentation files (the
``Software''), to deal in the Software without restriction, including
without limitation the rights to use, copy, modify, merge, publish,
distribute, sublicense, and/or sell copies of the Software, and to
permit persons to whom the Software is furnished to do so, subject to
the following conditions:

The above copyright notice and this permission notice shall be included
in all copies or substantial portions of the Software.

THE SOFTWARE IS PROVIDED ``AS IS'', WITHOUT WARRANTY OF ANY KIND,
EXPRESS OR IMPLIED, INCLUDING BUT NOT LIMITED TO THE WARRANTIES OF
MERCHANTABILITY, FITNESS FOR A PARTICULAR PURPOSE AND NONINFRINGEMENT.
IN NO EVENT SHALL THE AUTHORS OR COPYRIGHT HOLDERS BE LIABLE FOR ANY
CLAIM, DAMAGES OR OTHER LIABILITY, WHETHER IN AN ACTION OF CONTRACT,
TORT OR OTHERWISE, ARISING FROM, OUT OF OR IN CONNECTION WITH THE
SOFTWARE OR THE USE OR OTHER DEALINGS IN THE SOFTWARE.

\begin{center}\rule{0.5\linewidth}{\linethickness}\end{center}

Editor:
\href{mailto:srfi-editors\%20at\%20srfi\%20dot\%20schemers\%20dot\%20org}{Mike
Sperber}

Last modified: Wed Jul 3 09:04:14 MST 2013
