\section{Title}\label{title}

Ephemerons

\section{Author}\label{author}

John Cowan

\section{Status}\label{status}

This SRFI is currently in \emph{final} status. Here is
\href{http://srfi.schemers.org/srfi-process.html}{an explanation} of
each status that a SRFI can hold.

To provide input on this SRFI, please send email to
\texttt{srfi-124@nospamsrfi.schemers.org}. To subscribe to the list,
follow \href{http://srfi.schemers.org/srfi-list-subscribe.html}{these
instructions}. You can access previous messages via the mailing list
\href{http://srfi-email.schemers.org/srfi-124}{archive}.

\begin{itemize}
\tightlist
\item
  Received: 2015/9/6
\item
  Draft \#1 published: 2015/9/6
\item
  Draft \#2 published: 2015/10/26
\item
  Draft \#3 published: 2015/10/29
\item
  Draft \#4 published: 2015/11/2
\item
  Draft \#5 published: 2015/11/4
\item
  Finalized: 2015/11/6
\end{itemize}

\section{Abstract}\label{abstract}

An ephemeron is an object with two components called its \emph{key} and
its \emph{datum}. It differs from an ordinary pair as follows: if the
garbage collector (GC) can prove that there are no references to the key
except from the ephemeron itself and possibly from the datum, then it is
free to \emph{break} the ephemeron, dropping its reference to both key
and datum. In other words, an ephemeron can be broken when nobody else
cares about its key. Ephemerons can be used to construct weak vectors or
lists and (possibly in combination with finalizers) weak hash tables.

Much of this specification is derived with thanks from the MIT Scheme
Reference Manual.

\section{Issues}\label{issues}

None at present.

\section{Rationale}\label{rationale}

Weak references are a mechanism for building data structures that point
at objects without protecting them from garbage collection. An example
of such a data structure might be an entry in a lookup table that should
be removed if the rest of the program does not reference its key. Such
an entry must still point at its key to carry out comparisons, but
should not in itself prevent its key from being garbage collected.

A weak reference is a reference that points at an object without
preventing it from being garbage collected. The term strong reference is
used to distinguish normal references from weak ones. If there is no
path of strong references to some object, the garbage collector will
reclaim that object and mark any weak references to it to indicate that
it has been reclaimed.

If there is a path of strong references from an object A to an object B,
A is said to hold B strongly. If there is a path of references from an
object A to an object B, but every such path traverses at least one weak
reference, A is said to hold B weakly.

Ephemerons, unlike simple weak pairs, allow the correct handling of data
structures in which the value points to the key. For example, a record
may have a single field holding the key; if it's desired to use these
records as values in a weak hash table using this internal key,
ephemerons are necessary or the table will not really be weak.

Ephemerons are considerably heavier-weight than simple weak pairs,
because garbage-collecting ephemerons is more complicated than
garbage-collecting weak pairs. In MIT Scheme, each ephemeron needs five
words of storage rather than the two words needed by a weak pair.
However, while the GC needs to spend more time on ephemerons than on
other objects, the amount of time it spends on ephemerons can be made to
scale linearly with the number of live ephemerons, which is how a
copying GC's running time scales with the total number of live objects
anyway.

\section{Specification}\label{specification}

\texttt{(ephemeron?\ }\emph{object}\texttt{)}

Returns \texttt{\#t} if \emph{object} is an ephemeron; otherwise returns
\texttt{\#f}.

\texttt{(make-ephemeron\ }\emph{key datum}\texttt{)}

Returns a newly allocated ephemeron, with components \emph{key} and
\emph{datum}. Note that if \emph{key} and \emph{datum} are the same in
the sense of \texttt{eq?}, the ephemeron is effectively a weak reference
to the object.

\texttt{(ephemeron-broken?\ }\emph{ephemeron}\texttt{)}

Returns \#t if \emph{ephemeron} has been broken; otherwise returns \#f.

This procedure must be used with care. If it returns \texttt{\#f}, that
guarantees only that prior evaluations of \texttt{ephemeron-key} or
\texttt{ephemeron-datum} yielded the key or datum that was stored in
\emph{ephemeron}. However, it makes no guarantees about subsequent calls
to \texttt{ephemeron-key} or \texttt{ephemeron-datum}, because the GC
may run and break the ephemeron immediately after
\texttt{ephemeron-broken?} returns. Thus, the correct idiom to fetch an
ephemeron's key and datum and use them if the ephemeron is not broken
is:

\begin{verbatim}
     (let ((key (ephemeron-key ephemeron))
           (datum (ephemeron-datum ephemeron)))
       (if (ephemeron-broken? ephemeron)
           ... broken case ...
           ... code using key and datum ...))
\end{verbatim}

\texttt{(ephemeron-key\ }\emph{ephemeron}\texttt{)}

\texttt{(ephemeron-datum\ }\emph{ephemeron}\texttt{)}

These return the key or datum component, respectively, of
\emph{ephemeron}. If \emph{ephemeron} has been broken, these operations
return \texttt{\#f}, but they can also return \texttt{\#f} if that is
what was stored as the key or datum.

\texttt{(reference-barrier\ }\emph{key}\texttt{)}

\emph{This procedure is optional.}

This procedure ensures that the garbage collector does not break an
ephemeron containing an unreferenced key before a certain point in a
program. The program can invoke a \emph{reference barrier} on the key by
calling this procedure, which guarantees that even if the program does
not use the key, it will be considered strongly reachable until after
\texttt{reference-barrier} returns.

\section{Implementation}\label{implementation}

Ephemerons are currently available in
\href{http://www.gnu.org/software/mit-scheme/documentation/mit-scheme-ref/Ephemerons.html}{MIT
Scheme}, in
\href{http://docs.racket-lang.org/reference/ephemerons.html}{Racket},
and in Chibi, though current versions of Chibi have a bug; the Chibi GC
will not break an ephemeron properly if its datum refers to its key. The
MIT version allows ephemeron keys and datums to be mutated using
\texttt{set-ephemeron-key!} and \texttt{set-ephemeron-datum!}, but this
SRFI does not require this capacity, as it is tricky to implement on
systems where the GC runs in a separate thread. The effect of mutable
ephemerons can be achieved using immutable ephemerons that hold
\href{http://srfi.schemers.org/srfi-111/srfi-111.html}{SRFI 111} mutable
boxes, provided care is taken to always point to the box and not to its
contents.

The implementation in \texttt{ephemerons-trivial.scm} does not have any
hooks to the GC, but it is still a correct implementation, because there
are no guarantees that the GC will ever break any ephemerons, or run at
all, or even exist. (Thanks to Will Clinger for this insight.) It is
portable to any R7RS-small or R5RS + SRFI 9 system.

The implementation in \texttt{ephemerons-racket.scm} layers this SRFI's
semantics on top of Racket's native ephemerons. The idea here is that
the native-level ephemeron value is a pair containing the key and the
datum, so that the key can be reliably retrieved and a broken ephemeron
can be distinguished from one whose key or value is \texttt{\#f}.
Reference barriers are not supported.

The files \texttt{ephemeron-hash.scm} and \texttt{ephemeron-tree.scm}
are by Taylor Campbell, and constitute a demonstration written in
pseudo-Scheme of how ephemerons can be implemented in full. He has
offered to provide assistance in transforming them into Pre-Scheme for
Scheme48.

Ephemerons with built-in finalizers are also available in GHC's
implementation of Haskell under the name of
\href{https://hackage.haskell.org/package/base-4.8.1.0/docs/System-Mem-Weak.html}{weak
pointers}.

\section{References}\label{references}

The original paper on ephemerons is Barry Hayes,
\href{https://static.aminer.org/pdf/PDF/000/522/273/ephemerons_a_new_finalization_mechanism.pdf}{``Ephemerons:
a New Finalization Mechanism''}, \emph{Object-Oriented Languages,
Programming, Systems, and Applications}, 1997.

A useful implementation paper is Bruno Haible,
\href{http://www.haible.de/bruno/papers/cs/weak/WeakDatastructures-writeup.html}{``Weak
References: Data Types and Implementation''} posted 2005-04-24.

\href{http://www.inf.puc-rio.br/~roberto/docs/ry08-06.pdf}{``Eliminating
Cycles in Weak Tables''} is about the addition of ephemeron tables to
Lua 5.2. It explains how Lua's tricolor mark-sweep GC is extended to
handle them, unfortunately using the O(N\^{}2) algorithm of Hayes 1997.

\href{https://www.cs.hmc.edu/~oneill/papers/Blobs-SPACE.pdf}{``Extending
Garbage Collection to Complex Data Structures''} extends ephemerons,
which have a fixed GC strategy, to a new data structure called
\emph{blobs} that explain to the GC how they are to be processed, thus
allowing an arbitrary number of key-like and value-like pointers.

\section{Verse}\label{verse}

Reclaimer, spare that tree!\\
Take not a single bit!\\
It used to point to me,\\
Now I'm protecting it.\\
It was the reader's \texttt{cons}\\
That made it, paired by dot;\\
Now, GC, for the nonce,\\
Thou shalt reclaim it not.

That old familiar tree,\\
Whose \texttt{cdr}s and whose \texttt{car}s\\
Are spread o'er memory ---\\
And wouldst thou it
\href{http://catb.org/jargon/html/P/parse.html}{unparse}?\\
GC, cease and desist!\\
In it no free list store;\\
Oh spare that \href{http://catb.org/jargon/html/M/moby.html}{moby}
list\\
Now pointing throughout core!

It was my parent tree\\
When it was circular;\\
It pointed then to me:\\
I was its \texttt{cadadr}.\\
My \texttt{cdr} was a list,\\
My \texttt{car} a dotted pair ---\\
That tree will sore be missed\\
If it remains not there.

And now I to thee point,\\
A saving root, old friend!\\
Thou shalt remain disjoint\\
From free lists to the end.\\
Old tree! The
\href{https://en.wikipedia.org/wiki/Tracing_garbage_collection\#Na.C3.AFve_mark-and-sweep}{sweep}
still brave!\\
And, GC,
\href{https://en.wikipedia.org/wiki/Tracing_garbage_collection\#Na.C3.AFve_mark-and-sweep}{mark}
this well:\\
While I exist to save,\\
Thou shan't reclaim one cell.

\begin{quote}
\href{https://en.wikipedia.org/wiki/Guy_L._Steele,_Jr.}{The Great Quux}
(with \href{http://www.bartleby.com/248/131.html}{apologies} to
\href{https://en.wikipedia.org/wiki/George_Pope_Morris}{George Pope
Morris})
\end{quote}

(Pedantic note: In Scheme, pairs constructed by \texttt{read} are
officially immutable, so this scenario is technically impossible.)

\section{Copyright}\label{copyright}

Copyright (C) John Cowan (2015).

Permission is hereby granted, free of charge, to any person obtaining a
copy of this software and associated documentation files (the
``Software''), to deal in the Software without restriction, including
without limitation the rights to use, copy, modify, merge, publish,
distribute, sublicense, and/or sell copies of the Software, and to
permit persons to whom the Software is furnished to do so, subject to
the following conditions:

The above copyright notice and this permission notice shall be included
in all copies or substantial portions of the Software.

THE SOFTWARE IS PROVIDED ``AS IS'', WITHOUT WARRANTY OF ANY KIND,
EXPRESS OR IMPLIED, INCLUDING BUT NOT LIMITED TO THE WARRANTIES OF
MERCHANTABILITY, FITNESS FOR A PARTICULAR PURPOSE AND NONINFRINGEMENT.
IN NO EVENT SHALL THE AUTHORS OR COPYRIGHT HOLDERS BE LIABLE FOR ANY
CLAIM, DAMAGES OR OTHER LIABILITY, WHETHER IN AN ACTION OF CONTRACT,
TORT OR OTHERWISE, ARISING FROM, OUT OF OR IN CONNECTION WITH THE
SOFTWARE OR THE USE OR OTHER DEALINGS IN THE SOFTWARE.

\begin{center}\rule{0.5\linewidth}{\linethickness}\end{center}

Editor:
\href{mailto:srfi-editors\%20at\%20srfi\%20dot\%20schemers\%20dot\%20org}{Arthur
A. Gleckler}
