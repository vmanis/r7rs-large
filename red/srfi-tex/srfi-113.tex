\section{Title}\label{title}

Sets and bags

\section{Author}\label{author}

John Cowan

\section{Status}\label{status}

This SRFI is currently in ``final'' status. To see an explanation of
each status that a SRFI can hold, see
\href{http://srfi.schemers.org/srfi-process.html}{here}. To provide
input on this SRFI, please
\href{mailto:srfi\%20minus\%20113\%20at\%20srfi\%20dot\%20schemers\%20dot\%20org}{mail
to
\texttt{\textless{}srfi\ minus\ 113\ at\ srfi\ dot\ schemers\ dot\ org\textgreater{}}}.
See \href{../srfi-list-subscribe.html}{instructions here} to subscribe
to the list. You can access previous messages via
\href{mail-archive/maillist.html}{the archive of the mailing list}. You
can access post-finalization messages via
\href{http://srfi.schemers.org/srfi-111/post-mail-archive/maillist.html}{the
archive of the mailing list}.

\begin{itemize}
\tightlist
\item
  Received:
  \href{http://srfi.schemers.org/srfi-113/srfi-113-1.1.html}{2013/05/31}
\item
  Draft: 2013/06/01-2013/08/01
\item
  Revised:
  \href{http://srfi.schemers.org/srfi-113/srfi-113-1.2.html}{2013/06/09}
\item
  Revised:
  \href{http://srfi.schemers.org/srfi-113/srfi-113-1.3.html}{2013/07/03}
\item
  Revised:
  \href{http://srfi.schemers.org/srfi-113/srfi-113-1.4.html}{2013/12/04}
\item
  Revised:
  \href{http://srfi.schemers.org/srfi-113/srfi-113-1.5.html}{2014/03/31}
\item
  Revised:
  \href{http://srfi.schemers.org/srfi-113/srfi-113-1.6.html}{2014/08/15}
\item
  Revised:
  \href{http://srfi.schemers.org/srfi-113/srfi-113-1.8.html}{2014/08/19}
\item
  Revised:
  \href{http://srfi.schemers.org/srfi-113/srfi-113-1.9.html}{2014/08/22}
\item
  Revised:
  \href{http://srfi.schemers.org/srfi-113/srfi-113-1.10.html}{2014/08/22}
\item
  Revised:
  \href{http://srfi.schemers.org/srfi-113/srfi-113-1.11.html}{2014/08/29}
\item
  Revised:
  \href{http://srfi.schemers.org/srfi-113/srfi-113-1.12.html}{2014/11/16}
\item
  Revised:
  \href{http://srfi.schemers.org/srfi-113/srfi-113-1.13.html}{2014/11/28}
\item
  Final:
  \href{http://srfi.schemers.org/srfi-113/srfi-113-1.14.html}{2014/11/29}
\item
  Revised to fix errata: 2016/5/14 (Add post-finalization note.)
\item
  Revised to fix errata: 2016/11/29 (Fixed calls to `set-map` and
  `bag-map`.)
\item
  Revised to fix errata: 2017/1/3 (Replaced references to ``character
  set'' with ``set or bag.'')
\end{itemize}

\textbf{Post-finalization note 1}: Because SRFI 114 has been deprecated
by SRFI 128, it is recommended that implementers make use of SRFI 128
rather than SRFI 114 comparators where comparators are specified in this
SRFI. Specifically, the procedures \texttt{set}, \texttt{bag},
\texttt{set-unfold}, \texttt{bag-unfold}, \texttt{set-map},
\texttt{list-\textgreater{}set}, \texttt{list-\textgreater{}bag}, and
\texttt{alist-\textgreater{}bag}, should accept SRFI 128 rather than
SRFI 114 comparators as arguments. By the same token, the results of
\texttt{set-element-comparator} and \texttt{bag-element-comparator}, as
well as the values of \texttt{set-comparator} and
\texttt{bag-comparator}, should be SRFI 128 comparators. The sample
implementation has been updated to depend on SRFI 128 rather than SRFI
114.

\textbf{Post-finalization note 2}: The order of arguments of
\texttt{set-map} and \texttt{set-unfold} are not consistent with those
of \texttt{hash-table-map} and \texttt{hash-table-unfold} in
\href{https://srfi.schemers.org/srfi-125/}{SRFI 125}. Both SRFI 113 and
SRFI 125 are going to be part of R7RS large (also known as the
\href{http://trac.sacrideo.us/wg/wiki/RedEdition}{Red Edition}), and the
plan is to make them consistent there, following the order in SRFI 125.
Furthermore, \href{https://srfi.schemers.org/srfi-146/}{SRFI 146} (in
draft status at the time of this writing) uses the order of SRFI 125.
Since this problem was discovered after SRFI 113 was finalized, and it
wasn't an error in the SRFI, it's too late to fix it here. However, John
Cowan, the author, encourages implementers to consider adopting the
order of SRFI 125. Thanks to Marc Nieper-Wißkirchen for reporting this
mismatch, and for providing a version of SRFI 113 with the recommended
argument order in the implementation of SRFI 146.

\section{Abstract}\label{abstract}

\emph{Sets} and \emph{bags} (also known as multisets) are unordered
collections that can contain any Scheme object. Sets enforce the
constraint that no two elements can be the same in the sense of the
set's associated \emph{equality predicate}; bags do not.

\section{Rationale}\label{rationale}

Sets are a standard part of the libraries of many high-level programming
languages, including Smalltalk,
\href{http://docs.oracle.com/javase/6/docs/api/java/util/Set.html}{Java},
and \href{http://www.cplusplus.com/reference/set/set/}{C++}.
\href{http://docs.racket-lang.org/reference/sets.html}{Racket} provides
general sets similar to those of this proposal, though with fewer
procedures, and there is a Chicken egg called \texttt{sets}
(unfortunately undocumented), which provides a minimal set of
procedures. \href{http://srfi.schemers.org/srfi-1/srfi-1.html}{SRFI 1}
also provides a list-based implementation of sets.

Bags are useful for counting anything from a fixed set of possibilities,
e.g. the number of each type of error in a log file or the number of
uses of each word in a lexicon drawn from a body of documents. Although
other data structures can serve the same purpose, using bags clearly
expresses the programmer's intent and allows for optimization.

Insofar as possible, the names in this SRFI are harmonized with the
names used for ordered collections (lists, strings, vectors, and
bytevectors) in Scheme. However, \texttt{size} is used instead of
\texttt{length} to express the number of elements in a collection,
because \texttt{length} implies order.

It's possible to use the general sets of this SRFI to contain
characters, but the use of
\href{http://srfi.schemers.org/srfi-14/srfi-14.html}{SRFI 14} is
recommended instead. The names and facilities in this SRFI are
harmonized with SRFI 14, except that SRFI 14 does not contain analogues
of the \texttt{set-search!}, \texttt{set\textgreater{}?},
\texttt{set\textless{}=?}, \texttt{set\textgreater{}=?},
\texttt{set-remove}, or \texttt{set-partition} procedures.

Sets and bags do not have a lexical syntax representation. It's possible
to use \href{http://srfi.schemers.org/srfi-108/srfi-108.html}{SRFI 108}
quasi-literal constructors to create them in code, but this SRFI does
not standardize how that is done.

The interface to general sets and bags depends on
\href{http://srfi.schemers.org/srfi-114/srfi-114.html}{SRFI 114}
comparators, despite that SRFI having a higher number than this one for
\href{http://www.catb.org/jargon/html/H/hysterical-reasons.html}{hysterical
raisins}. Comparators conveniently package the equality predicate of the
set with the hash function or comparison procedure needed to implement
the set efficiently.

\section{Specification}\label{specification}

Sets and bags are mutually disjoint, and disjoint from other types of
Scheme objects.

It is an error for any procedure defined in this SRFI to be invoked on
sets or bags with distinct comparators (in the sense of \texttt{eq?}).

It is an error to mutate any object while it is contained in a set or
bag.

It is an error to add an object to a set or bag which does not satisfy
the type test predicate of the comparator.

It is an error to add or remove an object for a set or a bag while
iterating over it.

\subsection{Linear update}\label{linear-update}

The procedures of this SRFI, by default, are ``pure functional'' ---
they do not alter their parameters. However, this SRFI also defines
``linear-update'' procedures, all of whose names end in \texttt{!}. They
have hybrid pure-functional/side-effecting semantics: they are allowed,
but not required, to side-effect one of their parameters in order to
construct their result. An implementation may legally implement these
procedures as pure, side-effect-free functions, or it may implement them
using side effects, depending upon the details of what is the most
efficient or simple to implement in terms of the underlying
representation.

It is an error to rely upon these procedures working by side effect. For
example, this is not guaranteed to work:

\begin{verbatim}
        (let* ((set1 (set 'a 'b 'c))      ; set1 = {a,b,c}.
               (set2 (set-adjoin! set1 'd)))   ; Add d to {a,b,c}.
          set1) ; Could be either {a,b,c} or {a,b,c,d}.
\end{verbatim}

However, this is well-defined:

\begin{verbatim}
        (let ((set1 (set 'a 'b 'c)))
          (set-adjoin! set1 'd)) ; Add d to {a,b,c}.
\end{verbatim}

So clients of these procedures write in a functional style, but must
additionally be sure that, when the procedure is called, there are no
other live pointers to the potentially-modified set or bag (hence the
term ``linear update'').

There are two benefits to this convention:

\begin{itemize}
\item
  Implementations are free to provide the most efficient possible
  implementation, either functional or side-effecting.
\item
  Programmers may nonetheless continue to assume that sets are purely
  functional data structures: they may be reliably shared without
  needing to be copied, uniquified, and so forth.
\end{itemize}

In practice, these procedures are most useful for efficiently
constructing sets and bags in a side-effecting manner, in some limited
local context, before passing the set or bag outside the local
construction scope to be used in a functional manner.

Scheme provides no assistance in checking the linearity of the
potentially side-effected parameters passed to these functions ---
there's no linear type checker or run-time mechanism for detecting
violations.

Note that if an implementation uses no side effects at all, it is
allowed to return existing sets and bags rather than newly allocated
ones, even where this SRFI explicitly says otherwise.

\subsection{Comparator restrictions}\label{comparator-restrictions}

Implementations of this SRFI are allowed to place restrictions on the
comparators that the procedures accept. In particular, an implementation
may require comparators to provide a comparison procedure.
Alternatively, an implementation may require comparators to provide a
hash function, unless the equality predicate of the comparator is
\texttt{eq?}, \texttt{eqv?}, \texttt{equal?}, \texttt{string=?}, or
\texttt{string-ci=?}. Implementations must not require the provision of
both a comparison procedure and a hash function.

\subsection{Index}\label{Index}

\begin{itemize}
\item
  \protect\hyperlink{Constructors}{Constructors}: \texttt{set},
  \texttt{set-contains?}, \texttt{set-unfold}
\item
  \protect\hyperlink{Predicates}{Predicates}: \texttt{set?},
  \texttt{set-empty?}, \texttt{set-disjoint?}
\item
  \protect\hyperlink{Accessors}{Accessors}: \texttt{set-member},
  \texttt{set-element-comparator}
\item
  \protect\hyperlink{Updaters}{Updaters}: \texttt{set-adjoin},
  \texttt{set-adjoin!}, \texttt{set-replace}, \texttt{set-replace!},
  \texttt{set-delete}, \texttt{set-delete!}, \texttt{set-delete-all},
  \texttt{set-delete-all!}, \texttt{set-search!}
\item
  \protect\hyperlink{Thewholeset}{The whole set}: \texttt{set-size},
  \texttt{set-find}, \texttt{set-count}, \texttt{set-any?},
  \texttt{set-every?}
\item
  \protect\hyperlink{Mappingandfolding}{Mapping and folding}:
  \texttt{set-map}, \texttt{set-for-each}, \texttt{set-fold},
  \texttt{set-filter}, \texttt{set-filter!}, \texttt{set-remove},
  \texttt{set-remove!}, \texttt{set-partition}, \texttt{set-partition!}
\item
  \protect\hyperlink{Copyingandconversion}{Copying and conversion}:
  \texttt{set-copy}, \texttt{set-\textgreater{}list},
  \texttt{list-\textgreater{}set}, \texttt{list-\textgreater{}set!}
\item
  \protect\hyperlink{Subsets}{Subsets}: \texttt{set=?},
  \texttt{set\textless{}?}, \texttt{set\textgreater{}?},
  \texttt{set\textless{}=?}, \texttt{set\textgreater{}=?}
\item
  \protect\hyperlink{Settheoryoperations}{Set theory operations}:
  \texttt{set-union}, \texttt{set-intersection},
  \texttt{set-difference}, \texttt{set-xor}, \texttt{set-union!},
  \texttt{set-intersection!}, \texttt{set-difference!},
  \texttt{set-xor!}
\item
  \protect\hyperlink{Additionalbagprocedures}{Additional bag
  procedures}: \texttt{bag-sum}, \texttt{bag-sum!},
  \texttt{bag-product}, \texttt{bag-product!},
  \texttt{bag-element-count}, \texttt{bag-for-each-unique},
  \texttt{bag-fold-unique}, \texttt{bag-increment!},
  \texttt{bag-decrement!}, \texttt{bag-\textgreater{}set},
  \texttt{set-\textgreater{}bag}, \texttt{set-\textgreater{}bag!}
\item
  \protect\hyperlink{Comparators}{Comparators}: \texttt{set-comparator},
  \texttt{bag-comparator}
\end{itemize}

\subsection{Set procedures}\label{Setprocedures}

\hypertarget{Constructors}{\subsubsection{Constructors}\label{Constructors}}

\texttt{(set\ }\emph{comparator} \emph{element} \ldots{} \texttt{)}

Returns a newly allocated empty set. The \emph{comparator} argument is a
\href{http://srfi.schemers.org/srfi-114/srfi-114.html}{SRFI 114}
comparator, which is used to control and distinguish the elements of the
set. The \emph{elements} are used to initialize the set.

\texttt{(set-unfold\ }\emph{comparator stop? mapper successor
seed}\texttt{)}

Create a newly allocated set as if by \texttt{set} using
\emph{comparator}. If the result of applying the predicate \emph{stop?}
to \emph{seed} is true, return the set. Otherwise, apply the procedure
\emph{mapper} to \emph{seed}. The value that \emph{mapper} returns is
added to the set. Then get a new seed by applying the procedure
\emph{successor} to \emph{seed}, and repeat this algorithm.

\hypertarget{Predicates}{\subsubsection{Predicates}\label{Predicates}}

\texttt{(set?\ }\emph{obj}\texttt{)}

Returns \texttt{\#t} if \emph{obj} is a set, and \texttt{\#f} otherwise.

\texttt{(set-contains?\ }\emph{set element}\texttt{)}

Returns \texttt{\#t} if \emph{element} is a member of \emph{set} and
\texttt{\#f} otherwise.

\texttt{(set-empty?\ }\emph{set}\texttt{)}

Returns \texttt{\#t} if \emph{set} has no elements and \texttt{\#f}
otherwise.

\texttt{(set-disjoint?\ }\emph{set\textsubscript{1}
set\textsubscript{2}}\texttt{)}

Returns \texttt{\#t} if \emph{set\textsubscript{1}} and
\emph{set\textsubscript{2}} have no elements in common and \texttt{\#f}
otherwise.

\hypertarget{Accessors}{\subsubsection{Accessors}\label{Accessors}}

\texttt{(set-member\ }\emph{set element default}\texttt{)}

Returns the element of \emph{set} that is equal, in the sense of
\emph{set's} equality predicate, to \emph{element}. If \emph{element} is
not a member of \emph{set}, \emph{default} is returned.

\texttt{(set-element-comparator\ }\emph{set}\texttt{)}

Returns the comparator used to compare the elements of \emph{set}.

\hypertarget{Updaters}{\subsubsection{Updaters}\label{Updaters}}

\texttt{(set-adjoin\ }\emph{set element} \ldots{}\texttt{)}

The \texttt{set-adjoin} procedure returns a newly allocated set that
uses the same comparator as \emph{set} and contains all the values of
\emph{set}, and in addition each \emph{element} unless it is already
equal (in the sense of the comparator) to one of the existing or newly
added members. It is an error to add an element to \emph{set} that does
not return \texttt{\#t} when passed to the type test procedure of the
comparator.

\texttt{(set-adjoin!\ }\emph{set element} \ldots{}\texttt{)}

The \texttt{set-adjoin!} procedure is the same as \texttt{set-adjoin},
except that it is permitted to mutate and return the \emph{set} argument
rather than allocating a new set.

\texttt{(set-replace\ }\emph{set element}\texttt{)}

The \texttt{set-replace} procedure returns a newly allocated set that
uses the same comparator as \emph{set} and contains all the values of
\emph{set} except as follows: If \emph{element} is equal (in the sense
of \emph{set}'s comparator) to an existing member of \emph{set}, then
that member is omitted and replaced by \emph{element}. If there is no
such element in \emph{set}, then \emph{set} is returned unchanged.

\texttt{(set-replace!\ }\emph{set element}\texttt{)}

The \texttt{set-replace!} procedure is the same as\texttt{set-replace},
except that it is permitted to mutate and return the \emph{set} argument
rather than allocating a new set.

\texttt{(set-delete\ }\emph{set element} \ldots{}\texttt{)}

\texttt{(set-delete!\ }\emph{set element} \ldots{}\texttt{)}

\texttt{(set-delete-all\ }\emph{set element-list}\texttt{)}

\texttt{(set-delete-all!\ }\emph{set element-list}\texttt{)}

The \texttt{set-delete} procedure returns a newly allocated set
containing all the values of \emph{set} except for any that are equal
(in the sense of \emph{set}'s comparator) to one or more of the
\emph{elements}. Any \emph{element} that is not equal to some member of
the set is ignored.

The \texttt{set-delete!} procedure is the same as \texttt{set-delete},
except that it is permitted to mutate and return the \emph{set} argument
rather than allocating a new set.

The \texttt{set-delete-all} and \texttt{set-delete-all!} procedures are
the same as \texttt{set-delete} and \texttt{set-delete!}, except that
they accept a single argument which is a list of elements to be deleted.

\texttt{(set-search!\ }\emph{set element failure success}\texttt{)}

The \emph{set} is searched for \emph{element}. If it is not found, then
the \emph{failure} procedure is tail-called with two continuation
arguments, \emph{insert} and \emph{ignore}, and is expected to tail-call
one of them. If \emph{element} is found, then the \emph{success}
procedure is tail-called with the matching element of \emph{set} and two
continuations, \emph{update} and \emph{remove}, and is expected to
tail-call one of them.

The effects of the continuations are as follows (where \emph{obj} is any
Scheme object):

\begin{itemize}
\item
  Invoking \texttt{(}\emph{insert obj}\texttt{)} causes \emph{element}
  to be inserted into \emph{set}.
\item
  Invoking \texttt{(}\emph{ignore obj}\texttt{)} causes \emph{set} to
  remain unchanged.
\item
  Invoking \texttt{(}\emph{update new-element obj}\texttt{)} causes
  \emph{new-element} to be inserted into \emph{set} in place of
  \emph{element}.
\item
  Invoking \texttt{(}\emph{remove obj}\texttt{)} causes the matching
  element of \emph{set} to be removed from it.
\end{itemize}

In all cases, two values are returned: the possibly updated \emph{set}
and \emph{obj}.

\hypertarget{Thewholeset}{\subsubsection{The whole
set}\label{Thewholeset}}

\texttt{(set-size\ }\emph{set}\texttt{)}

Returns the number of elements in \emph{set} as an exact integer.

\texttt{(set-find\ }\emph{predicate set failure}\texttt{)}

Returns an arbitrarily chosen element of \emph{set} that satisfies
\emph{predicate}, or the result of invoking \emph{failure} with no
arguments if there is none.

\texttt{(set-count\ }\emph{predicate set}\texttt{)}

Returns the number of elements of \emph{set} that satisfy
\emph{predicate} as an exact integer.

\texttt{(set-any?\ }\emph{predicate set}\texttt{)}

Returns \texttt{\#t} if any element of \emph{set} satisfies
\emph{predicate}, or \texttt{\#f} otherwise. Note that this differs from
the SRFI 1 analogue because it does not return an element of the set.

\texttt{(set-every?\ }\emph{predicate set}\texttt{)}

Returns \texttt{\#t} if every element of \emph{set} satisfies
\emph{predicate}, or \texttt{\#f} otherwise. Note that this differs from
the SRFI 1 analogue because it does not return an element of the set.

\hypertarget{Mappingandfolding}{\subsubsection{Mapping and
folding}\label{Mappingandfolding}}

\texttt{(set-map\ }\emph{comparator proc set}\texttt{)}

Applies \emph{proc} to each element of \emph{set} in arbitrary order and
returns a newly allocated set, created as if by
\texttt{(set\ }\emph{comparator}\texttt{)}, which contains the results
of the applications. For example:

\begin{verbatim}
        (set-map string-ci-comparator symbol->string (set eq? 'foo 'bar 'baz))
             => (set string-ci-comparator "foo" "bar" "baz")
\end{verbatim}

Note that, when \emph{proc} defines a mapping that is not 1:1, some of
the mapped objects may be equivalent in the sense of \emph{comparator's}
equality predicate, and in this case duplicate elements are omitted as
in the set constructor. For example:

\begin{verbatim}
(set-map (lambda (x) (quotient x 2))
         integer-comparator
         (set integer-comparator 1 2 3 4 5))
 => (set integer-comparator 0 1 2)
\end{verbatim}

If the elements are the same in the sense of \texttt{eqv?}, it is
unpredictable which one will be preserved in the result.

\texttt{(set-for-each\ }\emph{proc set}\texttt{)}

Applies \emph{proc} to \emph{set} in arbitrary order, discarding the
returned values. Returns an unspecified result.

\texttt{(set-fold\ }\emph{proc nil set}\texttt{)}

Invokes \emph{proc} on each member of \emph{set} in arbitrary order,
passing the result of the previous invocation as a second argument. For
the first invocation, \emph{nil} is used as the second argument. Returns
the result of the last invocation, or \emph{nil} if there was no
invocation.

\texttt{(set-filter\ }\emph{predicate set}\texttt{)}

Returns a newly allocated set with the same comparator as \emph{set},
containing just the elements of \emph{set} that satisfy
\emph{predicate}.

\texttt{(set-filter!\ }\emph{predicate set}\texttt{)}

A linear update procedure that returns a set containing just the
elements of \emph{set} that satisfy \emph{predicate}.

\texttt{(set-remove\ }\emph{predicate set}\texttt{)}

Returns a newly allocated set with the same comparator as \emph{set},
containing just the elements of \emph{set} that do not satisfy
\emph{predicate}.

\texttt{(set-remove!\ }\emph{predicate set}\texttt{)}

A linear update procedure that returns a set containing just the
elements of \emph{set} that do not satisfy \emph{predicate}.

\texttt{(set-partition\ }\emph{predicate set}\texttt{)}

Returns two values: a newly allocated set with the same comparator as
\emph{set} that contains just the elements of \emph{set} that satisfy
\emph{predicate}, and another newly allocated set, also with the same
comparator, that contains just the elements of \emph{set} that do not
satisfy \emph{predicate}.

\texttt{(set-partition!\ }\emph{predicate set}\texttt{)}

A linear update procedure that returns two sets containing the elements
of \emph{set} that do and do not, respectively, not satisfy
\emph{predicate}.

\hypertarget{Copyingandconversion}{\subsubsection{Copying and
conversion}\label{Copyingandconversion}}

\texttt{(set-copy\ }\emph{set}\texttt{)}

Returns a newly allocated set containing the elements of \emph{set}, and
using the same comparator.

\texttt{(set-\textgreater{}list\ }\emph{set}\texttt{)}

Returns a newly allocated list containing the members of \emph{set} in
unspecified order.

\texttt{(list-\textgreater{}set\ }\emph{comparator list}\texttt{)}

Returns a newly allocated set, created as if by \texttt{set} using
\emph{comparator}, that contains the elements of \emph{list}. Duplicate
elements (in the sense of the equality predicate) are omitted.

\texttt{(list-\textgreater{}set!\ }\emph{set list}\texttt{)}

Returns a set that contains the elements of both \emph{set} and
\emph{list}. Duplicate elements (in the sense of the equality predicate)
are omitted.

\hypertarget{Subsets}{\subsubsection{Subsets}\label{Subsets}}

Note: The following three predicates do not obey the trichotomy law and
therefore do not constitute a total order on sets.

\texttt{(set=?\ }\emph{set\textsubscript{1} set\textsubscript{2}}
\ldots{}\texttt{)}

Returns \texttt{\#t} if each \emph{set} contains the same elements.

\texttt{(set\textless{}?\ }\emph{set\textsubscript{1}
set\textsubscript{2}} \ldots{}\texttt{)}

Returns \texttt{\#t} if each \emph{set} other than the last is a proper
subset of the following \emph{set}, and \texttt{\#f} otherwise.

\texttt{(set\textgreater{}?\ }\emph{set\textsubscript{1}
set\textsubscript{2}} \ldots{}\texttt{)}

Returns \texttt{\#t} if each \emph{set} other than the last is a proper
superset of the following \emph{set}, and \texttt{\#f} otherwise.

\texttt{(set\textless{}=?\ }\emph{set\textsubscript{1}
set\textsubscript{2}} \ldots{}\texttt{)}

Returns \texttt{\#t} if each \emph{set} other than the last is a subset
of the following \emph{set}, and \texttt{\#f} otherwise.

\texttt{(set\textgreater{}=?\ }\emph{set\textsubscript{1}
set\textsubscript{2}} \ldots{}\texttt{)}

Returns \texttt{\#t} if each \emph{set} other than the last is a
superset of the following \emph{set}, and \texttt{\#f} otherwise.

\hypertarget{Settheoryoperations}{\subsubsection{Set theory
operations}\label{Settheoryoperations}}

\texttt{(set-union\ }\emph{set\textsubscript{1} set\textsubscript{2}}
\ldots{}\texttt{)}

\texttt{(set-intersection\ }\emph{set\textsubscript{1}
set\textsubscript{2}} \ldots{}\texttt{)}

\texttt{(set-difference\ }\emph{set\textsubscript{1}
set\textsubscript{2}} \ldots{}\texttt{)}

\texttt{(set-xor\ }\emph{set\textsubscript{1}
set\textsubscript{2}}\texttt{)}

Return a newly allocated set that is the union, intersection, asymmetric
difference, or symmetric difference of the \emph{sets}. Asymmetric
difference is extended to more than two sets by taking the difference
between the first set and the union of the others. Symmetric difference
is not extended beyond two sets. Elements in the result set are drawn
from the first set in which they appear.

\texttt{(set-union!\ }\emph{set\textsubscript{1} set\textsubscript{2}}
\ldots{}\texttt{)}

\texttt{(set-intersection!\ }\emph{set\textsubscript{1}
set\textsubscript{2}} \ldots{}\texttt{)}

\texttt{(set-difference!\ }\emph{set\textsubscript{1}
set\textsubscript{2}} \ldots{}\texttt{)}

\texttt{(set-xor!\ }\emph{set\textsubscript{1}
set\textsubscript{2}}\texttt{)}

Linear update procedures returning a set that is the union,
intersection, asymmetric difference, or symmetric difference of the
\emph{sets}. Asymmetric difference is extended to more than two sets by
taking the difference between the first set and the union of the others.
Symmetric difference is not extended beyond two sets. Elements in the
result set are drawn from the first set in which they appear.

\subsection{Bag procedures}\label{Bagprocedures}

Bags are like sets, but can contain the same object more than once.
However, if two elements that are the same in the sense of the equality
predicate, but not in the sense of \texttt{eqv?}, are both included, it
is not guaranteed that they will remain distinct when retrieved from the
bag. It is an error for a single procedure to be invoked on bags with
different comparators.

The procedures for creating and manipulating bags are the same as those
for sets, except that \texttt{set} is replaced by \texttt{bag} in their
names, and that adjoining an element to a bag is effective even if the
bag already contains the element. If two elements in a bag are the same
in the sense of the bag's comparator, the implementation may in fact
store just one of them.

The \texttt{bag-union}, \texttt{bag-intersection},
\texttt{bag-difference}, and \texttt{bag-xor} procedures (and their
linear update analogues) behave as follows when both bags contain
elements that are equal in the sense of the bags' comparator:

\begin{itemize}
\item
  For \texttt{bag-union}, the number of equal elements in the result is
  the largest number of equal elements in any of the original bags.
\item
  For \texttt{bag-intersection}, the number of equal elements in the
  result is the smallest number of equal elements in any of the original
  bags.
\item
  For \texttt{bag-difference}, the number of equal elements in the
  result is the number of equal elements in the first bag, minus the
  number of elements in the other bags (but not less than zero).
\item
  For \texttt{bag-xor}, the number of equal elements in the result is
  the absolute value of the difference between the number of equal
  elements in the first and second bags.
\end{itemize}

\hypertarget{Additionalbagprocedures}{\subsubsection{Additional bag
procedures}\label{Additionalbagprocedures}}

\texttt{(bag-sum\ }\emph{set\textsubscript{1} set\textsubscript{2}}
\ldots{} \texttt{)}

\texttt{(bag-sum!\ }\emph{bag\textsubscript{1} bag\textsubscript{2}}
\ldots{} \texttt{)}

The \texttt{bag-sum} procedure returns a newly allocated bag containing
all the unique elements in all the \emph{bags}, such that the count of
each unique element in the result is equal to the sum of the counts of
that element in the arguments. It differs from \texttt{bag-union} by
treating identical elements as potentially distinct rather than
attempting to match them up.

The \texttt{bag-sum!} procedure is equivalent except that it is
linear-update.

\texttt{(bag-product\ }\emph{n bag}\texttt{)}

\texttt{(bag-product!\ }\emph{n bag}\texttt{)}

The \texttt{bag-product} procedure returns a newly allocated bag
containing all the unique elements in \emph{bag}, where the count of
each unique element in the bag is equal to the count of that element in
\texttt{bag} multiplied by \emph{n}.

The \texttt{bag-product!} procedure is equivalent except that it is
linear-update.

\texttt{(bag-unique-size\ }\emph{bag}\texttt{)}

Returns the number of unique elements of \emph{bag}.

\texttt{(bag-element-count\ }\emph{bag element}\texttt{)}

Returns an exact integer representing the number of times that
\emph{element} appears in \emph{bag}.

\texttt{(bag-for-each-unique\ }\emph{proc bag}\texttt{)}

Applies \emph{proc} to each unique element of \emph{bag} in arbitrary
order, passing the element and the number of times it occurs in
\emph{bag}, and discarding the returned values. Returns an unspecified
result.

\texttt{(bag-fold-unique\ }\emph{proc nil bag}\texttt{)}

Invokes \emph{proc} on each unique element of \emph{bag} in arbitrary
order, passing the number of occurrences as a second argument and the
result of the previous invocation as a third argument. For the first
invocation, \emph{nil} is used as the third argument. Returns the result
of the last invocation.

\texttt{(bag-increment!\ }\emph{bag\texttt{\ }element\texttt{\ }count}\texttt{)}

\texttt{(bag-decrement!\ }\emph{bag\texttt{\ }element\texttt{\ }count}\texttt{)}

Linear update procedures that return a bag with the same elements as
\emph{bag}, but with the element count of \emph{element} in \emph{bag}
increased or decreased by the exact integer \emph{count} (but not less
than zero).

\texttt{(bag-\textgreater{}set\ }\emph{bag}\texttt{)}

\texttt{(set-\textgreater{}bag\ }\emph{set}\texttt{)}

\texttt{(set-\textgreater{}bag!\ }\emph{bag set}\texttt{)}

The \texttt{bag-\textgreater{}set} procedure returns a newly allocated
set containing the unique elements (in the sense of the equality
predicate) of \emph{bag}. The \texttt{set-\textgreater{}bag} procedure
returns a newly allocated bag containing the elements of \emph{set}. The
\texttt{set-\textgreater{}bag!} procedure returns a bag containing the
elements of both \emph{bag} and \emph{set}. In all cases, the comparator
of the result is the same as the comparator of the argument or
arguments.

\texttt{(bag-\textgreater{}alist\ }\emph{bag}\texttt{)}

\texttt{(alist-\textgreater{}bag\ }\emph{comparator alist}\texttt{)}

The \texttt{bag-\textgreater{}alist} procedure returns a newly allocated
alist whose keys are the unique elements of \emph{bag} and whose values
are the number of occurrences of each element. The
\texttt{alist-\textgreater{}bag} returning a newly allocated bag based
on \emph{comparator}, where the keys of \emph{alist} specify the
elements and the corresponding values of \emph{alist} specify how many
times they occur.

\hypertarget{Comparators}{\subsection{Comparators}\label{Comparators}}

The following comparators are used to compare sets or bags, and allow
sets of sets, bags of sets, etc.

\texttt{set-comparator}

\texttt{bag-comparator}

Note that these comparators do not provide comparison procedures, as
there is no ordering between sets or bags. It is an error to compare
sets or bags with different element comparators.

\section{Implementation}\label{implementation}

The implementation places the identifiers defined above into the
\texttt{sets} library.

Sets and bags are implemented as a thin veneer over hashtables.

The implementation registers \texttt{set-comparator} and
\texttt{bag-comparator} with SRFI 114's default comparator, assuming the
sample implementation of SRFI 114 is being used. Scheme implementers who
provide their own implementations of SRFI 114 must change this part of
the code.

The \href{http://srfi.schemers.org/srfi-113/sets.tar.gz}{sample
implementation} contains the following files:

\begin{itemize}
\tightlist
\item
  \texttt{sets-impl.scm} -- implementation of general sets and bags
\item
  \texttt{sets.sld} -- an R7RS library named \texttt{(sets)}
\item
  \texttt{sets.scm} -- a Chicken library
\item
  \texttt{sets-test.scm} -- test suite
\end{itemize}

The test suite will work with the Chicken \texttt{test} egg, which is
provided on Chibi as the \texttt{(chibi\ test)} library.

\section{Copyright}\label{copyright}

Copyright (C) John Cowan 2013. All Rights Reserved.

Permission is hereby granted, free of charge, to any person obtaining a
copy of this software and associated documentation files (the
``Software''), to deal in the Software without restriction, including
without limitation the rights to use, copy, modify, merge, publish,
distribute, sublicense, and/or sell copies of the Software, and to
permit persons to whom the Software is furnished to do so, subject to
the following conditions:

The above copyright notice and this permission notice shall be included
in all copies or substantial portions of the Software.

THE SOFTWARE IS PROVIDED ``AS IS'', WITHOUT WARRANTY OF ANY KIND,
EXPRESS OR IMPLIED, INCLUDING BUT NOT LIMITED TO THE WARRANTIES OF
MERCHANTABILITY, FITNESS FOR A PARTICULAR PURPOSE AND NONINFRINGEMENT.
IN NO EVENT SHALL THE AUTHORS OR COPYRIGHT HOLDERS BE LIABLE FOR ANY
CLAIM, DAMAGES OR OTHER LIABILITY, WHETHER IN AN ACTION OF CONTRACT,
TORT OR OTHERWISE, ARISING FROM, OUT OF OR IN CONNECTION WITH THE
SOFTWARE OR THE USE OR OTHER DEALINGS IN THE SOFTWARE.

\begin{center}\rule{0.5\linewidth}{\linethickness}\end{center}

Editor:
\href{mailto:srfi-editors\%20at\%20srfi\%20dot\%20schemers\%20dot\%20org}{Mike
Sperber}
