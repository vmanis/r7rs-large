\section{Title}\label{title}

Queues based on lists

\section{Author}\label{author}

John Cowan \textless{}cowan@ccil.org\textgreater{}

\section{Status}\label{status}

This SRFI is currently in ``final'' status. To see an explanation of
each status that a SRFI can hold, see
\href{http://srfi.schemers.org/srfi-process.html}{here}. To provide
input on this SRFI, please
\href{mailto:srfi\%20minus\%20117\%20at\%20srfi\%20dot\%20schemers\%20dot\%20org}{mail
to
\texttt{\textless{}srfi\ minus\ 117\ at\ srfi\ dot\ schemers\ dot\ org\textgreater{}}}.
See \href{../srfi-list-subscribe.html}{instructions here} to subscribe
to the list. You can access previous messages via
\href{mail-archive/maillist.html}{the archive of the mailing list}.

\begin{itemize}
\tightlist
\item
  Received:
  \href{http://srfi.schemers.org/srfi-117/srfi-117-1.1.html}{2014/12/02}
\item
  Draft: 2014/12/02-2015/02/02
\item
  Revised:
  \href{http://srfi.schemers.org/srfi-117/srfi-117-1.2.html}{2014/12/04}
\item
  Revised:
  \href{http://srfi.schemers.org/srfi-117/srfi-117-1.3.html}{2014/12/06}
\item
  Revised:
  \href{http://srfi.schemers.org/srfi-117/srfi-117-1.4.html}{2015/01/11}
\item
  Revised:
  \href{http://srfi.schemers.org/srfi-117/srfi-117-1.5.html}{2015/06/08}
\item
  Finalized: 2015/08/25
\item
  Revised to fix errata: 2015/09/06, 2015/09/10, 2016/7/26
\end{itemize}

\section{Abstract}\label{abstract}

List queues are mutable ordered collections that can contain any Scheme
object. Each list queue is based on an ordinary Scheme list containing
the elements of the list queue by maintaining pointers to the first and
last pairs of the list. It's cheap to add or remove elements from the
front of the list or to add elements to the back, but not to remove
elements from the back. List queues are disjoint from other types of
Scheme objects.

\section{Issues}\label{issues}

No issues at present.

\section{Rationale}\label{rationale}

The list queues described in this SRFI are objects of a disjoint type
that provide an ordered sequence of arbitrary Scheme objects. Like
lists, they provide sequential access to their elements; unlike lists,
they normally should not share storage with other list queues, unless
special precautions have been taken.

Technically, a queue is an object that makes it efficient to remove
elements from the front and to add elements to the back. List queues
also make it efficient to add elements to the front, but removing
elements from the back is inefficient. Therefore, because these objects
do not provide the normal behavioral guarantees of
\href{http://en.wikipedia.org/wiki/Double-ended_queue}{deques}, they are
not referred to as deques. True deques will be provided in a future
SRFI.

Historically, objects of this form were called ``tconc structures''
(where ``tconc'' is short for ``tail concatenate''), and Lispers have
tended to roll their own using pairs. This version uses a record to hold
the first-pair and last-pair pointers rather than a pair, but uses pairs
and the empty list internally.

Because the representation of list queues as lists is exposed by the
implementation, it's not necessary to provide a comprehensive API for
list queues. Instead,
\href{http://srfi.schemers.org/srfi-1/srfi-1.html}{SRFI 1} and other
list APIs can serve the same purpose, using the conversion procedures to
convert between list queues and lists fairly cheaply. Consequently, the
API provided here over and above the bare necessities of queueing and
dequeueing elements is roughly analogous to the R7RS-small API for
lists. It also subsumes the
\href{http://wiki.call-cc.org/man/4/Unit\%20data-structures\#queues}{Chicken}
and
\href{http://people.csail.mit.edu/jaffer/slib/Queues.html\#Queues}{​SLIB}
APIs.

\section{Specification}\label{specification}

\subsection{Constructors}\label{Constructors}

\texttt{(make-list-queue\ }list {[} last {]}\texttt{)}

Returns a newly allocated list queue containing the elements of list in
order. The result shares storage with list. If the last argument is not
provided, this operation is O(n) where n is the length of list.

However, if last is provided, \texttt{make-list-queue} returns a newly
allocated list queue containing the elements of the list whose first
pair is first and whose last pair is last. It is an error if the pairs
do not belong to the same list. Alternatively, both first and last can
be the empty list. In either case, the operation is O(1).

Note: To apply a non-destructive list procedure to a list queue and
return a new list queue, use
\texttt{(make-list-queue\ (}proc\texttt{\ (list-queue-list\ }list-queue\texttt{)))}.

\texttt{(list-queue\ }element \ldots{}\texttt{)}

Returns a newly allocated list queue containing the elements. This
operation is O(n) where n is the number of elements.

\texttt{(list-queue-copy\ }list-queue\texttt{)}

Returns a newly allocated list queue containing the elements of
list-queue. This operation is O(n) where n is the length of list-queue

\texttt{(list-queue-unfold\ }stop? mapper successor seed {[} queue
{]}\texttt{)}

Performs the following algorithm:

If the result of applying the predicate stop? to seed is true, return
queue. Otherwise, apply the procedure mapper to seed, returning a value
which is added to the front of queue. Then get a new seed by applying
the procedure successor to seed, and repeat this algorithm.

If queue is omitted, a newly allocated list queue is used.

\texttt{(list-queue-unfold-right\ }stop? mapper successor seed {[} queue
{]}\texttt{)}

Performs the following algorithm:

If the result of applying the predicate stop? to seed is true, return
the list queue. Otherwise, apply the procedure mapper to seed, returning
a value which is added to the back of the list queue. Then get a new
seed by applying the procedure successor to seed, and repeat this
algorithm.

If queue is omitted, a newly allocated list queue is used.

\subsection{Predicates}\label{Predicates}

\texttt{(list-queue?\ }obj\texttt{)}

Returns \texttt{\#t} if obj is a list queue, and \texttt{\#f} otherwise.
This operation is O(1).

\texttt{(list-queue-empty?\ }list-queue\texttt{)}

Returns \texttt{\#t} if list-queue has no elements, and \texttt{\#f}
otherwise. This operation is O(1).

\subsection{Accessors}\label{Accessors}

\texttt{(list-queue-front\ }list-queue\texttt{)}

Returns the first element of list-queue. If the list queue is empty, it
is an error. This operation is O(1).

\texttt{(list-queue-back\ }list-queue\texttt{)}

Returns the last element of list-queue. If the list queue is empty, it
is an error. This operation is O(1).

\texttt{(list-queue-list\ }list-queue\texttt{)}

Returns the list that contains the members of list-queue in order. The
result shares storage with list-queue. This operation is O(1).

\texttt{(list-queue-first-last\ }list-queue\texttt{)}

Returns two values, the first and last pairs of the list that contains
the members of list-queue in order. If list-queue is empty, returns two
empty lists. The results share storage with list-queue. This operation
is O(1).

\subsection{Mutators}\label{Mutators}

\texttt{(list-queue-add-front!\ }list-queue element\texttt{)}

Adds element to the beginning of list-queue. Returns an unspecified
value. This operation is O(1).

\texttt{(list-queue-add-back!\ }list-queue element\texttt{)}

Adds element to the end of list-queue. Returns an unspecified value.
This operation is O(1).

\texttt{(list-queue-remove-front!\ }list-queue\texttt{)}

Removes the first element of list-queue and returns it. If the list
queue is empty, it is an error. This operation is O(1).

\texttt{(list-queue-remove-back!\ }list-queue\texttt{)}

Removes the last element of list-queue and returns it. If the list queue
is empty, it is an error. This operation is O(n) where n is the length
of list-queue, because queues do not not have backward links.

\texttt{(list-queue-remove-all!\ }list-queue\texttt{)}

Removes all the elements of list-queue and returns them in order as a
list. This operation is O(1).

\texttt{(list-queue-set-list!\ }list-queue list {[} last {]}\texttt{)}

Replaces the list associated with list-queue with list, effectively
discarding all the elements of list-queue in favor of those in list.
Returns an unspecified value. This operation is O(n) where n is the
length of list. If last is provided, it is treated in the same way as in
\texttt{make-list-queue}, and the operation is O(1).

Note: To apply a destructive list procedure to a list queue, use
\texttt{(list-queue-set-list!\ (}proc\texttt{\ (list-queue-list\ }list-queue\texttt{)))}.

\texttt{(list-queue-append\ }list-queue \ldots{}\texttt{)}

Returns a list queue which contains all the elements in front-to-back
order from all the list-queues in front-to-back order. The result does
not share storage with any of the arguments. This operation is O(n) in
the total number of elements in all queues.

\texttt{(list-queue-append!\ }list-queue \ldots{}\texttt{)}

Returns a list queue which contains all the elements in front-to-back
order from all the list-queues in front-to-back order. It is an error to
assume anything about the contents of the list-queues after the
procedure returns. This operation is O(n) in the total number of queues,
not elements. It is not part of the R7RS-small list API, but is included
here for efficiency when pure functional append is not required.

\texttt{(list-queue-concatenate\ }list-of-list-queues\texttt{)}

Returns a list queue which contains all the elements in front-to-back
order from all the list queues which are members of list-of-list-queues
in front-to-back order. The result does not share storage with any of
the arguments. This operation is O(n) in the total number of elements in
all queues. It is not part of the R7RS-small list API, but is included
here to make appending a large number of queues possible in Schemes that
limit the number of arguments to \texttt{apply}.

\subsection{Mapping}\label{Mapping}

\texttt{(list-queue-map\ }proc list-queue\texttt{)}

Applies proc to each element of list-queue in unspecified order and
returns a newly allocated list queue containing the results. This
operation is O(n) where n is the length of list-queue.

\texttt{(list-queue-map!\ }proc list-queue\texttt{)}

Applies proc to each element of list-queue in front-to-back order and
modifies list-queue to contain the results. This operation is O(n) in
the length of list-queue. It is not part of the R7RS-small list API, but
is included here to make transformation of a list queue by mutation more
efficient.

\texttt{(list-queue-for-each\ }proc list-queue\texttt{)}

Applies proc to each element of list-queue in front-to-back order,
discarding the returned values. Returns an unspecified value. This
operation is O(n) where n is the length of list-queue.

\section{Implementation}\label{implementation}

The sample implementation places the identifiers defined above into the
\texttt{list-queue} library.

The \href{http://srfi.schemers.org/srfi-117/list-queues.tar.gz}{sample
implementation} contains the following files:

\begin{itemize}
\tightlist
\item
  \texttt{list-queues-impl.scm} -- implementation of queues
\item
  \texttt{list-queues.sld} -- an R7RS library named
  \texttt{(list-queue)}
\item
  \texttt{list-queues.scm} -- a Chicken library
\item
  \texttt{list-queues-test.scm} -- tests using the Chicken test egg
\end{itemize}

\section{Copyright}\label{copyright}

Copyright © John Cowan, 2014. All Rights Reserved.

Permission is hereby granted, free of charge, to any person obtaining a
copy of this software and associated documentation files (the
``Software''), to deal in the Software without restriction, including
without limitation the rights to use, copy, modify, merge, publish,
distribute, sublicense, and/or sell copies of the Software, and to
permit persons to whom the Software is furnished to do so, subject to
the following conditions:

The above copyright notice and this permission notice shall be included
in all copies or substantial portions of the Software.

THE SOFTWARE IS PROVIDED ``AS IS'', WITHOUT WARRANTY OF ANY KIND,
EXPRESS OR IMPLIED, INCLUDING BUT NOT LIMITED TO THE WARRANTIES OF
MERCHANTABILITY, FITNESS FOR A PARTICULAR PURPOSE AND NONINFRINGEMENT.
IN NO EVENT SHALL THE AUTHORS OR COPYRIGHT HOLDERS BE LIABLE FOR ANY
CLAIM, DAMAGES OR OTHER LIABILITY, WHETHER IN AN ACTION OF CONTRACT,
TORT OR OTHERWISE, ARISING FROM, OUT OF OR IN CONNECTION WITH THE
SOFTWARE OR THE USE OR OTHER DEALINGS IN THE SOFTWARE.

\begin{center}\rule{0.5\linewidth}{\linethickness}\end{center}

Editor:
\href{mailto:srfi-editors\%20at\%20srfi\%20dot\%20schemers\%20dot\%20org}{Michael
Sperber}

Last modified: Tue Jun 9 08:44:01 MST 2015
