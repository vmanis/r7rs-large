\section{Hash tables}
\redno{7} What hash table library should R7RS-large provide?

\begin{quote}
SRFI 69 is a basic hash table library. SRFI 125 is an upward compatible
extension of it. Likewise, R6RS provides a basic hash table library, of
which SRFI 126 is an upward compatible extension. Since the interface of
SRFI 125 depends on SRFI 128, voting for SRFI 125 will cause your vote
on comparator libraries to be automatically changed to SRFI 128.
\end{quote}

\TODO{This section specifies some procedures that are then labelled
  “deprecated”. Should they remain?}

\TODO{This section makes references to R6RS and Common Lisp. Should
  those references remain?}

This SRFI defines an interface to hash tables, which are widely
recognized as a fundamental data structure for a wide variety of
applications. A hash table is a data structure that:

\begin{itemize}
\tightlist
\item
  Is disjoint from all other types.
\item
  Provides a mapping from objects known as \emph{keys} to corresponding
  objects known as \emph{values}.

  \begin{itemize}
  \tightlist
  \item
    Keys may be any Scheme objects in some kinds of hash tables, but are
    restricted in other kinds.
  \item
    Values may be any Scheme objects.
  \end{itemize}
\item
  Has no intrinsic order for the key-value \emph{associations} it
  contains.
\item
  Provides an \emph{equality predicate} which defines when a proposed
  key is the same as an existing key. No table may contain more than one
  value for a given key.
\item
  Provides a \emph{hash function} which maps a candidate key into a
  non-negative exact integer.
\item
  Supports mutation as the primary means of setting the contents of a
  table.
\item
  Provides key lookup and destructive update in (expected) amortized
  constant time, provided a satisfactory hash function is available.
\item
  Does not guarantee that whole-table operations work in the presence of
  concurrent mutation of the whole hash table (values may be safely
  mutated).
\end{itemize}

\setlibname{hash-table}


\subsubsection{Pronunciation}\label{Pronunciation}

The slash in the names of some procedures can be pronounced ``with''.

\subsection{Specification}\label{specification}

The procedures in this SRFI are in the \texttt{(srfi\ 125)} library (or
\texttt{(srfi\ :125)} on R6RS).

All references to ``executing in expected amortized constant time''
presuppose that a satisfactory hash function is available. Arbitrary or
impure hash functions can make a hash of any implementation.

Hash tables are allowed to cache the results of calling the equality
predicate and hash function, so programs cannot rely on the hash
function being called exactly once for every primitive hash table
operation: it may be called zero, one, or more times.

It is an error if the procedure argument of \texttt{hash-table-find},
\texttt{hash-table-count}, \texttt{hash-table-map},
\texttt{hash-table-for-each}, \texttt{hash-table-map!},
\texttt{hash-table-map-\textgreater{}list}, \texttt{hash-table-fold} or
\texttt{hash-table-prune!} mutates the hash table being walked.

It is an error to pass two hash tables that have different comparators
or equality predicates to any of the procedures of this SRFI.

Implementations are permitted to ignore user-specified hash functions in
certain circumstances. Specifically, if the equality predicate, whether
passed as part of a comparator or explicitly, is more fine-grained (in
the sense of R7RS-small section 6.1) than \texttt{equal?}, the
implementation is free --- indeed, is encouraged --- to ignore the
user-specified hash function and use something implementation-dependent.
This allows the use of addresses as hashes, in which case the keys must
be rehashed if they are moved by the garbage collector. Such a hash
function is unsafe to use outside the context of implementation-provided
hash tables. It can of course be exposed by an implementation as an
extension, with suitable warnings against inappropriate uses.

It is an error to mutate a key during or after its insertion into a hash
table in such a way that the hash function of the table will return a
different result when applied to that key.

\subsubsection{Index}\label{Index}

\begin{itemize}
\tightlist
\item
  \href{http://srfi.schemers.org/srfi-125/srfi-125.html\#Constructors}{Constructors}:
  \texttt{make-hash-table}, \texttt{hash-table},
  \texttt{hash-table-unfold}, \texttt{alist-\textgreater{}hash-table}
\end{itemize}

\begin{itemize}
\tightlist
\item
  \href{http://srfi.schemers.org/srfi-125/srfi-125.html\#Predicates}{Predicates}:
  \texttt{hash-table?}, \texttt{hash-table-contains?},
  \texttt{hash-table-exists?} (deprecated), \texttt{hash-table-empty?},
  \texttt{hash-table=?}, \texttt{hash-table-mutable?}
\end{itemize}

\begin{itemize}
\tightlist
\item
  \href{http://srfi.schemers.org/srfi-125/srfi-125.html\#Accessors}{Accessors}:
  \texttt{hash-table-ref}, \texttt{hash-table-ref/default}
\end{itemize}

\begin{itemize}
\tightlist
\item
  \href{http://srfi.schemers.org/srfi-125/srfi-125.html\#Mutators}{Mutators}:
  \texttt{hash-table-set!}, \texttt{hash-table-delete!},
  \texttt{hash-table-intern!}, \texttt{hash-table-update!},
  \texttt{hash-table-update!/default}, \texttt{hash-table-pop!},
  \texttt{hash-table-clear!}
\end{itemize}

\begin{itemize}
\tightlist
\item
  \href{http://srfi.schemers.org/srfi-125/srfi-125.html\#Thewholehashtable}{The
  whole hash table}: \texttt{hash-table-size}, \texttt{hash-table-keys},
  \texttt{hash-table-values}, \texttt{hash-table-entries},
  \texttt{hash-table-find}, \texttt{hash-table-count}
\end{itemize}

\begin{itemize}
\tightlist
\item
  \href{http://srfi.schemers.org/srfi-125/srfi-125.html\#Mappingandfolding}{Mapping
  and folding}: \texttt{hash-table-map}, \texttt{hash-table-for-each},
  \texttt{hash-table-walk} (deprecated), \texttt{hash-table-map!},
  \texttt{hash-table-map-\textgreater{}list}, \texttt{hash-table-fold},
  \texttt{hash-table-prune!}
\end{itemize}

\begin{itemize}
\tightlist
\item
  \href{http://srfi.schemers.org/srfi-125/srfi-125.html\#Copyingandconversion}{Copying
  and conversion}: \texttt{hash-table-copy},
  \texttt{hash-table-empty-copy},
  \texttt{hash-table-\textgreater{}alist}
\end{itemize}

\begin{itemize}
\tightlist
\item
  \href{http://srfi.schemers.org/srfi-125/srfi-125.html\#Hashtablesassets}{Hash
  tables as sets}: \texttt{hash-table-union!},
  \texttt{hash-table-merge!} (deprecated),
  \texttt{hash-table-intersection!}, \texttt{hash-table-difference!},
  \texttt{hash-table-xor!}
\end{itemize}

\begin{itemize}
\tightlist
\item
  \href{http://srfi.schemers.org/srfi-125/srfi-125.html\#Hashfunctionsandreflectivity}{Hash
  functions and reflectivity} (deprecated): \texttt{hash},
  \texttt{string-hash}, \texttt{string-ci-hash},
  \texttt{hash-by-identity}, \texttt{hash-table-equivalence-function},
  \texttt{hash-table-hash-function}
\end{itemize}

\subsubsection{Constructors}\label{Constructors}

\begin{entry}{%
  \Proto{make-hash-table}{ comparator arg
    \ldots}{procedure}{hash-table???}
  \Rproto{make-hash-table}{ comparator}{procedure}{hash-table???}
  \Rproto{make-hash-table}{equality-predicate hash-function arg
    \ldots}{procedure}{hash-table???}
  \Rproto{make-hash-table}{equality-predicate hash-function}{procedure}{hash-table???}
  \Rproto{make-hash-table}{equality-predicate arg \ldots}{procedure}{hash-table???}
  \Rproto{make-hash-table}{equality-predicate}{procedure}{hash-table???}}

  Returns a newly allocated hash table whose equality predicate and
  hash function are extracted from \emph{comparator}. Alternatively,
  for backward compatibility with SRFI 69 the equality predicate and
  hash function can be passed as separate arguments; this usage is
  deprecated.

  As mentioned above, implementations are free to use an appropriate
  implementation-dependent hash function instead of the specified hash
  function, provided that the specified equality predicate is a
  refinement of the \texttt{equal?} predicate. This applies whether
  the hash function and equality predicate are passed as separate
  arguments or packaged up into a comparator.

  If an equality predicate rather than a comparator is provided, the
  ability to omit the \emph{hash-function} argument is severely
  limited.  The implementation must provide hash functions appropriate
  for use with the predicates \texttt{eq?}, \texttt{eqv?},
  \texttt{equal?}, \texttt{string=?}, and \texttt{string-ci=?}, and
  may extend this list.  But if any unknown equality predicate is
  provided without a hash function, an error should be signaled. The
  constraints on equality predicates and hash functions are given in
  \href{http://srfi.schemers.org/srfi-128/srfi-128.html}{SRFI 128}.

  The meaning of any further arguments is implementation-dependent.
  However, implementations which support the ability to specify the
  initial capacity of a hash table should interpret a non-negative
  exact integer as the specification of that capacity. In addition, if
  the symbols \texttt{thread-safe}, \texttt{weak-keys},
  \texttt{ephemeral-keys}, \texttt{weak-values}, or
  \texttt{ephemeral-values} are present, implementations should create
  thread-safe hash tables, hash tables with weak keys or ephemeral
  keys, or hash tables with weak or ephemeral values respectively.
  Implementations are free to use ephemeral keys or values when weak
  keys or values respectively have been requested. To avoid collision
  with the \emph{hash-function} argument, none of these arguments can
  be procedures.

  (R6RS \texttt{make-eq-hashtable}, \texttt{make-eqv-hashtable}, and
  \texttt{make-hashtable}; Common Lisp \texttt{make-hash-table})
\end{entry}

\begin{entry}{%
  \Proto{hash-table}{ comparator key value \ldots}{procedure}{hash-table???}}

  \TODO{I've been converting entry descriptions to match R7RS, with
    separate prototypes for omitted arguments, that doesn't work here,
    because the number of keys and values doesn't match. Fix somehow.}

  Returns a newly allocated hash table, created as if by
  \texttt{make-hash-table} using \emph{comparator}. For each pair of
  arguments, an association is added to the new hash table with
  \emph{key} as its key and \emph{value} as its value. If the
  implementation supports immutable hash tables, this procedure
  returns an immutable hash table.  If the same key (in the sense of
  the equality predicate) is specified more than once, it is an error.
\end{entry}

\begin{entry}{%
  \Proto{hash-table-unfold}{ stop? mapper successor seed
    comparator arg \ldots}{procedure}{???}}

  \TODO{This one looks garbled from the Pandoc conversion. Check
    against SRFI.}

  Create a new hash table as if by \texttt{make-hash-table} using
  \emph{comparator} and the \emph{args}. If the result of applying the
  predicate \emph{stop?} to \emph{seed} is true, return the hash
  table.  Otherwise, apply the procedure \emph{mapper} to \emph{seed}.
  \emph{Mapper} returns two values, which are inserted into the hash
  table as the key and the value respectively. Then get a new seed by
  applying the procedure \emph{successor} to \emph{seed}, and repeat
  this algorithm.
\end{entry}

\begin{entry}{%
  \Proto{alist->hash-table}{ alist comparator arg \ldots}{procedure}{hash-table???}
  \Rproto{alist->hash-table}{ alist equality-predicate hash-function
    arg \ldots}{procedure}{hash-table???}
  \Rproto{alist->hash-table}{ alist equality-predicate hash-function}{procedure}{hash-table???}
  \Rproto{alist->hash-table}{ alist equality-predicate arg \ldots}{procedure}{hash-table???}
  \Rproto{alist->hash-table}{ alist equality-predicate}{procedure}{hash-table???}}

  Returns a newly allocated hash-table as if by
  \texttt{make-hash-table} using \emph{comparator} and the
  \emph{args}. It is then initialized from the associations of
  \emph{alist}. Associations earlier in the list take precedence over
  those that come later. The second form \TODO{and subsequent forms} is for compatibility with
  SRFI 69, and is deprecated.
\end{entry}

\subsubsection{Predicates}\label{Predicates}

\begin{entry}{%
  \Proto{hash-table?}{ obj}{procedure}{boolean???}}

  Returns \texttt{\#t} if \emph{obj} is a hash table, and \texttt{\#f}
  otherwise. (R6RS \texttt{hashtable?}; Common Lisp
  \texttt{hash-table-p})
\end{entry}

\begin{entry}{%
  \Proto{hash-table-contains?}{ hash-table key}{procedure}{boolean???}
  \Proto{hash-table-exists?}{ hash-table key}{procedure}{boolean???}}

  Returns \texttt{\#t} if there is any association to \emph{key} in
  \emph{hash-table}, and \texttt{\#f} otherwise. Must execute in
  expected amortized constant time. The \texttt{hash-table-exists?}
  procedure is the same as \texttt{hash-table-contains?}, is provided
  for backward compatibility with SRFI 69, and is deprecated. (R6RS
  \texttt{hashtable-contains?})
\end{entry}

\begin{entry}{%
  \Proto{hash-table-empty?}{ hash-table}{procedure}{boolean???}}

  Returns \texttt{\#t} if \emph{hash-table} contains no associations,
  and \texttt{\#f} otherwise.
\end{entry}

\begin{entry}{%
  \Proto{hash-table=?}{ value-comparator hash-table$_1$
    hash-table$_2$}{procedure}{boolean???}} 

  Returns \texttt{\#t} if \emph{hash-table$_1$} and
  \emph{hash-table$_2$} have the same keys (in the sense of their
  common equality predicate) and each key has the same value (in the
  sense of \emph{value-comparator}), and \texttt{\#f} otherwise.
\end{entry}

\begin{entry}{%
  \Proto{hash-table-mutable?}{ hash-table}{procedure}{boolean???}}

  Returns \texttt{\#t} if the hash table is mutable. Implementations
  may or may not support immutable hash tables. (R6RS
  \texttt{hashtable-mutable?})
\end{entry}

\subsubsection{Accessors}\label{Accessors}

The following procedures, given a key, return the corresponding value.

\begin{entry}{%
  \Proto{hash-table-ref}{ hash-table key failure
    success}{procedure}{value???}
  \Rproto{hash-table-ref}{ hash-table key failure}{procedure}{value???}
  \Rproto{hash-table-ref}{ hash-table key}{procedure}{value???}}

  Extracts the value associated to \emph{key} in \emph{hash-table},
  invokes the procedure \emph{success} on it, and returns its result;
  if \emph{success} is not provided, then the value itself is
  returned. If \emph{key} is not contained in \emph{hash-table} and
  \emph{failure} is supplied, then \emph{failure} is invoked on no
  arguments and its result is returned. Otherwise, it is an
  error. Must execute in expected amortized constant time, not
  counting the time to call the procedures.  SRFI 69 does not support
  the \emph{success} procedure.
\end{entry}

\begin{entry}{%
  \Proto{hash-table-ref/default}{ hash-table key default}{procedure}{value???}}

  Semantically equivalent to, but may be more efficient than, the
  following code:

\begin{verbatim}
  (hash-table-ref hash-table key (lambda () default))
\end{verbatim}

(R6RS \texttt{hashtable-ref}; Common Lisp \texttt{gethash})
\end{entry}

\subsubsection{Mutators}\label{Mutators}

The following procedures alter the associations in a hash table either
unconditionally, or conditionally on the presence or absence of a
specified key. It is an error to add an association to a hash table
whose key does not satisfy the type test predicate of the comparator
used to create the hash table.

\begin{entry}{%
  \Proto{hash-table-set!}{ hash-table arg \ldots}{procedure}{unspecified}}

  Repeatedly mutates \emph{hash-table}, creating new associations in
  it by processing the arguments from left to right. The \emph{args}
  alternate between keys and values. Whenever there is a previous
  association for a key, it is deleted. It is an error if the type
  check procedure of the comparator of \emph{hash-table}, when invoked
  on a key, does not return \texttt{\#t}. Likewise, it is an error if
  a key is not a valid argument to the equality predicate of
  \emph{hash-table}. Returns an unspecified value. Must execute in
  expected amortized constant time per key. SRFI 69, R6RS
  \texttt{hashtable-set!} and Common Lisp \texttt{(setf\ gethash)} do
  not handle multiple associations.
\end{entry}

\begin{entry}{%
  \Proto{hash-table-delete!}{ hash-table key \ldots}{procedure}{integer}}

  Deletes any association to each \emph{key} in \emph{hash-table} and
  returns the number of keys that had associations. Must execute in
  expected amortized constant time per key. SRFI 69, R6RS
  \texttt{hashtable-delete!}, and Common Lisp \texttt{remhash} do not
  handle multiple associations.
\end{entry}

\begin{entry}{%
  \Proto{hash-table-intern!}{ hash-table key failure}{procedure}{value???}}

  Effectively invokes \texttt{hash-table-ref} with the given arguments
  and returns what it returns. If \emph{key} was not found in
  \emph{hash-table}, its value is set to the result of calling
  \emph{failure}. Must execute in expected amortized constant time.

  \TODO{If failure is invoked, is that the return value of
    hash-table-intern!?}
\end{entry}

\begin{entry}{%
  \Proto{hash-table-update!}{ hash-table key updater failure success}{procedure}{unspecified???}
  \Rproto{hash-table-update!}{ hash-table key updater failure}{procedure}{unspecified???}
  \Rproto{hash-table-update!}{ hash-table key updater}{procedure}{unspecified???}}

  Semantically equivalent to, but may be more efficient than, the
  following code:

\begin{quote}
  \texttt{(hash-table-set!}\emph{hash-table key}\texttt{\
    (}\emph{updater} \texttt{(hash-table-ref}\emph{hash-table key
    failure success}\texttt{)))}
\end{quote}

Must execute in expected amortized constant time. Returns an
unspecified value. (SRFI 69 and R6RS \texttt{hashtable-update!} do not
support the \emph{success} procedure)
\end{entry}

\begin{entry}{%
  \Proto{hash-table-update!/default}{ hash-table key updater
    default}{procedure}{unspecified???}}

  Semantically equivalent to, but may be more efficient than, the
  following code:

\begin{quote}
  \texttt{(hash-table-set!}\emph{hash-table key}\texttt{\
    (}\emph{updater} \texttt{(hash-table-ref/default\
  }\emph{hash-table key default}\texttt{)))}
\end{quote}

Must execute in expected amortized constant time. Returns an
unspecified value.
\end{entry}

\begin{entry}{%
  \Proto{hash-table-pop!}{ hash-table}{procedure}{key??? value???}}

  Chooses an arbitrary association from \emph{hash-table} and removes
  it, returning the key and value as two values.
\end{entry}

\begin{entry}{%
  \Proto{hash-table-clear!}{ hash-table}{procedure}{unspecified???}}

  Delete all the associations from \emph{hash-table}. (R6RS
  \texttt{hashtable-clear!}; Common Lisp \texttt{clrhash})
\end{entry}

\subsubsection{The whole hash table}\label{Thewholehashtable}

These procedures process the associations of the hash table in an
unspecified order.

\begin{entry}{%
  \Proto{hash-table-size}{ hash-table}{procedure}{integer???}}

  Returns the number of associations in \emph{hash-table} as an exact
  integer. Should execute in constant time. (R6RS
  \texttt{hashtable-size}; Common Lisp \texttt{hash-table-count}.)
\end{entry}

\begin{entry}{%
  \Proto{hash-table-keys}{ hash-table}{procedure}{list}}

  Returns a newly allocated list of all the keys in \emph{hash-table}.
  R6RS \texttt{hashtable-keys} returns a vector.
\end{entry}

\begin{entry}{%
  \Proto{hash-table-values}{ hash-table}{procedure}{list}}

  Returns a newly allocated list of all the keys in \emph{hash-table}.
\end{entry}

\begin{entry}{%
  \Proto{hash-table-entries}{ hash-table}{procedure}{list list}}

  Returns two values, a newly allocated list of all the keys in
  \emph{hash-table} and a newly allocated list of all the values in
  \emph{hash-table} in the corresponding order. R6RS
  \texttt{hash-table-entries} returns vectors.
\end{entry}

\begin{entry}{%
  \Proto{hash-table-find}{ proc hash-table failure}{procedure}{value
    \textrm{or} \schfalse???}}

  For each association of \emph{hash-table}, invoke \emph{proc} on its
  key and value. If \emph{proc} returns true, then
  \texttt{hash-table-find} returns what \emph{proc} returns. If all
  the calls to \emph{proc} return \texttt{\#f}, return the result of
  invoking the thunk \emph{failure}.
\end{entry}

\begin{entry}{%
  \Proto{hash-table-count}{ pred hash-table}{procedure}{integer???}}

For each association of \emph{hash-table}, invoke \emph{pred} on its key
and value. Return the number of calls to \emph{pred} which returned
true.
\end{entry}

\subsubsection{Mapping and folding}\label{Mappingandfolding}

These procedures process the associations of the hash table in an
unspecified order.

\begin{entry}{%
  \Proto{hash-table-map}{ proc comparator hash-table}{procedure}{hash-table???}}

  Returns a newly allocated hash table as if by
  \texttt{(make-hash-table}~\emph{comparator}\texttt{)}. Calls
  \emph{proc} for every association in \emph{hash-table} with the
  value of the association. The key of the association and the result
  of invoking \emph{proc} are entered into the new hash table. Note
  that this is \emph{not} the result of lifting mapping over the
  domain of hash tables, but it is considered more useful.
\end{entry}

\begin{entry}{%
  \Proto{hash-table-for-each}{ proc hash-table}{procedure}{unspecified???}
  \Proto{hash-table-walk}{ hash-table proc}{procedure}{unspecified}}

  Calls \emph{proc} for every association in \emph{hash-table} with
  two arguments: the key of the association and the value of the
  association.  The value returned by \emph{proc} is
  discarded. Returns an unspecified value. The
  \texttt{hash-table-walk} procedure is equivalent to
  \texttt{hash-table-for-each} with the arguments reversed, is
  provided for backward compatibility with SRFI 69, and is
  deprecated. (Common Lisp \texttt{maphash})
\end{entry}

\begin{entry}{%
  \Proto{hash-table-map!}{ proc hash-table}{procedure}{unspecified???}}

  Calls \emph{proc} for every association in \emph{hash-table} with
  two arguments: the key of the association and the value of the
  association.  The value returned by \emph{proc} is used to update
  the value of the association. Returns an unspecified value.
\end{entry}

\begin{entry}{%
  \Proto{hash-table-map->list}{ proc hash-table}{procedure}{list}}

  Calls \emph{proc} for every association in \emph{hash-table} with
  two arguments: the key of the association and the value of the
  association.  The values returned by the invocations of \emph{proc}
  are accumulated into a list, which is returned.
\end{entry}

\begin{entry}{%
  \Proto{hash-table-fold}{ proc seed hash-table}{procedure}{value???}
  \Proto{hash-table-fold}{ hash-table proc seed}{procedure}{value???}}

  Calls \emph{proc} for every association in \emph{hash-table} with
  three arguments: the key of the association, the value of the
  association, and an accumulated value \emph{val}. \emph{Val} is
  \emph{seed} for the first invocation of \emph{procedure}, and for
  subsequent invocations of \emph{proc}, the returned value of the
  previous invocation. The value returned by \texttt{hash-table-fold}
  is the return value of the last invocation of \emph{proc}. The order
  of arguments with \emph{hash-table} as the first argument is
  provided for SRFI 69 compatibility, and is deprecated.
\end{entry}

\begin{entry}{%
  \Proto{hash-table-prune!}{ proc hash-table}{procedure}{unspecified???}}

  Calls \emph{proc} for every association in \emph{hash-table} with
  two arguments, the key and the value of the association, and removes
  all associations from \emph{hash-table} for which \emph{proc}
  returns true.  Returns an unspecified value.
\end{entry}

\subsubsection{Copying and conversion}

\begin{entry}{%
  \Proto{hash-table-copy}{ hash-table
    mutable?}{procedure}{hash-table???}
  \Proto{hash-table-copy}{ hash-table?}{procedure}{hash-table???}}

  Returns a newly allocated hash table with the same properties and
  associations as \emph{hash-table}. If the second argument is present
  and is true, the new hash table is mutable. Otherwise it is
  immutable provided that the implementation supports immutable hash
  tables. SRFI 69 \texttt{hash-table-copy} does not support a second
  argument. (R6RS \texttt{hashtable-copy})
\end{entry}

\begin{entry}{%
  \Proto{hash-table-empty-copy}{ hash-table}{procedure}{hash-table???}}

  Returns a newly allocated mutable hash table with the same
  properties as \emph{hash-table}, but with no associations.
\end{entry}

\begin{entry}{%
  \Proto{hash-table->alist}{ hash-table}{procedure}{alist???}}

  Returns an alist with the same associations as \emph{hash-table} in
  an unspecified order.
\end{entry}

\subsubsection{Hash tables as sets}\label{Hashtablesassets}

\begin{entry}{%
  \Proto{hash-table-union!}{ hash-table$_1$
    hash-table$_2$}{procedure}{hash-table???}
  \Proto{hash-table-merge!}{ hash-table$_1$
    hash-table$_2$}{procedure}{hash-table???}}

  Adds the associations of \emph{hash-table$_2$} to
  \emph{hash-table$_1$} and returns \emph{hash-table$_1$}. If a key
  appears in both hash tables, its value is set to the value appearing
  in \emph{hash-table$_1$}. Returns \emph{hash-table$_1$}. The
  \texttt{hash-table-merge!}  procedure is the same as
  \texttt{hash-table-union!}, is provided for compatibility with SRFI
  69, and is deprecated.
\end{entry}

\begin{entry}{%
  \Proto{hash-table-intersection!}{ hash-table$_1$
    hash-table$_2$}{procedure}{hash-table???}}

  Deletes the associations from \emph{hash-table$_1$} which don't also
  appear in \emph{hash-table$_2$} and returns \emph{hash-table$_1$}.
\end{entry}

\begin{entry}{%
  \Proto{hash-table-difference!}{ hash-table$_1$
    hash-table$_2$}{procedure}{hash-table}}

  Deletes the associations of \emph{hash-table$_1$} whose keys are
  also present in \emph{hash-table$_2$} and returns
  \emph{hash-table$_1$}.
\end{entry}

\begin{entry}{%
  \Proto{hash-table-xor!}{ hash-table$_1$ hash-table$_2$}{procedure}{hash-table}}

  Deletes the associations of \emph{hash-table$_1$} whose keys are
  also present in \emph{hash-table$_2$}, and then adds the
  associations of \emph{hash-table$_2$} whose keys are not present in
  \emph{hash-table$_1$} to \emph{hash-table$_1$}. Returns
  \emph{hash-table$_1$}.
\end{entry}

\subsubsection{Hash functions and
reflectivity}\label{Hashfunctionsandreflectivity}

\TODO{If we are removing deprecated features, then this whole section
  goes. If it remains, I will reformat it.}

These functions are made part of this SRFI solely for compatibility with
SRFI 69, and are deprecated.

\texttt{(hash\ }\emph{obj}\texttt{)}

The same as SRFI 128's \texttt{default-hash} procedure, except that it
must accept (and should ignore) an optional second argument.

\texttt{(string-hash\ }\emph{obj}\texttt{)}

Similar to SRFI 128's \texttt{string-hash} procedure, except that it
must accept (and should ignore) an optional second argument. It is
incompatible with the procedure of the same name exported by SRFI 128
and \href{http://srfi.schemers.org/srfi-126/srfi-126.html}{SRFI 126}.

\texttt{(string-ci-hash\ }\emph{obj}\texttt{)}

Similar to SRFI 128's \texttt{string-ci-hash} procedure, except that it
must accept (and should ignore) an optional second argument. It is
incompatible with the procedure of the same name exported by SRFI 128
and \href{http://srfi.schemers.org/srfi-126/srfi-126.html}{SRFI 126}.

\texttt{(hash-by-identity\ }\emph{obj}\texttt{)}

The same as SRFI 128's \texttt{default-hash} procedure, except that it
must accept (and should ignore) an optional second argument. However, if
the implementation replaces the hash function associated with the
\texttt{eq?} predicate with an implementation-dependent alternative, it
is an error to call this procedure at all.

\texttt{(hash-table-equivalence-function\ }\emph{hash-table}\texttt{)}

Returns the equivalence procedure used to create \emph{hash-table}.

\texttt{(hash-table-hash-function\ }\emph{hash-table}\texttt{)}

Returns the hash function used to create \emph{hash-table}. However, if
the implementation has replaced the user-specified hash function with an
implementation-specific alternative, the implementation may return
\texttt{\#f} instead.

Editor:
\href{mailto:srfi-editors\%20at\%20srfi\%20dot\%20schemers\%20dot\%20org}{Arthur
A. Gleckler}
