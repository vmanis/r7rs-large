\section{Generators}
\redno{12} What generator library should R7RS-large provide?

\begin{quote}
SRFI 121 is a collection of routines for manipulating generators, which
are nullary procedures that are invoked repeatedly to generate a
sequence of values. They provide lightweight laziness.
\end{quote}


This SRFI defines utility procedures that create, transform, and consume
generators. A generator is simply a procedure with no arguments that
works as a source of a series of values. Every time it is called, it
yields a value. Generators may be finite or infinite; a finite generator
returns an end-of-file object to indicate that it is exhausted. For
example, \texttt{read-char}, \texttt{read-line}, and \texttt{read} are
generators that generate characters, lines, and objects from the current
input port. Generators provide lightweight laziness.

\setlibname{generator}


Generators can be divided into two classes, finite and infinite. Both
kinds of generators can be invoked an indefinite number of times. After
a finite generator has generated all its values, it will return an
end-of-file object for all subsequent calls. A generator is said to be
\emph{exhausted} if calling it will return an end-of-file object. By
definition, infinite generators can never be exhausted.

A generator is said to be in an \emph{undefined state} if it cannot be
determined exactly how many values it has generated. This arises because
it is impossible to tell by inspecting a generator whether it is
exhausted or not. For example,
\texttt{(generator-fold\ +\ 0\ (generator\ 1\ 2\ 3)\ (generator\ 1\ 2))}
will compute 0 + 1 + 1 + 2 + 2 = 6, at which time the second generator
will be exhausted. If the first generator is invoked, however, it may
return either 3 or an end-of-file object, depending on whether the
implementation of \texttt{generator-fold} has invoked it or not.
Therefore, the first generator is said to be in an undefined state.

\subsection{Generator constructors}\label{generator-constructors}

The result of a generator constructor is just a procedure, so printing
it doesn't show much. In the examples in this section we use
\texttt{generator->list} to convert the generator to a
list.

These procedures have names ending with \texttt{generator}.


\begin{entry}{%
  \Proto{generator}{ arg \ldots}{procedure}{generator}}

The simplest finite generator. Generates each of its arguments in turn.
When no arguments are provided, it returns an empty generator that
generates no values.
\end{entry}

\begin{entry}{%
  \Proto{make-iota-generator}{ count start step}{procedure}{generator}
  \Proto{make-iota-generator}{ count start}{procedure}{generator}
  \Proto{make-iota-generator}{ count}{procedure}{generator}}

Creates a finite generator of a sequence of count numbers. The sequence
begins with start (which defaults to 0) and increases by step (which
defaults to 1). If both start and step are exact, it generates exact
numbers; otherwise it generates inexact numbers. The exactness of count
doesn't affect the exactness of the results.

\begin{verbatim}
(generator->list (make-iota-generator 3 8))
  ⇒ (8 9 10)
\end{verbatim}

\begin{verbatim}
(generator->list (make-iota-generator 3 8 2))
  ⇒ (8 10 12)
\end{verbatim}
\end{entry}

\begin{entry}{%
  \Proto{make-range-generator}{ start end step}{procedure}{generator}
  \Rproto{make-range-generator}{ start end}{procedure}{generator}
  \Rproto{make-range-generator}{ start}{procedure}{generator}}

  Creates a generator of a sequence of numbers. The sequence
  begins with start, increases by step (default 1), and continues
  while the number is less than end, or forever if end is omitted. If
  both start and step are exact, it generates exact numbers; otherwise
  it generates inexact numbers. The exactness of end doesn't affect
  the exactness of the results.

\begin{verbatim}
(generator->list (make-range-generator 3) 4)
  ⇒ (3 4 5 6)
\end{verbatim}
\begin{verbatim}
(generator->list (make-range-generator 3 8))
  ⇒ (3 4 5 6 7)
\end{verbatim}
\begin{verbatim}
(generator->list (make-range-generator 3 8 2))
  ⇒ (3 5 7)
\end{verbatim}
\end{entry}

\begin{entry}{%
  \Proto{make-coroutine-generator}{ proc}{procedure}{generator}}

  Creates a
  generator from a coroutine.

  The proc argument is a procedure that takes one argument,
  yield. When called, \texttt{make-coroutine-generator} immediately
  returns a generator g. When g is called, proc runs until it calls
  yield. Calling yield causes the execution of proc to be suspended,
  and g returns the value passed to yield.

  Whether this generator is finite or infinite depends on the behavior
  of proc. If proc returns, it is the end of the sequence --- g
  returns an end-of-file object from then on. The return value of proc
  is ignored.

  The following code creates a generator that produces a series 0, 1,
  and 2 (effectively the same as \texttt{(make-range-generator\ 0\ 3)}
  and binds it to \texttt{g}.

\begin{verbatim}
(define g
  (make-coroutine-generator
   (lambda (yield) (let loop ((i 0))
               (when (< i 3) (yield i) (loop (+ i 1)))))))

(generator->list g) ⇒ (0 1 2)
\end{verbatim}
\end{entry}


\begin{entry}{%
  \Proto{list>generator}{ lis}{procedure}{generator}  
  \Proto{vector>generator}{ vec start end}{procedure}{generator}
  \Rproto{vector>generator}{ vec start}{procedure}{generator}
  \Rproto{vector>generator}{ vec}{procedure}{generator}
  \Proto{reverse-vector>generator}{ vec start end}{procedure}{generator}
  \Rproto{reverse-vector>generator}{ vec start}{procedure}{generator}
  \Rproto{reverse-vector>generator}{ vec}{procedure}{generator}
  \Proto{string>generator}{ str start end}{procedure}{generator}
  \Rproto{string>generator}{ str start}{procedure}{generator}
  \Rproto{string>generator}{ str}{procedure}{generator}
  \Proto{bytevector>generator}{ bytevector start end}{procedure}{generator}
  \Proto{bytevector>generator}{ bytevector start}{procedure}{generator}
  \Proto{bytevector>generator}{ bytevector}{procedure}{generator}} 

 These procedures return generators that yield each element of
  the given argument. Mutating the underlying object will affect the
  results of the generator.

\begin{verbatim}
(generator->list (list->generator '(1 2 3 4 5)))
  ⇒ (1 2 3 4 5)
(generator->list (vector->generator '#(1 2 3 4 5)))
  ⇒ (1 2 3 4 5)
(generator->list (reverse-vector->generator '#(1 2 3 4 5)))
  ⇒ (5 4 3 2 1)
(generator->list (string->generator "abcde"))
  ⇒ (#\a #\b #\c #\d #\e)
\end{verbatim}

The generators returned by the constructors are exhausted once all
elements are retrieved; the optional start-th and end-th arguments can
limit the range the generator walks across.

For \texttt{reverse-vector->generator}, the first value is the element
right before the end-th element, and the last value is the start-th
element. For all the other constructors, the first value the generator
yields is the start-th element, and it ends right before the end-th
element.

\begin{verbatim}
(generator->list (vector->generator '#(a b c d e) 2))
  ⇒ (c d e)
(generator->list (vector->generator '#(a b c d e) 2 4))
  ⇒ (c d)
(generator->list (reverse-vector->generator '#(a b c d e) 2))
  ⇒ (e d c)
(generator->list (reverse-vector->generator '#(a b c d e) 2 4))
  ⇒ (d c)
(generator->list (reverse-vector->generator '#(a b c d e) 0 2))
  ⇒ (b a)
\end{verbatim}
\end{entry}

\begin{entry}{%
  \Proto{make-for-each-generator}{ for-each obj}{procedure}{generator}}

  A
  generator constructor that converts any collection obj to a
  generator that returns its elements using a for-each procedure
  appropriate for obj. This must be a procedure that when called as
  (for-each proc obj) calls proc on each element of obj. Examples of
  such procedures are \texttt{for-each}, \texttt{string-for-each}, and
  \texttt{vector-for-each} from R7RS. The value returned by for-each
  is ignored. The generator is finite if the collection is finite,
  which would typically be the case.

  The collections need not be conventional ones (lists, strings, etc.)
  as long as \emph{for-each} can invoke a procedure on everything that
  counts as a member. For example, the following procedure allows
  \texttt{for-each-generator} to generate the digits of an integer
  from least to most significant:

\begin{verbatim}
(define (for-each-digit proc n)
  (when (> n 0)
    (let-values (((div rem) (truncate/ n 10)))
      (proc rem)
      (for-each-digit proc div))))
\end{verbatim}
\end{entry}

\begin{entry}{%
  \Proto{make-unfold-generator}{ stop? mapper successor seed}{procedure}{}}

A generator constructor similar to
\href{http://srfi.schemers.org/srfi-1/srfi-1.html}{SRFI 1's}
\texttt{unfold}.

The stop? predicate takes a seed value and determines whether to stop.
The mapper procedure calculates a value to be returned by the generator
from a seed value. The successor procedure calculates the next seed
value from the current seed value.

For each call of the resulting generator, stop? is called with the
current seed value. If it returns true, then the generator returns an
end-of-file object. Otherwise, it applies mapper to the current seed
value to get the value to return, and uses successor to update the seed
value.

This generator is finite unless stop? never returns true.

\begin{verbatim}
(generator->list (make-unfold-generator
                      (lambda (s) (> s 5))
                      (lambda (s) (* s 2))
                      (lambda (s) (+ s 1))
                      0))
  ⇒ (0 2 4 6 8 10)
\end{verbatim}
\end{entry}

\subsection{Generator operations}

The following procedures accept one or more generators and return a new
generator without consuming any elements from the source generator(s).
In general, the result will be a finite generator if the arguments are.

The names of these procedures are prefixed with \texttt{g}.

\begin{entry}{%
  \Proto{gcons*}{ item \ldots{} gen}{procedure}{generator}}

Returns a generator that adds items in front of gen. Once the items have
been consumed, the generator is guaranteed to tail-call gen.

\begin{verbatim}
(generator->list (gcons* 'a 'b (make-range-generator 0 2)))
 ⇒ (a b 0 1)
\end{verbatim}
\end{entry}

\begin{entry}{%
  \Proto{gappend}{ gen \ldots}{procedure}{generator}}

Returns a generator that yields the items from the first given
generator, and once it is exhausted, from the second generator, and so
on.

\begin{verbatim}
(generator->list 
  (gappend (make-range-generator 0 3) 
           (make-range-generator 0 2)))
 ⇒ (0 1 2 0 1)

(generator->list (gappend))
 ⇒ ()
\end{verbatim}
\end{entry}

\begin{entry}{%
  \Proto{gcombine}{ proc seed gen gen$_2$ \ldots}{procedure}{}}

\TODO{clarify this}

A generator for mapping with state. It yields a sequence of sub-folds
over proc.

The proc argument is a procedure that takes as many arguments as the
input generators plus one. It is called as \texttt{(}proc
v$_1$ v$_2$ \ldots seed), where
v$_1$, v$_2$, \ldots are the values yielded
from the input generators, and seed is the current seed value. It must
return two values, the yielding value and the next seed. The result
generator is exhausted when any of the \emph{gen\textsubscript{n}}
generators is exhausted, at which time all the others are in an
undefined state.
\end{entry}

\begin{entry}{%
  \Proto{gfilter}{ pred gen}{procedure}{generator} 
  \Proto{gremove}{ pred gen}{procedure}{generator}}

Return generators that yield the items from the source generator, except
those on which pred answers false or true respectively.
\end{entry}

\begin{entry}{%
  \Proto{gtake}{ gen k padding}{procedure}{generator}
  \Rproto{gtake}{ gen k}{procedure}{generator}
  \Proto{gdrop}{ gen k}{procedure}{generator}}

These are generator analogues of SRFI 1 \texttt{take} and \texttt{drop}.
\texttt{Gtake} returns a generator that yields (at most) the first k
items of the source generator, while \texttt{gdrop} returns a generator
that skips the first k items of the source generator.

These won't complain if the source generator is exhausted before
generating k items. By default, the generator returned by \texttt{gtake}
terminates when the source generator does, but if you provide the
padding argument, then the returned generator will yield exactly k
items, using the padding value as needed to provide sufficient
additional values.
\end{entry}

\begin{entry}{%
  \Proto{gtake-while}{ pred gen}{procedure}{generator} 
  \Proto{gdrop-while}{ pred gen}{procedure}{generator}}

The generator analogues of SRFI-1 \texttt{take-while} and
\texttt{drop-while}. The generator returned from \texttt{gtake-while}
yields items from the source generator as long as pred returns true for
each. The generator returned from \texttt{gdrop-while} first reads and
discards values from the source generator while pred returns true for
them, then starts yielding items returned by the source.
\end{entry}

\begin{entry}{%
  \Proto{gdelete}{ item gen =}{procedure}{generator}
  \Proto{gdelete}{ item gen}{procedure}{generator}}

Creates a generator that returns whatever gen returns, except for any
items that are the same as item in the sense of =, which defaults to
\texttt{equal?}. The = predicate is passed exactly two arguments, of
which the first was generated by gen before the second.

\begin{verbatim}
(generator->list (gdelete 3 (generator 1 2 3 4 5 3 6 7)))
  ⇒ (1 2 4 5 6 7)
\end{verbatim}
\end{entry}

\begin{entry}{%
  \Proto{gdelete-neighbor-dups}{ gen =}{procedure}{generator}
  \Proto{gdelete-neighbor-dups}{ gen}{procedure}{generator}}

Creates a generator that returns whatever gen returns, except for any
items that are equal to the preceding item in the sense of =, which
defaults to \texttt{equal?}. The = predicate is passed exactly two
arguments, of which the first was generated by gen before the second.

\begin{verbatim}
(generator->list 
  (gdelete-neighbor-dups 
    (list->generator '(a a b c a a a d c))))
  ⇒ (a b c a d c)
\end{verbatim}
\end{entry}

\begin{entry}{%
  \Proto{gindex}{ value-gen index-gen}{procedure}{generator}}

Creates a generator that returns elements of value-gen specified by the
indices (non-negative exact integers) generated by index-gen. It is an
error if the indices are not strictly increasing, or if any index
exceeds the number of elements generated by value-gen. The result
generator is exhausted when either generator is exhausted, at which time
the other is in an undefined state.

\begin{verbatim}
(generator->list 
  (gindex 
    (list->generator '(a b c d e f))
    (list->generator '(0 2 4))))
  ⇒ (a c e)
\end{verbatim}
\end{entry}

\begin{entry}{%
  \Proto{gselect}{ value-gen truth-gen}{procedure}{generator}}

Creates a generator that returns elements of value-gen that correspond
to the values generated by truth-gen. If the current value of truth-gen
is true, the current value of value-gen is generated, but otherwise not.
The result generator is exhausted when either generator is exhausted, at
which time the other is in an undefined state.

\begin{verbatim}
(generator->list 
  (gselect 
    (list->generator '(a b c d e f))
    (list->generator '(#t #f #f #t #t #f))))
  ⇒ (a d e)
\end{verbatim}
\end{entry}

\subsection{Consuming generated values}\label{consuming-generated-values}

Unless otherwise noted, these procedures consume all the values
available from the generator that is passed to them, and therefore will
not return if one or more generator arguments are infinite. They have
names prefixed with \texttt{generator}.

\begin{entry}{%
  \Proto{generator->list}{ generator k}{procedure}{list}
  \Rproto{generator->list}{ generator}{procedure}{list}}

Reads items from generator and returns a newly allocated list of them.
By default, it reads until the generator is exhausted.

If an optional argument k is given, it must be a non-negative integer,
and the list ends when either k items are consumed, or \emph{generator}
is exhausted; therefore \emph{generator} can be infinite in this case.
\end{entry}

\begin{entry}{%
  \Proto{generator->reverse-list}{ generator k}{procedure}{list}
  \Rproto{generator->reverse-list}{ generator}{procedure}{list}}

Reads items from generator and returns a newly allocated list of them in
reverse order. By default, this reads until the generator is exhausted.

If an optional argument k is given, it must be a non-negative integer,
and the list ends when either k items are read, or \emph{generator} is
exhausted; therefore \emph{generator} can be infinite in this case.
\end{entry}

\begin{entry}{%
  \Proto{generator->vector}{ generator k}{procedure}{vector}
  \Rproto{generator->vector}{ generator}{procedure}{vector}}

Reads items from generator and returns a newly allocated vector of them.
By default, it reads until the generator is exhausted.

If an optional argument k is given, it must be a non-negative integer,
and the list ends when either k items are consumed, or \emph{generator}
is exhausted; therefore \emph{generator} can be infinite in this case.
\end{entry}

\begin{entry}{%
  \Proto{generator->vector!}{ vector at generator}{procedure}{integer}}

Reads items from generator and puts them into vector starting at index
at, until vector is full or generator is exhausted. \emph{Generator} can
be infinite. The number of elements generated is returned.
\end{entry}

\begin{entry}{%
  \Proto{generator->string}{ generator k}{procedure}{string}
  \Rproto{generator->string}{ generator}{procedure}{string}}

Reads items from generator and returns a newly allocated string of them.
It is an error if the items are not characters. By default, it reads
until the generator is exhausted.

If an optional argument k is given, it must be a non-negative integer,
and the string ends when either k items are consumed, or
\emph{generator} is exhausted; therefore \emph{generator} can be
infinite in this case.
\end{entry}

\begin{entry}{%
  \Proto{generator-fold}{ proc seed gen$_1$ gen$_2$ \ldots}{procedure}{value}}

Works like SRFI 1 \texttt{fold} on the values generated by the generator
arguments.

When one generator is given, for each value v generated by gen, proc is
called as \texttt{(proc\ v\ r)}, where r is the current accumulated
result; the initial value of the accumulated result is seed, and the
return value from proc becomes the next accumulated result. When gen is
exhausted, the accumulated result at that time is returned from
\texttt{generator-fold}.

When more than one generator is given, proc is invoked on the values
returned by all the generator arguments followed by the current
accumulated result. The procedure terminates when any of the
\emph{gen\textsubscript{n}} generators is exhausted, at which time all
the others are in an undefined state.

\begin{verbatim}
(with-input-from-string "a b c d e"
  (lambda () (generator-fold cons 'z read)))
  ⇒ (e d c b a . z)
\end{verbatim}
\end{entry}

\begin{entry}{%
  \Proto{generator-for-each}{ proc gen gen$_2$ \ldots}{procedure}{unspecified}}

A generator analogue of \texttt{for-each} that consumes generated values
using side effects. Repeatedly applies proc on the values yielded by
gen, gen$_2$ \ldots{} until any one of the generators is
exhausted. The values returned from proc are discarded. The procedure
terminates when any of the \emph{gen\textsubscript{n}} generators is
exhausted, at which time all the others are in an undefined state.
Returns an unspecified value.
\end{entry}

\begin{entry}{%
  \Proto{generator-find}{ pred gen}{procedure}{value}}

Returns the first item from the generator gen that satisfies the
predicate pred, or \texttt{\#f} if no such item is found before gen is
exhausted. If gen is infinite, \texttt{generator-find} will not return
if it cannot find an appropriate item.
\end{entry}

\begin{entry}{%
  \Proto{generator-count}{ pred gen}{procedure}{integer}}

Returns the number of items available from the generator gen that
satisfy the predicate pred.
\end{entry}

\begin{entry}{%
  \Proto{generator-any}{ pred gen}{procedure}{value}}

\TODO{This returns either a value from the generator, or
  false. However, as a return type, \textit{value}~or~\schfalse{}
  looks redundant.}

Applies \emph{pred} to each item from \emph{gen}. As soon as it yields a
true value, the value is returned without consuming the rest of
\emph{gen}. If \emph{gen} is exhausted, returns \texttt{\#f}.
\end{entry}

\begin{entry}{%
  \Proto{generator-every}{ pred gen}{procedure}{value}}

\TODO{\textit{value}~or~\schtrue?}

Applies \emph{pred} to each item from \emph{gen}. As soon as it yields a
false value, the value is returned without consuming the rest of
\emph{gen}. If \emph{gen} is exhausted, returns the last value returned
by \emph{pred}, or \texttt{\#t} if \emph{pred} was never called.
\end{entry}

\begin{entry}{%
  \Proto{generator-unfold}{ gen unfold arg \ldots}{procedure}{???}}

Equivalent to \texttt{(}unfold
\texttt{eof-object?\ (lambda\ (x)\ x)\ (lambda\ (x)\ (}gen\texttt{))\ (}gen\texttt{)}
arg \ldots{}\texttt{)}. The values of gen are unfolded into the
collection that unfold creates.

The signature of the unfold procedure is \texttt{(}unfold stop? mapper
successor seed args \ldots{}\texttt{)}. Note that the
\texttt{vector-unfold} and \texttt{vector-unfold-right} of
\href{http://srfi.schemers.org/srfi-43/srfi-43.html}{SRFI 43} and
\href{http://srfi.schemers.org/srfi-133/srfi-133.html}{SRFI 133} do not
have this signature and cannot be used with this procedure. To unfold
into a vector, use SRFI 1's \texttt{unfold} and then apply
\texttt{list->vector} to the result.

\begin{verbatim}
;; Iterates over string and unfolds into 
;; a list using SRFI 1 unfold
(generator-unfold 
  (make-for-each-generator string-for-each "abc") 
  unfold)
  ⇒ (#\a #\b #\c)
\end{verbatim}
\end{entry}

Editor:
\href{mailto:srfi-editors\%20at\%20srfi\%20dot\%20schemers\%20\%20\%20\%20dot\%20org}{Arthur
Gleckler}

Last modified: Wed Jun 10 08:57:14 MST 2015
