\section{Comparators}
\redno{19}  What comparator library should R7RS-large provide?

\begin{quote}
\href{http://srfi.schemers.org/srfi-128/srfi-128.html}{{​}SRFI 128} is a
comparator library whose interface is used by SRFI 113 and SRFI 125. It
was not formally balloted but was voted in based on the votes for those
libraries, as \texttt{(scheme\ comparator)}.
\end{quote}

\textbf{Post-finalization note}: Because of the extremely high cost of
conforming to the first and third conditions of \texttt{default-hash},
implementers may disregard those conditions and examine only a bounded
portion of the argument.

This SRFI provides \emph{comparators}, which bundle a type test
predicate, an equality predicate, an ordering predicate, and a hash
function (the last two are optional) into a single Scheme object. By
packaging these procedures together, they can be treated as a single
item for use in the implementation of data structures.

\setlibname{comparator}

\st{The procedures in this SRFI are in the \mbox{\texttt{(srfi\ 128)}}
  library (or \mbox{\texttt{(srfi\ :128)}} on R6RS), but the sample
  implementation currently places them in the
  \mbox{\texttt{(comparators)}} library. This means it can't be used
  alongside SRFI 114, but there's no reason for anyone to do that.}

\subsection{Definitions}

A \emph{comparator} is an object of a disjoint type. It is a bundle of
procedures that are useful for comparing two objects in a total order.
It is an error if any of the procedures have side effects. There are
four procedures in the bundle:

\begin{itemize}
\item
  The \emph{type test predicate} returns \texttt{\#t} if its argument
  has the correct type to be passed as an argument to the other three
  procedures, and \texttt{\#f} otherwise.
\item
  The \emph{equality predicate} returns \texttt{\#t} if the two objects
  are the same in the sense of the comparator, and \texttt{\#f}
  otherwise. It is the programmer's responsibility to ensure that it is
  reflexive, symmetric, transitive, and can handle any arguments that
  satisfy the type test predicate.
\item
  The \emph{ordering predicate} returns \texttt{\#t} if the first object
  precedes the second in a total order, and \texttt{\#f} otherwise. Note
  that if it is true, the equality predicate must be false. It is the
  programmer's responsibility to ensure that it is irreflexive,
  antisymmetric, transitive, and can handle any arguments that satisfy
  the type test predicate.
\item
  The \emph{hash function} takes an object and returns an exact
  non-negative integer. It is the programmer's responsibility to ensure
  that it can handle any argument that satisfies the type test
  predicate, and that it returns the same value on two objects if the
  equality predicate says they are the same (but not necessarily the
  converse).
\end{itemize}

It is also the programmer's responsibility to ensure that all four
procedures provide the same result whenever they are applied to the same
object(s) (in the sense of \texttt{eqv?}), unless the object(s) have
been mutated since the last invocation. In particular, they must not
depend in any way on memory addresses in implementations where the
garbage collector can move objects in memory.

\subsection{Limitations}

The comparator objects defined in this SRFI are not applicable to
circular structure or to NaNs, or to objects containing any of these.
Attempts to pass any such objects to any procedure defined here, or to
any procedure that is part of a comparator defined here, is an error
except as otherwise noted.

\subsection{Index}

\begin{itemize}
\item
  \protect\hyperlink{Predicates}{Predicates}:\\
  \texttt{comparator ?comparator-ordered? comparator-hashable?}
\item
  \protect\hyperlink{Constructors}{Constructors}:\\
  \texttt{make-comparator make-pair-comparator make-list-comparator}\\
  \texttt{make-vector-comparator make-eq-comparator make-eqv-comparator}\\ 
  \texttt{make-equal-comparator}
\item
  \protect\hyperlink{Hashfunctions}{Standard hash functions}:\\
  \texttt{boolean-hash char-hash char-ci-hash string-hash}\\ 
  \texttt{string-ci-hash symbol-hash number-hash}
\item
  \protect\hyperlink{Boundsandsalt}{Bounds and salt}:\\
  \texttt{hash-bound hash-salt}
\item
  \protect\hyperlink{Defaultcomparators}{Default comparators}:\\
  \texttt{make-default-comparator default-hash comparator-register-default!}
\item
  \protect\hyperlink{Accessorsandinvokers}{Accessors and invokers}:\\
  \texttt{comparator-type-test-predicate
    comparator-equality-predicate}\\ 
  \texttt{comparator-ordering-predicate comparator-hash-function}\\
  \texttt{comparator-test-type comparator-check-type comparator-hash}
\item
  \protect\hyperlink{Comparisonpredicates}{Comparison predicates}:\\
  \texttt{=?\ <?\ >?\ <=?\ >=?}
\item
  \protect\hyperlink{Syntax}{Syntax}:
  \texttt{comparator-if<=>}
\end{itemize}

\subsubsection{Predicates}

\begin{entry}{%
  \Proto{comparator?}{ obj}{procedure}{boolean}}

  Returns \texttt{\#t} if \emph{obj} is a comparator, and \texttt{\#f}
  otherwise.
\end{entry}

\begin{entry}{%
  \Proto{comparator-ordered?}{ comparator}{procedure}{boolean}}

  Returns \texttt{\#t} if \emph{comparator} has a supplied ordering
  predicate, and \texttt{\#f} otherwise.
\end{entry}

\begin{entry}{%
  \Proto{comparator-hashable?}{ comparator}{procedure}{boolean}}

  Returns \texttt{\#t} if \emph{comparator} has a supplied hash
  function, and \texttt{\#f} otherwise.
\end{entry}

\hypertarget{Constructors}{\subsubsection{Constructors}}

The following comparator constructors all supply appropriate type test
predicates, equality predicates, ordering predicates, and hash functions
based on the supplied arguments. They are allowed to cache their
results: they need not return a newly allocated object, since
comparators are pure and functional. In addition, the procedures in a
comparator are likewise pure and functional.

\begin{entry}{%
  \Proto{make-comparator}{ type-test equality ordering hash}{procedure}{comparator}}

  Returns a comparator which bundles the \emph{type-test, equality,
    ordering}, and \emph{hash} procedures provided. However, if
  \emph{ordering} or \emph{hash} is \texttt{\#f}, a procedure is
  provided that signals an error on application. The predicates
  \texttt{comparator-ordered?} and/or \texttt{comparator-hashable?},
  respectively, will return \texttt{\#f} in these cases.

  Here are calls on \texttt{make-comparator} that will return useful
  comparators for standard Scheme types:

  \begin{itemize}
  \item \texttt{(make-comparator boolean?\ boolean=?\ (lambda\ (x\
      y)\ (and\ (not\ x)\ y))\ boolean-hash)} will return a comparator
    for booleans, expressing the ordering \texttt{\#f} < \texttt{\#t}
    and the standard hash function for booleans.
  \item \texttt{(make-comparator\ real?\ =\ <\ (lambda\ (x)\ (exact\
      (abs\ x))))} will return a comparator expressing the natural
    ordering of real numbers and a plausible (but not optimal) hash
    function.
  \item \texttt{(make-comparator\ string?\ string=?\ string<?\
      string-hash)} will return a comparator expressing the
    implementation's ordering of strings and the standard hash
    function.
  \item \texttt{(make-comparator\ string?\ string-ci=?\ string-ci<?\
      string-ci-hash} will return a comparator expressing the
    implementation's case-insensitive ordering of strings and the
    standard case-insensitive hash function.
  \end{itemize}
\end{entry}

\begin{entry}{%
  \Proto{make-pair-comparator}{ car-comparator cdr-comparator}{procedure}{comparator}}

  This procedure returns comparators whose functions behave as
  follows. \TODO{why plural?}

  \begin{itemize}
  \item The type test returns \texttt{\#t} if its argument is a pair,
    if the car satisfies the type test predicate of car-comparator,
    and the cdr satisfies the type test predicate of cdr-comparator.
  \item The equality function returns \texttt{\#t} if the cars are
    equal according to car-comparator and the cdrs are equal according
    to cdr-comparator, and \texttt{\#f} otherwise.
  \item The ordering function first compares the cars of its pairs
    using the equality predicate of car-comparator. If they are equal,
    then the ordering predicate of car-comparator is applied to the
    cars and its value is returned. Otherwise, the predicate compares
    the cdrs using the equality predicate of cdr-comparator. If they
    are not equal, then the ordering predicate of cdr-comparator is
    applied to the cdrs and its value is returned.
  \item The hash function computes the hash values of the car and the
    cdr using the hash functions of car-comparator and cdr-comparator
    respectively and then hashes them together in an
    implementation-defined way.
  \end{itemize}
\end{entry}

\begin{entry}{%
  \Proto{make-list-comparator}{ element-comparator type-test empty? head tail}{procedure}{comparator}}

  This procedure returns comparators whose functions behave as
  follows: \TODO{plural?}

  \begin{itemize}
  \item The type test returns \texttt{\#t} if its argument satisfies
    type-test and the elements satisfy the type test predicate of
    element-comparator.
  \item The total order defined by the equality and ordering functions
    is as follows (known as lexicographic order):

  \begin{itemize}
    \tightlist
  \item The empty sequence, as determined by calling empty?, compares
    equal to itself.
  \item The empty sequence compares less than any non-empty sequence.
  \item Two non-empty sequences are compared by calling the head
    procedure on each. If the heads are not equal when compared using
    element-comparator, the result is the result of that comparison.
    Otherwise, the results of calling the tail procedure are compared
    recursively.
  \end{itemize}
\item The hash function computes the hash values of the elements using
  the hash function of element-comparator and then hashes them
  together in an implementation-defined way.
\end{itemize}
\end{entry}

\begin{entry}{%
  \Proto{make-vector-comparator}{ element-comparator type-test length ref}{procedure}{comparator}}

  This procedure returns comparators whose functions behave as
  follows: \TODO{plural?}

  \begin{itemize}
  \item The type test returns \texttt{\#t} if its argument satisfies
    type-test and the elements satisfy the type test predicate of
    element-comparator.
  \item The equality predicate returns \texttt{\#t} if both of the
    following tests are satisfied in order: the lengths of the vectors
    are the same in the sense of =, and the elements of the vectors
    are the same in the sense of the equality predicate of
    element-comparator.
  \item The ordering predicate returns \texttt{\#t} if the results of
    applying length to the first vector is less than the result of
    applying length to the second vector. If the lengths are equal,
    then the elements are examined pairwise using the ordering
    predicate of element-comparator.  If any pair of elements returns
    \texttt{\#t}, then that is the result of the list comparator's
    ordering predicate; otherwise the result is \texttt{\#f}
  \item The hash function computes the hash values of the elements
    using the hash function of element-comparator and then hashes them
    together in an implementation-defined way.
  \end{itemize}

  Here is an example, which returns a comparator for byte vectors:

\begin{verbatim}
(make-vector-comparator
  (make-comparator exact-integer? = < number-hash)
  bytevector?
  bytevector-length
  bytevector-u8-ref)
\end{verbatim}
\end{entry}

\begin{entry}{%
  \Proto{make-eq-comparator}{}{procedure}{comparator}
  \Proto{make-eqv-comparator}{}{procedure}{comparator}
  \Proto{make-equal-comparator}{}{procedure}{comparator}}

  These procedures return comparators whose functions behave as
  follows:

  \begin{itemize}
  \item The type test returns \texttt{\#t} in all cases.
  \item The equality functions are \texttt{eq?}, \texttt{eqv?}, and
    \texttt{equal?} respectively.
  \item The ordering function is implementation-defined, except that
    it must conform to the rules for ordering functions. It may signal
    an error instead.
  \item The hash function is \texttt{default-hash}.
  \end{itemize}

  These comparators accept circular structure (in the case of
  \texttt{equal-comparator}, provided the implementation's
  \texttt{equal?}  predicate does so) and NaNs.
\end{entry}

\hypertarget{Hashfunctions}{\subsubsection{Standard hash
functions}}

These are hash functions for some standard Scheme types, suitable for
passing to \texttt{make-comparator}. Users may write their own hash
functions with the same signature. However, if programmers wish their
hash functions to be backward compatible with the reference
implementation of
\href{http://srfi.schemers.org/srfi-69/srfi-69.html}{SRFI 69}, they are
advised to write their hash functions to accept a second argument and
ignore it.

\begin{entry}{%
  \Proto{boolean-hash}{obj}{procedure}{integer}
  \Proto{char-hash}{obj}{procedure}{integer}
  \Proto{char-ci-hash}{obj}{procedure}{integer}
  \Proto{string-hash}{obj}{procedure}{integer}
  \Proto{string-ci-hash}{obj}{procedure}{integer}
  \Proto{symbol-hash}{obj}{procedure}{integer}
  \Proto{number-hash}{obj}{procedure}{integer}}

  These are suitable hash functions for the specified types. The hash
  functions \texttt{char-ci-hash} and \texttt{string-ci-hash} treat
  their argument case-insensitively. Note that while
  \texttt{symbol-hash} may return the hashed value of applying
  \texttt{symbol->string} and then \texttt{string-hash} to the symbol,
  this is not a requirement.
\end{entry}

\hypertarget{Boundsandsalt}{\subsubsection{Bounds and
salt}}

The following macros allow the callers of hash functions to affect their
behavior without interfering with the calling signature of a hash
function, which accepts a single argument (the object to be hashed) and
returns its hash value. They are provided as macros so that they may be
implemented in different ways: as a global variable, a SRFI 39 or R7RS
parameter, or an ordinary procedure, whatever is most efficient in a
particular implementation.

\begin{entry}{%
  \Proto{hash-bound}{}{syntax}{integer}}

  Hash functions should be written so as to return a number between 0
  and the largest reasonable number of elements (such as hash buckets)
  a data structure in the implementation might have. What that value
  is depends on the implementation. This value provides the current
  bound as a positive exact integer, typically for use by user-written
  hash functions. However, they are not required to bound their
  results in this way.
\end{entry}

\begin{entry}{%
  \Proto{hash-salt}{}{syntax}{integer}}

  A salt is random data in the form of a non-negative exact integer
  used as an additional input to a hash function in order to defend
  against dictionary attacks, or (when used in hash tables) against
  denial-of-service attacks that overcrowd certain hash buckets,
  increasing the amortized O(1) lookup time to O(n). Salt can also be
  used to specify which of a family of hash functions should be used
  for purposes such as cuckoo hashing. This macro provides the current
  value of the salt, typically for use by user-written hash
  functions. However, they are not required to make use of the current
  salt.

  The initial value is implementation-dependent, but must be less than
  the value of \texttt{(hash-bound)}, and should be distinct for
  distinct runs of a program unless otherwise specified by the
  implementation.  Implementations may provide a means to specify the
  salt value to be used by a particular invocation of a hash function.
\end{entry}

\hypertarget{Defaultcomparators}{\subsubsection{Default
comparators}}

\begin{entry}{%
  \Proto{make-default-comparator}{}{procedure}{integer}}

  Returns a comparator known as a default comparator that accepts
  Scheme values and orders them in some implementation-defined way,
  subject to the following conditions:

  \begin{itemize}
  \item Given disjoint types \emph{a} and \emph{b}, one of three
    conditions must hold:

  \begin{itemize}
    \tightlist
  \item All objects of type \emph{a} compare less than all objects of
    type \emph{b}.
  \item All objects of type \emph{a} compare greater than all objects
    of type \emph{b}.
  \item All objects of both type \emph{a} and type \emph{b} compare
    equal to each other. This is not permitted for any of the Scheme
    types mentioned below.
  \end{itemize}
\item The empty list must be ordered before all pairs.
\item When comparing booleans, it must use the total order
  \texttt{\#f} < \texttt{\#t}.
\item When comparing characters, it must use \texttt{char=?} and
  \texttt{char<?}.

  Note: In R5RS, this is an implementation-dependent order that is
  typically the same as Unicode codepoint order; in R6RS and R7RS, it
  is Unicode codepoint order.
\item When comparing pairs, it must behave the same as a comparator
  returned by \texttt{make-pair-comparator} with default comparators
  as arguments.
\item When comparing symbols, it must use an implementation-dependent
  total order. One possibility is to use the order obtained by
  applying \texttt{symbol->string} to the symbols and comparing them
  using the total order implied by \texttt{string<?}.
\item When comparing bytevectors, it must behave the same as a
  comparator created by the expression
  \texttt{(make-vector-comparator\ (make-comparator\ <\ =\
    number-hash)\ bytevector?\ bytevector-length\ bytevector-u8-ref)}.
\item When comparing numbers where either number is complex, since
  non-real numbers cannot be compared with \texttt{<}, the following
  least-surprising ordering is defined: If the real parts are < or >,
  so are the numbers; otherwise, the numbers are ordered by their
  imaginary parts. This can still produce somewhat surprising results
  if one real part is exact and the other is inexact.
\item When comparing real numbers, it must use \texttt{=} and
  \texttt{<}.
\item When comparing strings, it must use \texttt{string=?} and
  \texttt{string<?}.

  Note: In R5RS, this is lexicographic order on the
  implementation-dependent order defined by \texttt{char<?}; in R6RS
  it is lexicographic order on Unicode codepoint order; in R7RS it is
  an implementation-defined order.
\item When comparing vectors, it must behave the same as a comparator
  returned by \texttt{(make-vector-comparator\
    (make-default-comparator)\ vector?\ vector-length\ vector-ref)}.
\item When comparing members of types registered with
  \texttt{comparator-register-default!}, it must behave in the same
  way as the comparator registered using that function.
\end{itemize}

Default comparators use \texttt{default-hash} as their hash function.
\end{entry}

\begin{entry}{%
  \Proto{default-hash}{ obj}{procedure}{value}}

  This is the hash function used by default comparators, which accepts
  a Scheme value and hashes it in some implementation-defined way,
  subject to the following conditions:

\begin{itemize}
  \tightlist
\item When applied to a pair, it must return the result of hashing
  together the values returned by \texttt{default-hash} when applied
  to the car and the cdr.
\item When applied to a boolean, character, string, symbol, or number,
  it must return the same result as \texttt{boolean-hash},
  \texttt{char-hash}, \texttt{string-hash}, \texttt{symbol-hash}, or
  \texttt{number-hash} respectively.
\item When applied to a list or vector, it must return the result of
  hashing together the values returned by \texttt{default-hash} when
  applied to each of the elements.
\end{itemize}
\end{entry}

\begin{entry}{%
  \texttt{comparator-register-default!}{ comparator}{procedure}{unspecified}}

  Registers comparator for use by default comparators, such that if
  the objects being compared both satisfy the type test predicate of
  comparator, it will be employed by default comparators to compare
  them.  Returns an unspecified value. It is an error if any value
  satisfies both the type test predicate of comparator and any of the
  following type test predicates: \texttt{boolean?}, \texttt{char?},
  \texttt{null?}, \texttt{pair?}, \texttt{symbol?},
  \texttt{bytevector?}, \texttt{number?}, \texttt{string?},
  \texttt{vector?}, or the type test predicate of a comparator that
  has already been registered.

  This procedure is intended only to extend default comparators into
  territory that would otherwise be undefined, not to override their
  existing behavior. In general, the ordering of calls to
  \texttt{comparator-register-default!} should be irrelevant. However,
  implementations that support inheritance of record types may wish to
  ensure that default comparators always check subtypes before
  supertypes.
\end{entry}

\hypertarget{Accessorsandinvokers}{\subsubsection{Accessors and
Invokers}}

\begin{entry}{%
  \Proto{comparator-type-test-predicate}{ comparator}{procedure}{procedure}
  \Proto{comparator-equality-predicate}{ comparator}{procedure}{procedure}
  \Proto{comparator-ordering-predicate}{ comparator}{procedure}{procedure}
  \Proto{comparator-hash-function}{ comparator}{procedure}{procedure}}

  Return the four procedures of \emph{comparator}.
\end{entry}

\begin{entry}{%
  \Proto{comparator-test-type}{ comparator obj}{procedure}{boolean}}

  Invokes the type test predicate of \emph{comparator} on \emph{obj}
  and returns what it returns. More convenient than
  \texttt{comparator_type_test_predicate}, but less efficient when the
  predicate is called repeatedly.
\end{entry}

\begin{entry}{%
  \Proto{comparator-check-type}{ comparator obj}{procedure}{boolean}}

  Invokes the type test predicate of \emph{comparator} on \emph{obj}
  and returns true if it returns true, but signals an error
  otherwise. More convenient than
  \texttt{comparator_type_test_predicate}, but less efficient when the
  predicate is called repeatedly.

  \TODO{the return type here is misleading; if it's a boolean, then
    it's true. We don't have a standard way, in a prototype of saying
    “may signal an error”. I'm not sure what to do here; my instinct
    is just to leave it as is.}
\end{entry}

\begin{entry}{%
  \Proto{comparator-hash}{ comparator obj}{procedure}{integer}}

  Invokes the hash function of \emph{comparator} on obj and returns
  what it returns. More convenient than
  \texttt{comparator_hash_function}, but less efficient when the
  function is called repeatedly.

  Note: No invokers are required for the equality and ordering
  predicates, because \texttt{=?} and \texttt{<?} serve this function.
\end{entry}

\hypertarget{Comparisonpredicates}{\subsubsection{Comparison
predicates}}


\begin{entry}{%
  \Proto{=?}{ comparator object$_1$ object$_2$ object$_3$ \ldots{}}{procedure}{boolean}
  \Proto{<?}{ comparator object$_1$ object$_2$ object$_3$ \ldots{}}{procedure}{boolean}
  \Proto{>?}{ comparator object$_1$ object$_2$ object$_3$ \ldots{}}{procedure}{boolean}
  \Proto{<=?}{ comparator object$_1$ object$_2$ object$_3$ \ldots{}}{procedure}{boolean}
  \Proto{>=?}{ comparator object$_1$ object$_2$ object$_3$ \ldots{}}{procedure}{boolean}}

  These procedures are analogous to the number, character, and string
  comparison predicates of Scheme. They allow the convenient use of
  comparators to handle variable data types.

  These procedures apply the equality and ordering predicates of
  \emph{comparator} to the \emph{objects} as follows. If the specified
  relation returns \texttt{\#t} for all \emph{object\textsubscript{i}}
  and \emph{object\textsubscript{j}} where \emph{n} is the number of
  objects and 1 <= \emph{i < j <= n}, then the procedures return
  \texttt{\#t}, but otherwise \texttt{\#f}. Because the relations are
  transitive, it suffices to compare each object with its
  successor. The order in which the values are compared is
  unspecified.
\end{entry}

\hypertarget{Syntax}{\subsubsection{Syntax}}

\begin{entry}{%
  \Proto{comparator-if<=>}{ <comparator> <object$_1$> <object$_2$>
    <less-than> <equal-to> <greater-than>}{syntax}{value}
  \Proto{comparator-if<=>}{ <object$_1$> <object$_2$> <less-than> <equal-to> <greater-than>}{syntax}{value}}

  It is an error unless <comparator> evaluates to a comparator and
  <object$_1$> and <object$_2$> evaluate to objects that the
  comparator can handle. If the ordering predicate returns true when
  applied to the values of <object$_1$> and <object$_2$> in that
  order, then <less-than> is evaluated and its value returned.  If the
  equality predicate returns true when applied in the same way, then
  <equal-to> is evaluated and its value returned. If neither returns
  true, <greater-than> is evaluated and its value returned.

  If <comparator> is omitted, a default comparator is used.

  \TODO{remove angle brackets, use std param notation}

\end{entry}

Editor: \href{mailto:srfi-editors+at+srfi+dot+schemers+dot+org}{Arthur
A. Gleckler}
