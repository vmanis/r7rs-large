\section{Deques}
\redno{10} What immutable deque library should R7RS-large provide?

\begin{quote}
SRFI 134 is a library for immutable double-ended queues, with O(1)
addition and removal time at both ends.
\end{quote}

This SRFI defines immutable deques. A deque is a double-ended queue, a
sequence which allows elements to be added or removed efficiently from
either end. A structure is immutable when all its operations leave the
structure unchanged. Note that none of the \TODO{names of the}
procedures specified here ends with an exclamation point.

\setlibname{ideque}

\subsection{Procedure index}\label{procedure-index}

\protect\hyperlink{Constructors}{Constructors}:
\texttt{ideque,\ ideque-tabulate,\ ideque-unfold,\ ideque-unfold-right}

\protect\hyperlink{Predicates}{Predicates}:
\texttt{ideque?,\ ideque-empty?,\ ideque=,\ ideque-any,\ ideque-every}

\protect\hyperlink{Queueoperations}{Queue operations}:
\texttt{ideque-front,\ ideque-back,\ ideque-remove-front,\ ideque-remove-back,\ ideque-add-front,\ ideque-add-back}

\protect\hyperlink{Otheraccessors}{Other accessors}:
\texttt{ideque-ref,\ ideque-take,\ ideque-take-right,\ ideque-drop,\ ideque-drop-right,\ ideque-split-at}

\protect\hyperlink{Thewholedeque}{The whole ideque}:
\texttt{ideque-length,\ ideque-append,\ ideque-reverse,\ ideque-count,\ ideque-zip}

\protect\hyperlink{Mapping}{Mapping}:
\texttt{ideque-map,\ ideque-filter-map,\ ideque-for-each,\ ideque-for-each-right,\ ideque-fold,\ ideque-fold-right,\ ideque-append-map}

\protect\hyperlink{Filtering}{Filtering}:
\texttt{ideque-filter,\ ideque-remove,\ ideque-partition}

\protect\hyperlink{Searching}{Searching}:
\texttt{ideque-find,\ ideque-find-right,\ ideque-take-while,\ ideque-take-while-right,\ ideque-drop-while,\ ideque-drop-while-right,\ ideque-span,\ ideque-break}

\protect\hyperlink{Conversion}{Conversion}:
\texttt{list-\textgreater{}ideque,\ ideque-\textgreater{}list,\ generator-\textgreater{}ideque,\ ideque-\textgreater{}generator}


We specify required time efficiency upper bounds using big-O notation.
We note when, in some cases, there is ``slack'' between the required
bound and the theoretically optimal bound for an operation.
Implementations may use data structures with amortized time bounds, but
should document which bounds hold in only an amortized sense.

Deques are disjoint from all other Scheme types.

\subsubsection{Constructors}

\begin{entry}{%
  \Proto{ideque}{ element \ldots}{procedure}{ideque???}}

  Returns an ideque containing the \emph{elements}. The first element
  (if any) will be at the front of the ideque and the last element (if
  any) will be at the back. Takes $O(n)$ time, where $n$ is the
  number of elements.
\end{entry}

\begin{entry}{%
  \Proto{ideque-tabulate}{ n proc}{procedure}{ideque???}}

  Invokes the predicate \emph{proc} on every exact integer from 0
  (inclusive) to \emph{n} (exclusive). Returns an ideque containing
  the results in order of generation. Takes $O(n)$ time.
\end{entry}

\begin{entry}{%
  \Proto{ideque-unfold}{ stop? mapper successor seed}{procedure}{ideque???}}

  Invokes the predicate \emph{stop?} on \emph{seed}. If it returns
  false, generate the next result by applying \emph{mapper} to
  \emph{seed}, generate the next seed by applying \emph{successor} to
  \emph{seed}, and repeat this algorithm with the new seed. If
  \emph{stop?} returns true, return an ideque containing the results
  in order of accumulation. Takes $O(n)$ time.
\end{entry}

\begin{entry}{%
  \Proto{ideque-unfold-right}{ stop? mapper successor seed}{procedure}{ideque???}}

  Invokes the predicate \emph{stop?} on \emph{seed}. If it returns
  false, generate the next result by applying \emph{mapper} to
  \emph{seed}, generate the next seed by applying \emph{successor} to
  \emph{seed}, and repeat the algorithm with the new seed. If
  \emph{stop?} returns true, return an ideque containing the results
  in reverse order of accumulation. Takes $O(n)$ time.
\end{entry}

\subsubsection{Predicates}

\begin{entry}{%
  \Proto{ideque?}{ x}{procedure}{boolean???}}

  Returns \texttt{\#t} if \emph{x} is an ideque, and \texttt{\#f}
  otherwise. Takes $O(1)$ time.
\end{entry}

\begin{entry}{%
  \Proto{ideque-empty?}{ ideque}{procedure}{boolean???}}

  Returns \texttt{\#t} if \emph{ideque} contains zero elements, and
  \texttt{\#f} otherwise. Takes $O(1)$ time.
\end{entry}


\begin{entry}{%
  \Proto{ideque=}{ elt= ideque \ldots}{procedure}{boolean???}}

  Determines ideque equality, given an element-equality
  procedure. Ideque A equals ideque B if they are of the same length,
  and their corresponding elements are equal, as determined by
  \emph{elt=}. If the element-comparison procedure's first argument is
  from \emph{ideque\textsubscript{i}}, then its second argument is
  from \emph{ideque\textsubscript{i+1}}, i.e. it is always called as
  \texttt{(}\emph{elt= a b}\texttt{)} for \emph{a} an element of
  ideque A, and \emph{b} an element of ideque B.

  In the n-ary case, every \emph{ideque\textsubscript{i}} is compared
  to \emph{ideque\textsubscript{i+1}} (as opposed, for example, to
  comparing \emph{ideque$_1$} to every
  \emph{ideque\textsubscript{i}}, for i \textgreater{} 1). If there
  are zero or one ideque arguments, \texttt{ideque=} simply returns
  true. The name does not end in a question mark for compatibility
  with the SRFI-1 procedure \texttt{list=}.

  Note that the dynamic order in which the \emph{elt=} procedure is
  applied to pairs of elements is not specified. For example, if
  \texttt{ideque=} is applied to three ideques, A, B, and C, it may
  first completely compare A to B, then compare B to C, or it may
  compare the first elements of A and B, then the first elements of B
  and C, then the second elements of A and B, and so forth.

  The equality procedure must be consistent with \texttt{eq?}. Note
  that this implies that two ideques which are \texttt{eq?} are always
  \texttt{ideque=}, as well; implementations may exploit this fact to
  ``short-cut'' the element-by-element comparisons.
\end{entry}

\begin{entry}{%
  \Proto{ideque-any}{ pred ideque}{procedure}{value \textrm{or} boolean???}
  \Proto{ideque-every}{ pred ideque}{procedure}{ideque???}}

  Invokes \emph{pred} on the elements of the \emph{ideque} in order
  until one call returns a true/false value, which is then
  returned. If there are no elements, returns
  \texttt{\#f}/\texttt{\#t}. Takes $O(n)$ time.
\end{entry}

\subsubsection{Queue operations}

\begin{entry}{%
  \Proto{ideque-front}{ ideque}{procedure}{value???}
  \Proto{ideque-back}{ ideque}{procedure}{value???}}

  Returns the front/back element of \emph{ideque}. It is an error for
  \emph{ideque} to be empty. Takes $O(1)$ time.
\end{entry}

\begin{entry}{%
  \Proto{ideque-remove-front}{ ideque}{procedure}{ideque???}
  \Proto{ideque-remove-back}{ ideque}{procedure}{ideque???}}

  Returns an ideque with the front/back element of \emph{ideque}
  removed.  It is an error for \emph{ideque} to be empty. Takes $O(1)$
  time.
\end{entry}

\begin{entry}{%
  \Proto{ideque-add-front}{ ideque obj}{procedure}{ideque???}
  \Proto{ideque-add-back}{ ideque obj}{procedure}{ideque???}}

  Returns an ideque with \emph{obj} pushed to the front/back of
  \emph{ideque}. Takes $O(1)$ time.
\end{entry}

\subsubsection{Other accessors}

\begin{entry}{%
  \Proto{ideque-ref}{ ideque n}{procedure}{value???}}

  Returns the \emph{nth} element of \emph{ideque}. It is an error
  unless \emph{n} is less than the length of \emph{ideque}. Takes $O(n)$
  time.
\end{entry}

\begin{entry}{%
  \Proto{ideque-take}{ ideque n}{procedure}{ideque???}
  \Proto{ideque-take-right}{ ideque n}{procedure}{ideque???}}

  Returns an ideque containing the first/last \emph{n} elements of
  \emph{ideque}. It is an error if \emph{n} is greater than the length
  of \emph{ideque}. Takes $O(n)$ time.
\end{entry}

\begin{entry}{%
  \Proto{ideque-drop}{ ideque n}{procedure}{ideque???}
  \Proto{ideque-drop-right}{ ideque n}{procedure}{ideque???}}

  Returns an ideque containing all but the first/last \emph{n}
  elements of \emph{ideque}. It is an error if \emph{n} is greater
  than the length of \emph{ideque}. Takes $O(n)$ time.
\end{entry}

\begin{entry}{%
  \Proto{ideque-split-at}{ ideque n}{procedure}{ideque??? ideque???}}

Returns two values, the results of \texttt{(ideque-take}~\emph{ideque
  n}\texttt{)} and \texttt{(ideque-drop\ }\emph{ideque n}\texttt{)}
respectively, but may be more efficient. Takes $O(n)$ time.
\end{entry}

\subsubsection{The whole ideque}

\begin{entry}{%
  \Proto{ideque-length}{ ideque}{procedure}{integer???}}

  Returns the length of \emph{ideque} as an exact integer. May take
  $O(n)$ time, though $O(1)$ is optimal. \TODO{“may take $O(1)$ in the
    best case"??? I thought we were giving average cases.} 
\end{entry}

\begin{entry}{%
  \Proto{ideque-append}{ ideque \ldots}{procedure}{ideque???}}

  Returns an ideque with the contents of the \emph{ideque} followed by
  the others, or an empty ideque if there are none. Takes $O(kn)$ time,
  where $k$ is the number of ideques and $n$ is the number of elements
  involved, though \st{$O(k \log n)$ is possible.} some
    implementations may only require $O(k \log n)$. \TODO{I assume that's
    “lg”, not that it really matters.}
\end{entry}

\begin{entry}{%
  \Proto{ideque-reverse}{ ideque}{procedure}{ideque???}}

  Returns an ideque containing the elements of \emph{ideque} in
  reverse order. Takes $O(1)$ time.
\end{entry}

\begin{entry}{%
  \Proto{ideque-count}{ pred ideque}{procedure}{integer???}}

  \emph{Pred} is a procedure taking a single value and returning a
  single value. It is applied element-wise to the elements of ideque,
  and a count is tallied of the number of elements that produce a true
  value. This count is returned. Takes $O(n)$ time. The dynamic order of
  calls to pred is unspecified.
\end{entry}

\begin{entry}{%
  \Proto{ideque-zip}{ ideque$_1$ ideque$_2$ \ldots}{procedure}{ideque???}}

  Returns an ideque of lists (not ideques) each of which contains the
  corresponding elements of ideques in the order specified. Terminates
  when all the elements of any of the ideques have been
  processed. Takes $O(kn)$ time, where $k$ is the number of ideques
  and $n$ is the number of elements in the shortest ideque.
\end{entry}

\subsubsection{Mapping}

\begin{entry}{%
  \Proto{ideque-map}{ proc ideque}{procedure}{ideque???}}

  Applies \emph{proc} to the elements of \emph{ideque} and returns an
  ideque containing the results in order. The dynamic order of calls
  to \emph{proc} is unspecified. Takes $O(n)$ time.
\end{entry}

\begin{entry}{%
  \Proto{ideque-filter-map}{ proc ideque}{procedure}{ideque???}}

  Applies \emph{proc} to the elements of \emph{ideque} and returns an
  ideque containing the true (i.e. non-\texttt{\#f}) results in
  order. The dynamic order of calls to \emph{proc} is
  unspecified. Takes $O(n)$ time.
\end{entry}

\begin{entry}{%{%
  \Proto{ideque-for-each}{ proc ideque}{procedure}{unspecified???}
  \Proto{ideque-for-each-right}{ proc ideque}{procedure}{unspecified???}}

  Applies \emph{proc} to the elements of \emph{ideque} in
  forward/reverse order and returns an unspecified result. Takes $O(n)$
  time.
\end{entry}

\begin{entry}{%
  \Proto{ideque-fold}{ proc nil ideque}{procedure}{value???}
  \Proto{ideque-fold-right}{ proc nil ideque}{procedure}{value???}}

  Invokes \emph{proc} on the elements of \emph{ideque} in
  forward/reverse order, passing the result of the previous invocation
  as a second argument. For the first invocation, \emph{nil} is used
  as the second argument. Returns the result of the last invocation,
  or \emph{nil} if there was no invocation. Takes $O(n)$ time.
\end{entry}

\begin{entry}{%
  \Proto{ideque-append-map}{ proc ideque}{procedure}{ideque???}}

  Applies \emph{proc} to the elements of \emph{ideque}. It is an error
  if the result is not a list. Returns an ideque containing the
  elements of the lists in order. Takes $O(n)$ time, where n is the
  number of elements in all the lists returned.
\end{entry}

\subsubsection{Filtering}

\begin{entry}{%
  \Proto{ideque-filter}{ pred ideque}{procedure}{ideque???}
  \Proto{ideque-remove}{ pred ideque}{procedure}{ideque???}}

  Returns an ideque containing the elements of \emph{ideque} that
  do/do not satisfy \emph{pred}. Takes $O(n)$ time.

  \TODO{In the same order?}
\end{entry}

\begin{entry}{%
  \Proto{ideque-partition}{ proc ideque}{procedure}{ideque??? ideque???}}

Returns two values, the results of \texttt{(ideque-filter}~\emph{pred
  ideque}\texttt{)} and \texttt{(ideque-remove\ }\emph{pred
  ideque}\texttt{)} respectively, but may be more efficient. Takes
$O(n)$ time.
\end{entry}

\subsubsection{Searching}

\begin{entry}{%
  \Proto{ideque-find}{ pred ideque failure}{procedure}{value???}
  \Rproto{ideque-find}{ pred ideque}{procedure}{value???}
  \Proto{ideque-find-right}{ pred ideque failure}{procedure}{value???}
  \Rproto{ideque-find-right}{ pred ideque}{procedure}{value???}}

  Returns the first/last element of \emph{ideque} that satisfies
  \emph{pred}. If there is no such element, returns the result of
  invoking the thunk \emph{failure}; the default thunk is
  \texttt{(lambda\ ()\ \#f)}. Takes $O(n)$ time.
\end{entry}

\begin{entry}{%
  \Proto{ideque-take-while}{ pred ideque}{procedure}{ideque???}
  \Proto{ideque-take-while-right}{ pred ideque}{procedure}{ideque???}}

  Returns an ideque containing the longest initial/final prefix of
  elements in \emph{ideque} all of which satisfy \emph{pred}. Takes
  $O(n)$ time.
\end{entry}

\begin{entry}{%
  \Proto{ideque-drop-while}{ pred ideque}{procedure}{ideque???}
  \Proto{ideque-drop-while-right}{ pred ideque}{procedure}{ideque???}}

  Returns an ideque which omits the longest initial/final prefix of
  elements in \emph{ideque} all of which satisfy \emph{pred}, but
  includes all other elements of \emph{ideque}. Takes $O(n)$ time.
\end{entry}

\begin{entry}{%
  \Proto{ideque-span}{ pred ideque}{procedure}{ideque??? ideque???}
  \Proto{ideque-break}{ pred ideque}{procedure}{ideque??? ideque???}}

  Returns two values, the initial prefix of the elements of
  \emph{ideque} which do/do not satisfy \emph{pred}, and the remaining
  elements. Takes $O(n)$ time.

  \TODO{Explain why the 2 names.}
\end{entry}

\subsubsection{Conversion}

\begin{entry}{%
  \Proto{list->ideque}{ list}{procedure}{ideque???}
  \Proto{ideque->list}{ ideque}{procedure}{list???}}

  Conversion between ideque and list structures. FIFO order is
  preserved, so the front of a list corresponds to the front of an
  ideque. Each operation takes $O(n)$ time.
\end{entry}

\begin{entry}{%
  \Proto{generator->ideque}{ generator}{procedure}{ideque???}
  \Proto{ideque->generator}{ ideque}{procedure}{generator???}}

  Conversion between
  \href{http://srfi.schemers.org/srfi-121/srfi-121.html}{SRFI 121}
  generators and ideques. Each operation takes $O(n)$ time. A generator
  is a procedure that is called repeatedly with no arguments to
  generate consecutive values, and returns an end-of-file object when
  it has no more values to return.
\end{entry}

Editor: \href{mailto:srfi-editors+at+srfi+dot+schemers+dot+org}{Arthur
A. Gleckler}
