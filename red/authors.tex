\textbf{Authors}

\TODO{Rework this whole thing. I'm tentatively thinking that the title
  page should list the editors, followed by “Major Contributions by”,
  and the SRFI authors, in alphabetical order. Acknowledgements should
  appear in alphabetical order, in the document, with each person
  listed once, no matter how many SRFIs they are credited for. Not
  totally sure what to do about the copyright notices, since many of
  the original notices say “All Rights Reserved”, and the standard
  SRFI notice contradicts that, and only talks about “the
  Software”. Suggest writing to the SRFI authors and asking them if
  it's permissible to put the derived documentation under the R7RS
  copyright (R7RS, p3), which is in my non-legal opinion equivalent to
  public domain.}

\begin{itemize}
\item SRFI 1: List library. Olin Shivers.
  \verb|(scheme list)|

Copyright (C) Olin Shivers (1998, 1999). All Rights
Reserved.

The design of this library benefited greatly from the feedback provided
during the SRFI discussion phase. Among those contributing thoughtful
commentary and suggestions, both on the mailing list and by private
discussion, were Mike Ashley, Darius Bacon, Alan Bawden, Phil Bewig, Jim
Blandy, Dan Bornstein, Per Bothner, Anthony Carrico, Doug Currie, Kent
Dybvig, Sergei Egorov, Doug Evans, Marc Feeley, Matthias Felleisen, Will
Fitzgerald, Matthew Flatt, Dan Friedman, Lars Thomas Hansen, Brian
Harvey, Erik Hilsdale, Wolfgang Hukriede, Richard Kelsey, Donovan
Kolbly, Shriram Krishnamurthi, Dave Mason, Jussi Piitulainen, David
Pokorny, Duncan Smith, Mike Sperber, Maciej Stachowiak, Harvey J. Stein,
John David Stone, and Joerg F. Wittenberger. I am grateful to them for
their assistance.

I am also grateful the authors, implementors and documentors of all the
systems mentioned in the rationale. Aubrey Jaffer and Kent Pitman should
be noted for their work in producing Web-accessible versions of the R5RS
and \protect\hyperlink{CommonLisp}{Common Lisp} spec, which was a
tremendous aid.

This is not to imply that these individuals necessarily endorse the
final results, of course.


\item SRFI 133: Vector Library. John Cowan (based on SRFI 43 by Taylor
  Campbell). \verb|(scheme vector)|

Copyright (C) Taylor Campbell (2003). All rights reserved.
(And John Cowan?)

These acknowledgements are copied from SRFI 43.

Thanks to Olin Shivers for his wonderfully complete
\protect\hyperlink{SRFI-1}{list} and \protect\hyperlink{SRFI-13}{string}
packages; to all the members of the
\href{http://scheme-irc.webhop.org/}{\texttt{\#scheme} IRC channel} on
\href{http://www.freenode.net/}{Freenode} who nitpicked a great deal,
but also helped quite a lot in general, and helped test the reference
implementation in various Scheme systems; to Michael Burschik for his
numerous comments; to Sergei Egorov for helping to narrow down the
procedures; to Mike Sperber for putting up with an \emph{extremely}
overdue draft; to Felix Winkelmann for continually bugging me about
finishing up the SRFI so that it would be only overdue and not
withdrawn; and to everyone else who gave questions, comments, thoughts,
or merely attention to the SRFI.

\item SRFI 132: Sort Libraries. John Cowan (based on SRFI 32 by Olin Shivers)

This document is copyright (C) Olin Shivers (1998, 1999). All Rights
Reserved. (Copyright John Cowan as well?)

Olin thanked the authors of the open source consulted when designing
this library, particularly Richard O'Keefe, Donovan Kolbly and the MIT
Scheme Team. John thanks Will Clinger for his detailed comments, and
both Will Clinger and Alex Shinn for their implementation efforts.

\item SRFI 113: Sets and Bags. John Cowan

Copyright (C) John Cowan 2013. All Rights Reserved.

\item SRFI 14: Character-set Library. Olin Shivers.

Certain portions of this document -- the specific, marked segments of
text describing the \protect\hyperlink{R5RS}{R5RS} procedures -- were
adapted with permission from the R5RS report.

All other text is copyright (C) Olin Shivers (1998, 1999, 2000). All
Rights Reserved.


The design of this library benefited greatly from the feedback provided
during the SRFI discussion phase. Among those contributing thoughtful
commentary and suggestions, both on the mailing list and by private
discussion, were Paolo Amoroso, Lars Arvestad, Alan Bawden, Jim Bender,
Dan Bornstein, Per Bothner, Will Clinger, Brian Denheyer, Kent Dybvig,
Sergei Egorov, Marc Feeley, Matthias Felleisen, Will Fitzgerald, Matthew
Flatt, Arthur A. Gleckler, Ben Goetter, Sven Hartrumpf, Erik Hilsdale,
Shiro Kawai, Richard Kelsey, Oleg Kiselyov, Bengt Kleberg, Donovan
Kolbly, Bruce Korb, Shriram Krishnamurthi, Bruce Lewis, Tom Lord, Brad
Lucier, Dave Mason, David Rush, Klaus Schilling, Jonathan Sobel, Mike
Sperber, Mikael Staldal, Vladimir Tsyshevsky, Donald Welsh, and Mike
Wilson. I am grateful to them for their assistance.

I am also grateful the authors, implementors and documentors of all the
systems mentioned in the introduction. Aubrey Jaffer should be noted for
his work in producing Web-accessible versions of the R5RS spec, which
was a tremendous aid.

This is not to imply that these individuals necessarily endorse the
final results, of course.

During this document's long development period, great patience was
exhibited by Mike Sperber, who is the editor for the SRFI, and by
Hillary Sullivan, who is not.

\item SRFI 125: Intermediate hash tables. John Cowan, Will Clinger.

Copyright (C) John Cowan (2015).

Some of the language of this SRFI is copied from SRFI 69 with thanks to
its author, Panu Kalliokoski. However, he is not responsible for what I
have done with it. Thanks to Will Clinger for providing the sample
implementation, and to Taylan Ulrich Bayırlı/Kammer for his spirited
review.

I also acknowledge the members of the SRFI 125, 126, and 128 mailing
lists, especially Takashi Kato, Alex Shinn, Shiro Kawai, and Per
Bothner.

\item SRFI 116: Immutable List Library. John Cowan.

Copyright (C) John Cowan 2014. All Rights Reserved.

Without the work of Olin Shivers on
\href{http://srfi.schemers.org/srfi-1/srfi-1.html}{SRFI 1}, this SRFI
would not exist. Everyone acknowledged there is transitively
acknowledged here. This is not to imply that either Olin or anyone else
necessarily endorses the final results, of course.

\item SRFI 101: Purely Functional Random-Access Pairs and Lists. David Van Horn.

Copyright (C) David Van Horn 2009. All Rights Reserved.

I am grateful to the members of the
\href{http://www.ccs.neu.edu/research/prl/}{Northeastern University
Programming Research Laboratory} and \href{http://plt-scheme.org/}{PLT}
(and their intersection) for discussions during the pre-draft
development of this SRFI and the library that preceded it. We are all
indebted to Okasaki for his elegant solution to this problem. Much of
the specification is adapted from text intended to belong to the Scheme
community; I thank the editors and authors of the RnRS series
collectively for their efforts. I am grateful to Donovan Kolbly for
serving as SRFI editor and to Taylor R Campbell, Robert Bruce Findler,
Aubrey Jaffer, Shiro Kawai, and Alexey Radul for discussion during the
draft period. I thank William D Clinger, Robert Bruce Findler, and
Abdulaziz Ghuloum for help writing the implementations of \texttt{quote}
specific to Larceny, Ikarus, and PLT, respectively. Support was provided
by the National Science Foundation under Grant \#0937060 to the
Computing Research Association for the CIFellow Project.

\item SRFI 134: Immutable Deques. Kevin Wortman, John Cowan.

Copyright (C) John Cowan, Kevin Wortman (2015). All Rights Reserved.

\item SRFI 135: Immutable Texts. William D Clinger.

Copyright (C) William D Clinger (2016). All Rights Reserved.

For three decades, I have been hoping the Scheme standards would either
make strings immutable or add a new data type of immutable texts; with
Unicode, that hope became more urgent. During that time, I have
discussed this with far more people than I can now remember. Most of
those I do remember are among those acknowledged below by John Cowan or
Olin Shivers, but I am pleased to add Lars T Hansen, Chris Hanson, Felix
Klock, and Jonathan Rees to the list of those whose ideas (and
counter-arguments!) have contributed to this SRFI.

John Cowan, the author of SRFI 130, deserves special thanks for blessing
my desire to use SRFI 130 as the starting point for this SRFI, for
designing the spans API whose implementations tested the key ideas of
this SRFI's sample implementations, for chairing Working Group 2, and
for a lot more I won't mention here.

To acknowledge all those who contributed to SRFI 130 and to its
predecessor SRFI 13, written by Olin Shivers, I hereby reproduce John
Cowan's acknowledgements from SRFI 130:

\begin{quote}
Thanks to the members of the SRFI 130 mailing list who made this SRFI
what it now is, including Per Bothner, Arthur Gleckler, Shiro Kawai, Jim
Rees, and especially Alex Shinn, whose idea it was to make cursors and
indexes disjoint, and who provided the foof implementation. The
following acknowledgements by Olin Shivers are taken from SRFI 13:

\begin{quote}
The design of this library benefited greatly from the feedback provided
during the SRFI discussion phase. Among those contributing thoughtful
commentary and suggestions, both on the mailing list and by private
discussion, were Paolo Amoroso, Lars Arvestad, Alan Bawden, Jim Bender,
Dan Bornstein, Per Bothner, Will Clinger, Brian Denheyer, Mikael
Djurfeldt, Kent Dybvig, Sergei Egorov, Marc Feeley, Matthias Felleisen,
Will Fitzgerald, Matthew Flatt, Arthur A. Gleckler, Ben Goetter, Sven
Hartrumpf, Erik Hilsdale, Richard Kelsey, Oleg Kiselyov, Bengt Kleberg,
Donovan Kolbly, Bruce Korb, Shriram Krishnamurthi, Bruce Lewis, Tom
Lord, Brad Lucier, Dave Mason, David Rush, Klaus Schilling, Jonathan
Sobel, Mike Sperber, Mikael Staldal, Vladimir Tsyshevsky, Donald Welsh,
and Mike Wilson. I am grateful to them for their assistance.

I am also grateful to the authors, implementors and documentors of all
the systems mentioned in the introduction. Aubrey Jaffer and Kent Pitman
should be noted for their work in producing Web-accessible versions of
the \protect\hyperlink{R5RS}{R5RS} and Common Lisp spec, which was a
tremendous aid.

This is not to imply that these individuals necessarily endorse the
final results, of course.

During this document's long development period, great patience was
exhibited by Mike Sperber, who is the editor for the SRFI, and by
Hillary Sullivan, who is not.
\end{quote}
\end{quote}

As Olin said, we should not assume any of those individuals endorse this
SRFI.

\item SRFI 121: Generators. Shiro Kawai, John Cowan, Thomas Gilray.

Copyright (C) Shiro Kawai, John Cowan, Thomas Gilray (2015). All Rights
Reserved.

These procedures are drawn from the Gauche core and the Gauche module
\texttt{gauche.generator}, with some renaming to make them more
systematic, and with a few additions from the Python library
\href{https://docs.python.org/3/library/itertools.html}{\texttt{itertools}}.
Consequently, Shiro Kawai, the author of Gauche and its specifications,
is listed as first author of this SRFI. John Cowan served as editor and
shepherd. Thomas Gilray provided the sample implementation and a
valuable critique of the SRFI. Special acknowledgements to Kragen Javier
Sitaker for his extensive review.

\item SRFI 127: Lazy Sequences. John Cowan.

Copyright (C) John Cowan (2015). All Rights Reserved.

Without the work of Olin Shivers on
\href{http://srfi.schemers.org/srfi-1/srfi-1.html}{SRFI 1}, this SRFI
would not exist. Everyone acknowledged there is transitively
acknowledged here. This is not to imply that either Olin or anyone else
necessarily endorses the final results, of course.

Special thanks to Shiro Kawai, whose Gauche implementation of lazy
sequences inspired this one, and to Kragen Javier Sitaker, who did a
thorough review.

\item SRFI 41: Streams. Philip L. Bewig.

Copyright (C) Philip L. Bewig (2007). All Rights Reserved.

Jos Koot sharpened my thinking during many e-mail discussions, suggested
several discussion points in the text, and contributed the final version
of stream-match. Michael Sperber and Abdulaziz Ghuloum gave advice on
R6RS.

\item SRFI 111: Boxes. John Cowan.

Copyright (C) John Cowan 2013. All Rights Reserved.

\item SRFI 117: Queues based on lists. John Cowan.

Copyright © John Cowan, 2014. All Rights Reserved.

\item SRFI 124: Ephemerons.John Cowan.

Copyright (C) John Cowan (2015).

\item SRFI 128: Comparators (reduced). John Cowan.

Copyright (C) John Cowan (2015). All Rights Reserved.

This SRFI is a simplified and enhanced rewrite of
\href{http://srfi.schemers.org/srfi-114/srfi-114.html}{SRFI 114}, and
shares some of its design rationale and all of its acknowledgements. The
largest change is the replacement of the comparison procedure with the
ordering procedure. This allowed most of the special-purpose comparators
to be removed. In addition, many of the more specialized procedures, as
well as all but one of the syntax forms, have been removed as
unnecessary.

Special thanks to Taylan Ulrich Bayırlı/Kammer, whose insistence that
SRFI 114 was unacceptable inspired this redesign. Jörg Wittenberger
added Chicken-specific type declarations, which I have moved to
\texttt{comparators.scm}, as it is a Chicken-specific library. He also
provided Chicken-specific metadata and setup commands. Comments from
Shiro Kawai, Alex Shinn, and Kevin Wortman guided me to the current
design for bounds and salt.



\end{itemize}

Others: WG2 members.

Copyright notice

\TODO{This is really a report, rather than software. I think the
  statement below should be replaced by the (non-copyright) from the
  R$^7$RS report: “We intend this report to belong to the entire
  Scheme com- munity, and so we grant permission to copy it in whole
  or in part without fee. In particular, we encourage implementers of
  Scheme to use this report as a starting point for manuals and other
  documentation, modifying it as necessary.”. The individual SRFI
  comments can be included as a list, along with a statement that says
  that each author has explicitly granted permission for free use of
  their work.}

Permission is hereby granted, free of charge, to any person obtaining a
copy of this software and associated documentation files (the
``Software''), to deal in the Software without restriction, including
without limitation the rights to use, copy, modify, merge, publish,
distribute, sublicense, and/or sell copies of the Software, and to
permit persons to whom the Software is furnished to do so, subject to
the following conditions:

The above copyright notice and this permission notice shall be included
in all copies or substantial portions of the Software.

THE SOFTWARE IS PROVIDED ``AS IS'', WITHOUT WARRANTY OF ANY KIND,
EXPRESS OR IMPLIED, INCLUDING BUT NOT LIMITED TO THE WARRANTIES OF
MERCHANTABILITY, FITNESS FOR A PARTICULAR PURPOSE AND NONINFRINGEMENT.
IN NO EVENT SHALL THE AUTHORS OR COPYRIGHT HOLDERS BE LIABLE FOR ANY
CLAIM, DAMAGES OR OTHER LIABILITY, WHETHER IN AN ACTION OF CONTRACT,
TORT OR OTHERWISE, ARISING FROM, OUT OF OR IN CONNECTION WITH THE
SOFTWARE OR THE USE OR OTHER DEALINGS IN THE SOFTWARE.
