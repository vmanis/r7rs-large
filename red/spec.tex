% Cross-reference items here to the Red Edition specification on
% http://trac.sacrideo.us/wg/wiki/RedEdition. This command will
% be made null in future edits. 
\newcommand{\redno}[1]{%
  \textbf{[Red Edition item #1]}}

%%% This replaces the version of \includesrfi that's provided for
%%% working with the raw SRFIs, as produced by pandoc. NB:
%%% LaTeX's \include command refuses to nest, so I use the original
%%% TeX \input command.
\newcommand{\libname}{???Unknown???}
\newcommand{\includesrfi}[1]{%
  \input srfi-edited/srfi-#1.tex
  \renewcommand{\libname}{???Unknown???}}

%%% Handling library names.
\newcommand{\setlibname}[1]{%
  The names described in this section comprise the \texttt{(scheme
    #1)} library.
  \renewcommand{\libname}{#1}}
% Modern practice is to include a return type in procedure
% descriptions. Turn off the old proto macros, and include new ones
% that take an additional argument for the return type. New names are
% used, rather than the ones in R7RS, to avoid confusion & misuse. 
% Return value spec, make null if no return value types to be shown.
\newcommand{\protoret}[1]{%
  \,\ding{213}\,\textrm{\textit{#1}\/}}  % Best-looking arrow I could find. 
% Parenthesized prototype
\newcommand{\Proto}[4]{%
  \pproto{(\mainschindex{#1}\hbox{#1}{\it#2\/})%
    \protoret{#4}}{\libname\ library #3}}

% Variable prototype: no type argument just now, maybe later.
\newcommand{\Vproto}[2]{\mainschindex{#1}\pproto{#1}{\libname #2}}

% Extending an existing definition (\proto without the index entry)
\newcommand{\Rproto}[4]{%
  \pproto{(\hbox{#1}{\it#2\/})%
    \protoret{#4}}{\libname\ library #3}}

% (Eventually) Turn off the old ones, to avoid mixing them up.
\renewcommand{\proto}[3]{\errmessage{This shouldn't be used.}}
\renewcommand{\vproto}[3]{\errmessage{This shouldn't be used.}}
\renewcommand{\rproto}[3]{\errmessage{This shouldn't be used.}}


% Some of the SRFIs have icons relating to validated XHTML or the
% like. Provide an almost-null version of graphics inclusion that
% flags these, so they can be removed during proofreading. And for all
% I know, some SRFI may actually have a picture in it!
\renewcommand{\includegraphics}[2][]{%
  \textbf{Included graphics: #2}}

\TODO{Restore 2-column layout, once LaTeX artifacts have been fixed.}

\chapter*{Summary}
[This needs to be rewritten.]

This is a frozen version of the Red Edition ballot.

By unanimous consent, the libraries of R7RS-small are all required in
implementations of the Red Edition. This document specifies the
additional libraries required in the Red Edition. 




\todo{expand the summary so that it fills up the column.}

\vfill
\eject

\chapter*{Contents}
\addvspace{3.5pt}                  % don't shrink this gap
\renewcommand{\tocshrink}{-3.5pt}  % value determined experimentally
{\footnotesize
\tableofcontents
}

\vfill
\eject


Programming languages should be designed not by piling feature on top of
feature, but by removing the weaknesses and restrictions that make additional
features appear necessary.  Scheme demonstrates that a very small number
of rules for forming expressions, with no restrictions on how they are
composed, suffice to form a practical and efficient programming language
that is flexible enough to support most of the major programming
paradigms in use today.

The Revised$^7$ Report on The Algorithmic Language Scheme (R$^7$RS)
describes an elegant and powerful language that fulfills the promise
of the previous paragraph. In some cases, e.g., when using Scheme as
an extension language, that is all the programmer needs. Often,
though, modern programmers expect a set of libraries that will provide
the facilities they need to get the job done. Many Scheme
implementations have provided such libraries, but often the libraries
are  implementation-specific.  It would be desirable to
standardize as many of them as possible.

Accordingly, the Scheme community embarked upon the collection of a
group of Scheme Requests For Implementation, or SRFIs. These specify
particular libraries that can be provided in a compatible way across
implementations. 

More recently, it was decided that R$^7$RS would be a core language,
dubbed~“R$^7$RS-Small”, and that a Large language would be created
by identifying particular SRFIs, and adapting them to form a
consistent library collection.

This report, the Red Edition, is the first stage of defining
R$^7$RS-Large, concentrating primarily upon data structure
support. Subsequent reports, also named for colors, will provide
additional facilities.

\TODO{Check all descriptions to make sure that argument references are
  in appropriate font.}



\chapter{Basic libraries}\label{Basiclibraries}

\includesrfi{1}

\includesrfi{133}

\section{String library}
\redno{3} As no consensus was achieved on a string library, no string
library is provided in the Red Edition; one might be provided later.

This section is left in as a placeholder, and will be removed later in
the editing process.
 
\includesrfi{132}

\chapter{Sets and maps}\label{Setsandmaps}

\includesrfi{113}

\includesrfi{14}

\includesrfi{125}

\chapter{Immutability}\label{Immutability}

\includesrfi{116}

\includesrfi{101}

\includesrfi{134}

\includesrfi{135}

\chapter{Laziness}

\includesrfi{121}

\includesrfi{127}

\includesrfi{41}

\chapter{Miscellaneous}

\includesrfi{111}

\includesrfi{117}

\includesrfi{124}

\section{Titlecase library}
\redno{18} The vote was for no titlecase library. 

This section is left in as a placeholder, and will be removed later in
the editing process.

\includesrfi{128}
