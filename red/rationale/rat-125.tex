\section{Hash tables}

Hash tables themselves don't really need defending: almost all
dynamically typed languages, from awk to JavaScript to Lua to Perl to
Python to Common Lisp, and including many Scheme implementations,
provide them in some form as a fundamental data structure. Therefore,
what needs to be defended is not the data structure but the procedures.
This SRFI is at an intermediate level. It supports a great many
convenience procedures on top of the basic hash table interfaces
provided by \href{http://srfi.schemers.org/srfi-69/srfi-69.html}{SRFI
69} and
\href{http://www.r6rs.org/final/html/r6rs-lib/r6rs-lib-Z-H-14.html}{R6RS}.
Nothing in it adds power to what those interfaces provide, but it does
add convenience in the form of pre-debugged routines to do various
common things, and even some things not so commonly done but useful.

There is no mandated support for thread safety, immutability, or
weakness, though there are portable hooks for specifying these features.

This specification accepts separate equality predicates and hash
functions for backward compatibility, but strongly encourages the use of
\href{http://srfi.schemers.org/srfi-128/srfi-128.html}{SRFI 128}
comparators, which package a type test, an equality predicate, and a
hash function into a single bundle.

\subsubsection{SRFI 69 compatibility}\label{SRFI69compatibility}

This SRFI is downward compatible with SRFI 69. Some procedures have been
given new preferred names for compatibility with other SRFIs, but in all
cases the SRFI 69 names have been retained as deprecated synonyms;
implementations must still provide them, however. They appear in this
SRFI in small print.

There is one absolute incompatibility with SRFI 69: the reflective
procedure \texttt{hash-table-hash-function} may return \texttt{\#f},
which is not permitted by SRFI 69. See the specification for details.

\subsubsection{R6RS compatibility}\label{R6RScompatibility}

The relatively few hash table procedures in R6RS are all available in
this SRFI under somewhat different names. The only substantive
difference is that R6RS \texttt{hashtable-values} and
\texttt{hashtable-entries} return vectors, whereas in this SRFI
\texttt{hash-table-values} and \texttt{hash-table-entries} return lists.
This SRFI adopts SRFI 69's term \texttt{hash-table} rather than R6RS's
\texttt{hashtable}, because of the universal use of ``hash table''
rather than ``hashtable'' in other computer languages and in technical
prose generally. Besides, the English word \emph{hashtable} obviously
means something that can be \ldots{} hashted.

In addition, the \texttt{hashtable-ref} and \texttt{hashtable-update!}
of R6RS correspond to the \texttt{hash-table-ref/default} and
\texttt{hash-table-update!/default} of both SRFI 69 and this SRFI.

It would be trivial to provide the R6RS names on top of this SRFI.

\subsubsection{Common Lisp compatibility}\label{CommonLispcompatibility}

As usual, the Common Lisp names are completely different from the Scheme
names. Common Lisp provides the following capabilities that are not in
this SRFI:

\begin{itemize}
\tightlist
\item
  The constructor allows specifying the rehash size and rehash threshold
  of the new hash table. There are also accessors and mutators for these
  and for the current capacity (as opposed to size).
\end{itemize}

\begin{itemize}
\tightlist
\item
  There are hash tables based on \texttt{equalp} (which does not exist
  in Scheme).
\end{itemize}

\begin{itemize}
\tightlist
\item
  \texttt{With-hash-table-iterator} is a hash table external iterator
  implemented as a local macro.
\end{itemize}

\begin{itemize}
\tightlist
\item
  \texttt{Sxhash} is a implementation-specific hash function for the
  \texttt{equal} predicate. It has the property that objects in
  different instantiations of the same Lisp implementation that are
  \href{http://www.lispworks.com/documentation/lw61/CLHS/Body/03_bdbb.htm}{similar},
  a concept analogous to \texttt{equal} but defined across all
  instantiations, always return the same value from \texttt{sxhash}; for
  example, the symbol \texttt{xyz} will have the same \texttt{sxhash}
  result in all instantiations.
\end{itemize}

\subsubsection{Sources}\label{Sources}

The procedures in this SRFI are drawn primarily from SRFI 69 and R6RS.
In addition, the following sources are acknowledged:

\begin{itemize}
\tightlist
\item
  The \texttt{hash-table-mutable?} procedure and the second argument of
  \texttt{hash-table-copy} (which allows the creation of immutable hash
  tables) are from R6RS, renamed in the style of this SRFI.
\end{itemize}

\begin{itemize}
\tightlist
\item
  The \texttt{hash-table-intern!} procedure is from
  \href{http://docs.racket-lang.org/reference/hashtables.html}{Racket},
  renamed in the style of this SRFI.
\end{itemize}

\begin{itemize}
\tightlist
\item
  The \texttt{hash-table-find} procedure is a modified version of
  \texttt{table-search} in
  \href{http://www.iro.umontreal.ca/~gambit/doc/gambit-c.html\#Tables}{Gambit}.
\end{itemize}

\begin{itemize}
\tightlist
\item
  The procedures \texttt{hash-table-unfold} and
  \texttt{hash-table-count} were suggested by
  \href{http://srfi.schemers.org/srfi-1/srfi-1.html}{SRFI 1}.
\end{itemize}

\begin{itemize}
\tightlist
\item
  The procedures \texttt{hash-table=?} and \texttt{hash-table-map} were
  suggested by
  \href{http://hackage.haskell.org/packages/archive/containers/0.5.2.1/doc/html/Data-Map-Strict.html}{Haskell's
  Data.Map.Strict module}.
\end{itemize}

\begin{itemize}
\tightlist
\item
  The procedure \texttt{hash-table-map-\textgreater{}list} is from
  \href{http://www.gnu.org/software/guile/manual/html_node/Hash-Table-Reference.html}{Guile}.
\end{itemize}

The procedures \texttt{hash-table-empty?},
\texttt{hash-table-empty-copy}, \texttt{hash-table-pop!},
\texttt{hash-table-map!}, \texttt{hash-table-intersection!},
\texttt{hash-table-difference!}, and \texttt{hash-table-xor!} were added
for convenience and completeness.

The native hash tables of
\href{http://web.mit.edu/scheme_v9.0.1/doc/mit-scheme-ref/Hash-Tables.html}{MIT},
\href{http://sisc-scheme.org/manual/html/ch09.html\#Hashtables}{SISC},
\href{http://www-sop.inria.fr/indes/fp/Bigloo/doc/bigloo-7.html\#Hash-Tables}{Bigloo},
\href{http://s48.org/0.57/manual/s48manual_44.html}{Scheme48},
\href{http://www.cs.indiana.edu/scheme-repository/SCM/slib_2.html\#SEC13}{SLIB},
\href{http://www.rscheme.org/rs/b/0.7.3.4/5/html/c2143.html}{RScheme},
\href{https://ccrma.stanford.edu/software/snd/snd/s7.html\#hashtables}{Scheme
7},
\href{https://github.com/barak/scheme9/blob/master/lib/hash-table.scm}{Scheme
9},
\href{http://www.fifi.org/cgi-bin/info2www?(librep)Hash+Tables}{Rep},
and
\href{https://code.google.com/p/femtolisp/wiki/APIReference}{FemtoLisp}
were also investigated, but no additional procedures were incorporated.
