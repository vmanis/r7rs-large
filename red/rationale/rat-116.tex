\section{{Rationale}}\label{rationale}

The first question about this library is why it should exist at all. Why
not simply eliminate mutability from Scheme's ordinary pairs and use a
version of SRFI-1 that treats the linear-update procedures (with
\texttt{!}) as identical to their functional counterparts, as Racket
does? The main answer is that this approach breaks
R\textsuperscript{5}RS and R\textsuperscript{7}RS-small. All the data
structures in these versions of Scheme are inherently mutable, and
portable code is allowed to depend on that property.

R\textsuperscript{6}RS segregates \texttt{set-car!} and
\texttt{set-cdr!} into a separate library, thus allowing implementations
to provide immutable Scheme pairs if this library is not (transitively)
imported into a program, and mutable ones if it is. However, it is not
possible to write portable R\textsuperscript{6}RS programs that
differentiate between mutable and immutable pairs, for example by using
immutable pairs most of the time and mutable pairs where necessary.

Because of the Liskov Substitution Principle, it is not possible to
treat mutable pairs as either a subtype or a supertype of mutable ones;
they must be distinct, and if operations are to apply to both, they can
do so only by \emph{ad hoc} polymorphism of the kind that Scheme
traditionally avoids for several reasons, including clarity, efficiency,
and flexibility. This proposal, therefore, treats mutable and immutable
pairs separately, while allowing easy conversion from one to the other.

Rather than attempting to design this library from scratch, I have
chosen the conservative option of modifying
\href{http://srfi.schemers.org/srfi-1/srfi-1.html}{SRFI 1}.
Consequently, most of the rationale given in that document applies to
this one as well. I have made the following changes:

\begin{itemize}
\tightlist
\item
  Removed all linear-update procedures ending in \texttt{!}
\item
  Removed all references to circular lists (there will be a future SRFI
  for immutable bidirectional cycles).
\item
  Removed the O(n\textsuperscript{2}) lists-as-sets procedures (there
  will be a future SRFI supporting O(log n) immutable sets).
\item
  Inserted an \texttt{i} at a judicious place in each identifier,
  usually at the beginning. However, because ``icons'' means something
  else in both ordinary English and computer jargon, the basic
  constructor and its immediate relatives are named \texttt{ipair},
  \texttt{xipair} and \texttt{ipair*} instead.
\item
  Added procedures for conversion between ordinary and immutable pairs,
  lists, and trees.
\item
  Added an analogue of \texttt{apply} for applying a procedure to an
  immutable list of arguments.
\item
  Added SRFI 114 comparators for immutable pairs, lists, and dotted
  lists.
\end{itemize}

Note: In the prose, immutable pairs and lists are known as ipairs and
ilists throughout.

\subsection{{General discussion}}

A set of general criteria guided the design of the SRFI-1 library that
underlies this library. They are reproduced here.

List-filtering procedures such as \texttt{ifilter} or \texttt{idelete}
do not disorder lists. Elements appear in the answer list in the same
order as they appear in the argument list. This constrains
implementation, but seems like a desirable feature, since in many uses
of lists, order matters. (In particular, disordering an association list
is definitely a bad idea.)

Contrariwise, although the sample implementations of the list-filtering
procedures share longest common tails between argument and answer lists,
it is not part of the spec.

Because ilists are an inherently sequential data structure (unlike, say,
vectors), inspection procedures such as \texttt{ifind},
\texttt{ifind-tail}, \texttt{ifor-each}, \texttt{iany} and
\texttt{ievery} commit to a left-to-right traversal order of their
argument list.

However, constructors, such as \texttt{ilist-tabulate} and the mapping
procedures (\texttt{iappend-map}, \texttt{ipair-for-each},
\texttt{ifilter-map}, \texttt{imap-in-order}), do \emph{not} specify the
dynamic order in which their procedural argument is applied to its
various values.

Predicates return useful true values wherever possible. Thus
\texttt{iany} must return the true value produced by its predicate, and
\texttt{ievery} returns the final true value produced by applying its
predicate argument to the last element of its argument list.

No special status is accorded Scheme's built-in equality predicate. Any
functionality provided in terms of \texttt{eq?}, \texttt{eqv?},
\texttt{equal?} is also available using a client-provided equality
predicate.

These procedures are \emph{not} generic as between ordinary pairs/lists
and immutable pairs/lists; they are specific to immutable lists. Like
Olin, I prefer to keep the library simple and focused. However, there
are a few conversions between mutable and immutable lists provided.

\subsection{{Improper Lists}}\label{improper-lists}

Scheme does not properly have a list type, just as C does not have a
string type. Rather, Scheme has a binary-tuple type, from which one can
build binary trees. There is an \emph{interpretation} of Scheme values
that allows one to treat these trees as lists. The same interpretation
is applied to immutable pairs.

Because the empty list, written as \texttt{()}, is already immutable, it
is shared between mutable and immutable lists as the termination marker.
It is the only Scheme object that is both a mutable list and an
immutable list.

Users should note that dotted lists, whether mutable or immutable, are
not commonly used, and are considered by many Scheme programmers to be
an ugly artifact of Scheme's lack of a true list type. Dotted ilists are
\emph{not} fully supported by this SRFI. Most procedures are defined
only on proper ilists --- that is, \texttt{()}-terminated ilists. The
procedures that will also handle dotted ilists are specifically marked.
While this design decision restricts the domain of possible arguments
one can pass to these procedures, it has the benefit of allowing the
procedures to catch the error cases where programmers inadvertently pass
scalar values to an ilist procedure by accident, \emph{e.g.}, by
switching the arguments to a procedure call.
