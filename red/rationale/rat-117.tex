The list queues described in this SRFI are objects of a disjoint type
that provide an ordered sequence of arbitrary Scheme objects. Like
lists, they provide sequential access to their elements; unlike lists,
they normally should not share storage with other list queues, unless
special precautions have been taken.

Technically, a queue is an object that makes it efficient to remove
elements from the front and to add elements to the back. List queues
also make it efficient to add elements to the front, but removing
elements from the back is inefficient. Therefore, because these objects
do not provide the normal behavioral guarantees of
\href{http://en.wikipedia.org/wiki/Double-ended_queue}{deques}, they are
not referred to as deques. True deques will be provided in a future
SRFI.

Historically, objects of this form were called ``tconc structures''
(where ``tconc'' is short for ``tail concatenate''), and Lispers have
tended to roll their own using pairs. This version uses a record to hold
the first-pair and last-pair pointers rather than a pair, but uses pairs
and the empty list internally.

Because the representation of list queues as lists is exposed by the
implementation, it's not necessary to provide a comprehensive API for
list queues. Instead,
\href{http://srfi.schemers.org/srfi-1/srfi-1.html}{SRFI 1} and other
list APIs can serve the same purpose, using the conversion procedures to
convert between list queues and lists fairly cheaply. Consequently, the
API provided here over and above the bare necessities of queueing and
dequeueing elements is roughly analogous to the R7RS-small API for
lists. It also subsumes the
\href{http://wiki.call-cc.org/man/4/Unit\%20data-structures\#queues}{Chicken}
and
\href{http://people.csail.mit.edu/jaffer/slib/Queues.html\#Queues}{​SLIB}
APIs.
