\section{SRFI 135}

The \protect\hyperlink{R6RS-Rationale}{R6RS Rationale} identified
problems created by the mutability of strings, and several more problems
were mentioned by SRFI 130 or came up during its discussion period:

\begin{itemize}
\tightlist
\item
  Mutability complicates the specification of higher-order procedures
  that operate on strings.
\item
  Mutability inhibits several compiler optimizations, including common
  subexpression elimination.
\item
  Mutability complicates reasoning about programs that use strings.
\item
  Mutations invalidate the string cursors of SRFI 130.
\item
  Using a SRFI 130 string cursor that has been invalidated by mutation
  is an error, but that error is likely to go undetected, making
  programs harder to test and to debug.
\item
  Mutations can be expensive if strings are represented as encapsulated
  UTF-8 or UTF-16.
\item
  Although representations based on UTF-32 provide fast referencing as
  well as fast mutation, they occupy more space than representations
  based on UTF-8 or UTF-16.
\item
  Mutations preclude sharing of substructure that could save space while
  making \texttt{substring} and \texttt{string-append} run faster.
\end{itemize}

Recognizing the first three of these problems, while acknowledging that
removing mutable strings from the language would cause ``significant
compatibility problems for existing code''
\protect\hyperlink{R6RS-Rationale}{{[}R6RS-Rationale{]}}, the R6RS
standard banished \texttt{string-set!} and \texttt{string-fill!} to a
separate \texttt{(rnrs\ mutable-strings)} library in hope of
discouraging and/or deprecating mutation of strings.

The R7RS restored \texttt{string-set!} and \texttt{string-fill!} to the
\texttt{(scheme\ base)} library and added a new mutator,
\texttt{string-copy!}. Waiting for some revised standard to make strings
immutable is not viable.

We can, however, add a new data type of immutable texts capable of
replacing Scheme's string data type for all applications that do not
require mutation. Immutable texts do away with the problems listed above
while offering these advantages over mutable strings:

\begin{itemize}
\tightlist
\item
  space efficiency approaching that of UTF-8 or UTF-16
\item
  faster sequential access (if strings use UTF-8 or UTF-16)
\item
  faster random access (if strings use UTF-8 or UTF-16)
\item
  fast extraction of subtexts
\item
  faster concatenation of texts
\end{itemize}

SRFI 130 aims for the second of those advantages, but cannot achieve
that advantage in any portable implementation of its procedures.
Furthermore SRFI 130 introduces a new data type (cursors) that is hard
to use correctly (because its error situations are likely to go
undetected, partly because many of its operations accept both cursors
and character indexes, which are allowed but not required to be the same
things).

This SRFI offers all five advantages, at the expense of introducing a
new data type (immutable texts) that can be implemented portably and
efficiently and is easy to use correctly (partly because most of its
error situations are detected by its sample implementations, and partly
because its character indexes are the same as those used by Scheme's
mutable strings).

See also the \protect\hyperlink{Unicode}{discussion of Unicode}.

This SRFI is based upon SRFI 130, copying much of its structure and
wording, which should make it easier to compare this SRFI against SRFI
130 and to convert programs using SRFI 130 to use immutable texts
instead.

