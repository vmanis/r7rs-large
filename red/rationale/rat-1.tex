\section{list library rationale}

The set of basic list and pair operations provided by
R4RS/\protect\hyperlink{R5RS}{R5RS} Scheme is far from satisfactory.
Because this set is so small and basic, most implementations provide
additional utilities, such as a list-filtering function, or a ``left
fold'' operator, and so forth. But, of course, this introduces
incompatibilities -- different Scheme implementations provide different
sets of procedures.

I have designed a full-featured library of procedures for list
processing. While putting this library together, I checked as many
Schemes as I could get my hands on. (I have a fair amount of experience
with several of these already.) I missed Chez -- no on-line manual that
I can find -- but I hit most of the other big, full-featured Schemes.
The complete list of list-processing systems I checked is:

R4RS/\protect\hyperlink{R5RS}{R5RS} Scheme, MIT Scheme, Gambit, RScheme,
MzScheme, slib, \protect\hyperlink{CommonLisp}{Common Lisp}, Bigloo,
guile, T, APL and the SML standard basis

As a result, the library I am proposing is fairly rich.

Following this initial design phase, this library went through several
months of discussion on the SRFI mailing lists, and was altered in light
of the ideas and suggestions put forth during this discussion.

In parallel with designing this API, I have also written a reference
implementation. I have placed this source on the Net with an
unencumbered, ``open'' copyright. A few notes about the reference
implementation:

\begin{itemize}
\tightlist
\item
  Although I got procedure names and specs from many Schemes, I wrote
  this code myself. Thus, there are \emph{no} entanglements. Any Scheme
  implementor can pick this library up with no worries about copyright
  problems -- both commercial and non-commercial systems.
\item
  The code is written for portability and should be trivial to port to
  any Scheme. It has only four deviations from R4RS, clearly discussed
  in the comments:

  \begin{itemize}
  \tightlist
  \item
    Use of an \texttt{error} procedure;
  \item
    Use of the \protect\hyperlink{R5RS}{R5RS} \texttt{values} and a
    simple \texttt{receive} macro for producing and consuming multiple
    return values;
  \item
    Use of simple \texttt{:optional} and \texttt{let-optionals} macros
    for optional argument parsing and defaulting;
  \item
    Use of a simple \texttt{check-arg} procedure for argument checking.
  \end{itemize}
\item
  It is written for clarity and well-commented. The current source is
  768 lines of source code and 826 lines of comments and white space.
\item
  It is written for efficiency. Fast paths are provided for common
  cases. Side-effecting procedures such as \texttt{filter!} avoid
  unnecessary, redundant \texttt{set-cdr!}s which would thrash a
  generational GC's write barrier and the store buffers of fast
  processors. Functions reuse longest common tails from input parameters
  to construct their results where possible. Constant-space iterations
  are used in preference to recursions; local recursions are used in
  preference to consing temporary intermediate data structures.

  This is not to say that the implementation can't be tuned up for a
  specific Scheme implementation. There are notes in comments addressing
  ways implementors can tune the reference implementation for
  performance.
\end{itemize}

In short, I've written the reference implementation to make it as
painless as possible for an implementor -- or a regular programmer -- to
adopt this library and get good results with it.
