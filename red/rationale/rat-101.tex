\section{{Issues}}\label{issues}

\begin{itemize}
\tightlist
\item
  Procedure names have been chosen to be consistent with R6RS, even
  though in some cases such as \texttt{list-ref} and \texttt{list-tail}
  the choice seems poor since they include the prefix \texttt{list-}
  even though they do not operate on lists, but chains of pairs, i.e.
  lists and improper lists, and arbitrary objects, respectively.
  Although the names have remained the same, the descriptions have been
  corrected (e.g. using \emph{pair} or \emph{obj} instead of \emph{list}
  for parameter names). Should the names be changed as well?
\item
  To what extent should standard Scheme procedures and syntax that
  consume or construct lists be included in this proposal? For example,
  should all of the \texttt{(rnrs\ base)} library that deals with lists
  be included? By my count this would mean adding: \texttt{lambda},
  \texttt{apply}, \texttt{vector-\textgreater{}list},
  \texttt{list-\textgreater{}vector},
  \texttt{string-\textgreater{}list}, and
  \texttt{list-\textgreater{}string}. I am inclined to add these. Should
  all of the \texttt{(rnrs\ \ \ \ \ lists)} library be included? These
  procedures are easily defined in terms of what's given here, and no
  perfomance advantage is gained by implementig them ``under the hood''
  using the data structures in the reference implementation. I am
  inclined \emph{not} to include them.
\item
  Should a \texttt{car+cdr} procedure be added?
\item
  Should the current \texttt{syntax} and \texttt{procedures}
  sub-libraries be included?
\end{itemize}

\section{{Rationale}}\label{rationale}

Functional programming and list hacking go together like peanut butter
and jelly, eval and apply, syntax and semantics, or cursing and
recursing. But the traditional approach to implementing pairs and lists
results in index-based access (\texttt{list-ref}) requiring time
proportional the index being accessed. Moreover, indexed-based
functional update (\texttt{list-set}) becomes so inefficient as to be
nearly unspeakable. Instead, programmers revert the imperatives of the
state; they use a stateful data structure and imperative algorithms.

This SRFI intends to improve the situation by offering an alternative
implementation strategy based on Okasaki's purely functional
random-access lists {[}\protect\hyperlink{note-1}{1}{]}. Random-access
pairs and lists can be used as a replacement for traditional,
linear-access pairs and lists with no asymptotic loss of efficiency. In
other words, the typical list and pair operations such as \texttt{cons},
\texttt{car}, and \texttt{cdr}, all operate in \emph{O(1)} time as
usual. However, random-access lists additionally support index-based
access and functional update operations that are asymptotically cheaper;
\emph{O(log(n))} for random-access lists versus \emph{O(n)} for
linear-access lists, where \emph{n} is the length of the list being
access or updated. As such, many purely functional index-based list
algorithms become feasible by using a random-access list representation
for pairs and lists.

The requirements of this SRFI have been designed in such a way as to
admit portable library implementations of this feature, such as the
reference implementation, while at the same time admit more radical
implementations that embrace random-access pairs as the fundamental pair
representation.

