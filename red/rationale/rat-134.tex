\section{Immutable Deques}\label{rationale}

A double-ended queue, or \emph{deque} (pronounced ``deck'') is a
sequential data structure which allows elements to be added or removed
from either end in amortized O(1) time. It is a generalization of both a
queue and a stack, and can be used as either by disregarding the
irrelevant procedures.

This SRFI describes immutable deques, or \emph{ideques}. Immutable
structures are sometimes called \emph{persistent} and are closely
related to \emph{pure functional} (a.k.a. \emph{pure}) structures. The
availability of immutable data structures facilitates writing efficient
programs in the pure-functional style. Immutable deques can also be seen
as a bidirectional generalization of immutable lists, and some of the
procedures documented below are most useful in that context. Unlike the
immutable lists of
\href{http://srfi.schemers.org/srfi-116/srfi-116.html}{SRFI 116}, it is
efficient to produce modified versions of an ideque; unlike the list
queues of \href{http://srfi.schemers.org/srfi-117/srfi-117.html}{SRFI
117}, it is possible to efficiently return an updated version of an
ideque without mutating any earlier versions of it.

The specification was designed jointly by Kevin Wortman and John Cowan.
John Cowan is the editor and shepherd. The two-list implementation was
written by John Cowan.
